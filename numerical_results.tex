The main goal of this section is twofold. First, we show that the FE
discretization of the streamfunction formulation of the QGE \eqref{eqn:FEForm}
with the Argyris element produces accurate numerical approximations. To this
end, we benchmark our numerical results against those in the published
literature \cite{Vallis06,Cascon,Myers}. The second goal of this section is to
show that the numerical results follow the theoretical error estimates in
\autoref{thm:EnergyNorm} and \autoref{thm:Errors}.

The rest of the section is organized as follows: In \autoref{sec:Method} we
discuss the procedure for implementation for our numerical investigations which
includes a brief discussion on mesh refinement, node numbering, the Argyris
finite element, numerical schemes, and the method for error calculation;
Finally, in \autoref{sec:Tests} we present a careful numerical investigation of
the FE discretization of the streamfunction formulation of the QGE
\eqref{eqn:FEForm} with the Argyris element. The numerical results are
benchmarked against those in the published literature. The rates of convergence
of the FE discretization are compared to the theoretical rates in
\autoref{thm:EnergyNorm} and \autoref{thm:Errors}.

%In particular, we describe the specific transformation used in mapping the
%physical element to the reference element, which is essential for an efficient
%implementation. In Section \ref{sec:Newton}, we briefly describe the
%implementation details of the Newton method used to solve the nonlinear system
%of equations resulting from the FE discretization of the streamfunction
%formulation of the QGE \eqref{eqn:FEForm}. The global numbering utilized in the
%numerical experiments, which is important for the computational efficiency of
%the discretization, is the same as that presented in \cite{Fairag}.
