The main goal of this section is to develop a rigorous numerical analysis for
the FEM discretization of the QGE \eqref{eqn:SQGEFEF} by using the conforming
Argyris element.  First, in Theorem~\ref{thm:EnergyNorm} we prove error
estimates in the $H^2$ norm by using an approach similar to that used in
\cite{Cayco86}.  Second, in Theorem~\ref{thm:Errors}, we prove error estimates
in the $L^2$ and $H^1$ norms by using a duality argument.

First, we must introduce some bounds of the forms introduced in
\autoref{sec:QGEStrong}.
\begin{lemma} \label{lma:ContinuousForms}
  The linear form $(F,\chi)$, the bilinear forms $(\nabla \psi, \nabla \chi)$,
  $(\Delta \psi, \Delta \chi)$, $(\psi_x, \chi)$ and the trilinear form $b(\psi;
  \psi, \chi)$ are continuous \cite{Cayco86}: There exists a $\Gamma_0,\,
   \Gamma_2 > 0$ such that forall $\psi, \chi, \varphi\in X$
  \begin{align}
    (\nabla \psi, \nabla \chi) &\le \Gamma_0\, |\psi|_2 |\chi|_2,
      \label{eqn:a0cont} \\
    (\Delta \psi, \Delta \chi) &\le |\psi|_2\, |\chi|_2, \label{eqn:a1Cont} \\
    b(\psi;\varphi,\chi) &\le \Gamma_1 \|\psi\|_2\, \|\varphi\|_2\, \|\chi\|_2,
      \label{eqn:BH2Bounds} \\
    (\psi_x,\chi) &\le \Gamma_2 \, |\psi|_2 \, |\chi|_2, \label{eqn:a3Cont} \\
    (F,\chi) &\le \|F\|_{-2} \, |\chi|_2.
      \label{eqn:lCont}
  \end{align}
\end{lemma}
\begin{proof}
  We first start with the simplest proof, that being the proof for
  \eqref{eqn:a1Cont}. This follows directly from the H\"older inequality, i.e.
  \begin{equation*}
    (\Delta \psi, \Delta \chi) \le \|\Delta \psi\|_{L^p}\,\|\Delta \chi\|_{L^r},
    \quad \frac{1}{p} + \frac{1}{r} = 1
  \end{equation*}
  with $p=r=2$. \\
  The bound \eqref{eqn:lCont} can be proven by first noting that
  \begin{equation*}
    \frac{(F,\chi)}{|\chi|_2} \le \sup_{\chi \in X} \frac{(F,\chi)}{|\chi|_2}.
  \end{equation*}
  Multiplying both sides by $|\chi|_2$ gives the desired result.
  \begin{equation*}
    (F,\chi) \le \|F\|_{-2}|\chi|_2.
  \end{equation*}
  Next, we prove \eqref{eqn:a0cont} by first applying H\"older inequality with
  $p=r=2$ and therefore
  \begin{equation*}
    (\nabla \psi, \nabla \chi) \le \|\nabla \psi\|\, \|\nabla \chi\|.
  \end{equation*}
  Now, we apply the Poincar\'e inequality to get the desired result
  \begin{equation*}
    (\nabla \psi, \nabla \chi) \le \Gamma_0 |\psi|_2, |\chi|_2,
  \end{equation*}
  where $\Gamma_0$ is the Poincar\'e constant. \\
  Next, we will prove \eqref{eqn:a3Cont} by first applying the H\"older
  inequality to get
  \begin{equation*}
    (\psi_x, \chi) \le \|\psi_x\|\, \|\chi\|.
  \end{equation*}
  Noting that $\|\psi_x\| \le \|\nabla \psi\|$ and the applying the Poincar\'e
  inequality once to the term involving $\psi$ and twice to the term involving
  $\chi$ gives the desired result
  \begin{equation*}
    (\psi_x, \chi) \le \Gamma_2\,|\psi|_2\, |\chi|_2,
  \end{equation*}
  where $\Gamma_2$ is a multiplication of three Poincar\'e constants. \\
  Finally, for \eqref{eqn:BH2Bounds} we start by applying the H\"older inequality
  \begin{equation}
    b(\psi;\varphi,\chi) \le \|\Delta \psi\|_{L^p} \|\nabla \varphi\|_{L^q}
      \|\nabla \chi\|_{L^r},\quad \frac{1}{p}+\frac{1}{q}+\frac{1}{r}=1,
    \label{eqn:HolderB}
  \end{equation}
  where $1\le p,\,q,\,r\le \infty$. Thus, setting $p = 2$, and $q = r = 4$ gives
  \begin{equation*}
    b(\psi;\varphi,\chi) \le \|\Delta \psi\| \|\nabla \varphi\|_{L^4} \|\nabla
      \chi\|_{L^4}.
  \end{equation*}
  Then applying the Ladyzhenskaya inequality results in
  \begin{equation*}
    b(\psi;\varphi,\chi) \le \Gamma \|\Delta \psi\|
      \|\nabla \varphi\|^{\nicefrac{1}{2}} \|\Delta \varphi\|^{\nicefrac{1}{2}}
      \|\nabla \chi\|^{\nicefrac{1}{2}} \|\Delta \chi\|^{\nicefrac{1}{2}}.
  \end{equation*}
  Using the Poincar\'e inequality gives
  \begin{equation*}
    b(\psi;\varphi,\chi) \le \Gamma_1 \|\Delta \psi\|\, \|\Delta \varphi\|\,
      \|\Delta \chi\|.
  \end{equation*}
  where $\Gamma_1$ is the Poincar\'e constant.
\end{proof}

\begin{thm}
\label{thm:EnergyNorm}
  Let $\psi$ be the solution of \eqref{eqn:SQGEWF} and $\psi^h$ be the solution
  of \eqref{eqn:SQGEFEF}.
  Furthermore, assume that the following small data condition is satisfied:
  \begin{equation}
    Re^{-2} \, Ro \geq \Gamma_1 \, \| F \|_{-2} ,
    \label{eqn:small_data_condition}
  \end{equation}
  where $Re$ is the Reynolds number defined in \eqref{eqn:reynolds_number}, $Ro$
  is the Rossby number defined in \eqref{eqn:rossby_number}, $\Gamma_1$ is the
  continuity constant of the trilinear form $a_2$ in \eqref{eqn:BH2Bounds}, and $F$
  is the forcing term. Then the following error estimate holds:
  \begin{equation}
    |\psi - \psi^h|_2 \le C(Re, Ro, \Gamma_1, \Gamma_2, F) \,
      \inf_{\chi^h \in X^h} |\psi - \chi^h|_2 ,
    \label{eqn:EnergyNorm}
  \end{equation}
  where
  \begin{equation}
    C(Re, Ro, \Gamma_1, \Gamma_2, F)
      := \frac{ Ro^{-1} \Gamma_2 + 2 \, Re^{-1} + \Gamma_1 \, Re \, Ro^{-1} \,
        \| F \|_{-2} }
        { Re^{-1} - \Gamma_1 \, Re \, Ro^{-1} \, \| F \|_{-2} }
    \label{eqn:constant_definition}
  \end{equation}
  is a generic constant that can depend on $Re$, $Ro$, $\Gamma_1$, $F$, but
  \emph{not} on the meshsize $h$.
\end{thm}

\begin{remark}
  Note that the small data condition in Theorem~\ref{thm:EnergyNorm} involves
  \emph{both} the Reynolds number and the Rossby number, the latter quantifying
  the rotation effects in the QGE.

  Furthermore, note that the standard small data condition $Re^{-2} \geq
  \Gamma_1 \, \| F \|_{-2}$ used to prove the uniqueness for the steady-state 2D
  NSE \cite{Girault79,Girault86,Layton08} is significantly more restrictive for
  the QGE, since \eqref{eqn:small_data_condition} has the Rossby number (which
  is small when rotation effects are significant) on the left-hand side.  This
  is somewhat counterintuitive, since in general rotation effects are expected
  to help in proving the well-posedness of the system.  We think that the
  explanation for this puzzling situation is the following: Rotation effects do
  make the mathematical analysis of 3D flows more amenable by giving them a 2D
  character.  We, however, are concerned with 2D flows (the QGE).  In this case,
  the small data condition \eqref{eqn:small_data_condition} (needed in proving
  the uniqueness of the solution) indicates that rotation effects make the
  mathematical analysis of the (2D) QGE more complicated than that of the 2D
  NSE. %We emphasize that there is no contradiction, since, although the
  %rotation effects do help in the three-dimensional case, they seem to
  %complicate the situation if one compare the QGE with the two-dimensional (not
  %the three-dimensional) NSEi.
\end{remark}
\begin{proof}
  Since $X^h \subset X$, \eqref{eqn:SQGEWF} holds for all $\chi = \chi^h\in X^h$.
  Subtracting \eqref{eqn:SQGEFEF} from \eqref{eqn:SQGEWF} with $\chi=\chi^h \in
  X^h$ gives
  \begin{equation}
    \begin{split}
      Re^{-1}(\Delta \left[\psi - \psi^h\right],\Delta \chi^h)
        + b(\psi,\psi,\chi^h) - b(\psi^h,\psi^h,\chi^h) \\
        - Ro^{-1} (\left[\psi-\psi^h\right]_x,\chi^h) = 0 \qquad \forall \chi^h \in
    X^h.
  \end{split}
    \label{eqn:ErrorEq}
  \end{equation}
  Next, adding and subtracting $b(\psi^h,\psi,\chi^h)$ to \eqref{eqn:ErrorEq},
  we get:
  \begin{equation}
    \begin{split}
      & Re^{-1} (\Delta \left[\psi - \psi^h\right], \Delta \chi^h)
        + b(\psi,\psi,\chi^h) - b(\psi^h,\psi,\chi^h) %\\
        + b(\psi^h,\psi,\chi^h) - b(\psi^h,\psi^h,\chi^h) \\
      & \hspace*{3.0cm} - Ro^{-1} (\left[\psi-\psi^h\right]_x,\chi^h) = 0
        \qquad \forall \chi^h \in X^h.
    \end{split}
    \label{eqn:interError}
  \end{equation}
  The error $e$ can be decomposed as
  \begin{equation}
    e:= \psi-\psi^h = (\psi-\lambda^h)+(\lambda^h-\psi^h):= \eta + \Phi^h,
    \label{eqn:ErrorTrick}
  \end{equation}
  where $\lambda^h\in X^h$ is arbitrary.
  Thus, equation \eqref{eqn:interError} can be
  rewritten as
  \begin{equation}
    \begin{split}
      Re^{-1}(\Delta \eta + \Delta \Phi^h, \Delta \chi^h)
        + b(\eta+\Phi^h,\psi,\chi^h) + b(\psi^h,\eta+\Phi^h,\chi^h) \\
      + Ro^{-1} (\eta_x+\Phi^h_x,\chi^h) = 0 \qquad \forall \chi^h \in X^h.
    \end{split}
    \label{eqn:ErrorEta}
  \end{equation}
  Letting $\chi^h := \Phi^h$ in \eqref{eqn:ErrorEta}, we obtain
  \begin{equation}
    \begin{split}
      Re^{-1} (\Delta \Phi^h, \Delta \Phi^h) - Ro^{-1} (\Phi^h_x,\Phi^h)
        = -Re^{-1} (\Delta \eta, \Delta \Phi^h)
        - b(\eta;\psi,\Phi^h) - b(\psi^h;\psi,\Phi^h) \\
      - b(\psi^h;\eta,\Phi^h) - b(\psi^h;\Phi^h,\Phi^h)
        + Ro^{-1} (\eta,\Phi^h).
    \end{split}
    \label{eqn:ErrorVarphi}
  \end{equation}
  Note that, since $(\Phi^h_x,\Phi^h)=-(\Phi^h,\Phi^h_x)\; \forall
  \Phi^h \in X^h \subset X = H^2_0$, it follows that
  \begin{equation}
    (\Phi^h_x,\Phi^h)=0 .
    \label{eqn:a30}
  \end{equation}
  Also, it follows immediately from \eqref{eqn:b} that
  \begin{equation}
    b(\psi^h,\Phi^h,\Phi^h)=0 .
    \label{eqn:b0}
  \end{equation}
  Combining \eqref{eqn:b0}, \eqref{eqn:a30}, and \eqref{eqn:ErrorVarphi}, we get:
  \begin{equation}
    \begin{split}
      Re^{-1} (\Delta \Phi^h, \Delta \Phi^h) = -Re^{-1} (\Delta \eta, \Delta \Phi^h)
        - b(\Phi^h;\psi,\Phi^h) - b(\eta;\psi,\Phi^h) \\
      - b(\psi^h;\eta,\Phi^h) + R)^{-1} (\eta_x,\Phi^h).
    \end{split}
    \label{eqn:ErrorZeroed}
  \end{equation}
  Using
  \begin{equation*}
    (\Delta \Phi^h, \Delta \Phi^h) = |\Phi^h|^2_2
  \end{equation*}
  and inequalities \eqref{eqn:a1Cont} -- \eqref{eqn:a3Cont} in equation
  \eqref{eqn:ErrorZeroed} gives
  \begin{equation}
    \begin{split}
      Re^{-1} \, |\Phi^h|^2_2 \le Re^{-1} \,  |\eta|_2 \, |\Phi^h|_2 + \Gamma_1
        \biggl( |\eta|_2 \, |\psi|_2 \, |\Phi^h|_2 + |\psi^h|_2 \, |\eta|_2 \,
          |\Phi^h|_2 \biggr) \\
      + \Gamma_1 \, |\Phi^h|^2_2 \, |\psi|_2 + Ro^{-1} \, |\eta|_2 \, |\Phi^h|_2 .
    \end{split}
    \label{eqn:varphiIneq}
  \end{equation}
  Simplifying and rearranging terms in \eqref{eqn:varphiIneq} gives
  \begin{equation}
      |\Phi^h|_2 \le \left( Re^{-1} - \Gamma_1 \, | \psi |_2 \right)^{-1} \,
        \left( Re^{-1} + \Gamma_1 \, |\psi|_2 + \Gamma_1 \, |\psi^h|_2 + Ro^{-1}
          \right) \, |\eta|_2 .
    \label{eqn:phihIneq}
  \end{equation}
  Using \eqref{eqn:phihIneq} and the triangle inequality along with the
  stability estimates \eqref{eqn:stability_sqge} and
  \eqref{eqn:stability_fem_sqge} gives
  \begin{align}
    |e|_2 &\le |\eta|_2 + |\Phi^h|_2 \nonumber \\[0.2cm]
    &\le \left[ 1 + \frac{Re^{-1} + \Gamma_1 \, |\psi|_2 + \Gamma_1 \,
      |\psi^h|_2 + Ro^{-1}} {Re^{-1} - \Gamma_1 \, |\psi|_2} \right] \, |\eta|_2
      \nonumber \\[0.2cm]
    &\le \left[ 1 + \frac{Re^{-1} + \Gamma_1 \, \left( Re \, Ro^{-1} \, \| F
      \|_{-2} \right) + \Gamma_1 \, \left( Re \, Ro^{-1} \, \| F \|_{-2} \right)
      + Ro^{-1}} {Re^{-1} - \Gamma_1 \, \left( Re \, Ro^{-1} \, \| F \|_{-2}
      \right) } \right] \, |\eta|_2 \nonumber \\
    &= \left[ \frac{ Ro^{-1} + 2 \, Re^{-1} + \Gamma_1 \, Re \, Ro^{-1} \,
      \| F \|_{-2} } { Re^{-1} - \Gamma_1 \, Re \, Ro^{-1} \, \| F \|_{-2} }
        \right] \, | \psi-\lambda^h |_2 ,
    \label{eqn:EnergyError}
  \end{align}
  where $\lambda^h \in X^h$ is arbitrary.  Taking the infimum over $\lambda^h \in
  X^h$ in \eqref{eqn:EnergyError} proves the error estimate
  \eqref{eqn:EnergyNorm}.
\end{proof}

In \autoref{thm:EnergyNorm}, we proved an error estimate in the $H^2$ norm.
In \autoref{thm:Errors}, we will prove error estimates in the $L^2$ and $H^1$
norms by using a duality argument. To this end, we first notice that the QGE
\eqref{eqn:QGE_psi} can be written as
\begin{equation}
  \mathcal{N} \, \psi = Ro^{-1} \, F ,
  \label{eqn:qge_operator_formulation}
\end{equation}
where the nonlinear operator $\mathcal{N}$ is defined as
\begin{equation}
  \mathcal{N} \, \psi := Re^{-1} \, \Delta^2 \psi + J(\psi , \Delta \psi)
    - Ro^{-1} \, \frac{\partial \psi}{\partial x} .
  \label{eqn:nonlinear_operator}
\end{equation}
The linearization of $\mathcal{N}$ around $\psi$, a solution of
\eqref{eqn:QGE_psi}, yields the following \emph{linear} operator:
\begin{equation}
  \mathcal{L} \, \chi := Re^{-1} \, \Delta^2 \chi + J(\chi , \Delta \psi) +
    J(\psi, \Delta \chi) - Ro^{-1} \, \frac{\partial \chi}{\partial x} .
  \label{eqn:linear_operator}
\end{equation}
To find the dual problem associated with the QGE
\eqref{eqn:qge_operator_formulation}, we first define the \emph{dual operator}
$\mathcal{L}^*$ of $\mathcal{L}$:
\begin{equation}
  (\mathcal{L} \, \chi , \psi^*) = ( \chi , \mathcal{L}^* \, \psi^*)
    \qquad \forall \, \psi^* \in X .
  \label{eqn:dual_operator}
\end{equation}
To find $\mathcal{L}^*$, we use the standard procedure:
In \eqref{eqn:dual_operator}, we use the definition of $\mathcal{L}$ given in \eqref{eqn:linear_operator} and we ``integrate by parts" (i.e., use Green's theorem):
\begin{align}
  (\mathcal{L} \, \chi , \psi^*) =& \left( Re^{-1} \, \Delta^2 \chi
    + J(\chi , \Delta \psi) + J(\psi, \Delta \chi)
    - Ro^{-1} \, \frac{\partial \chi}{\partial x} \, , \, \psi^* \right) \nonumber \\
  =& \left( \chi \, , \, Re^{-1} \, \Delta^2 \, \psi^* - J(\psi , \Delta \psi^* )
    + Ro^{-1} \, \frac{\partial \psi^*}{\partial x} \right)
    + \biggl( J(\chi , \Delta \psi) , \psi^* \biggr) ,
  \label{eqn:dual_operator_1}
\end{align}
where to get the first term on the right-hand side of
\eqref{eqn:dual_operator_1} we used the skew-symmetry of the trilinear form
$a_2$ in the last two variables and Green's theorem (just as we did in the proof
of Theorem \ref{thm:stability_sqge}).
Next, we apply Green's theorem to the second term on the right-hand side of
\eqref{eqn:dual_operator_1}:
\begin{align}
  \biggl( J(\chi , \Delta \psi) , \psi^* \biggr) =& \chi_x \, \Delta \psi_y \, \psi^*
     - \chi_y \, \Delta \psi_x \, \psi^* \nonumber \\
  \stackrel{Green}{=}& - \chi \, \Delta \psi_{y x} \, \psi^* - \chi \, \Delta
    \psi_y \, \psi^*_x + \chi \, \Delta \psi_{x y} \, \psi^* + \chi \, \Delta
    \psi_x \, \psi^*_y \nonumber \\
  =& \biggl( \chi , J(\Delta \psi , \psi^*) \biggr) .
  \label{eqn:dual_operator_2}
\end{align}
Equations \eqref{eqn:dual_operator_1}-\eqref{eqn:dual_operator_2} imply:
\begin{align}
  (\mathcal{L} \, \chi , \psi^*) =& \left( \chi \, , \, Re^{-1} \, \Delta^2 \, \psi^*
    - J(\psi , \Delta \psi^* ) + Ro^{-1} \, \frac{\partial \psi^*}{\partial x} \right)
    + \biggl( \chi , J(\Delta \psi , \psi^*) \biggr) \nonumber \\
  =& ( \chi , \mathcal{L}^* \, \psi^*) .
  \label{eqn:dual_operator_3}
\end{align}
Thus, the \emph{dual operator} $\mathcal{L}^*$ is given by
\begin{equation}
  \mathcal{L}^* \, \psi^* = Re^{-1} \, \Delta^2 \, \psi^* - J(\psi , \Delta \psi^* )
    + J(\Delta \psi , \psi^* ) + Ro^{-1} \, \frac{\partial \psi^*}{\partial x} .
  \label{eqn:dual_operator_4}
\end{equation}
For any given $g \in L^2(\Omega)$, the weak formulation of the \emph{dual
problem} is:
\begin{equation}
  ( \mathcal{L}^* \, \psi^* , \chi ) = (g , \chi)
    \qquad \forall \, \chi \in X = H_0^2(\Omega) .
  \label{eqn:dual_operator_5}
\end{equation}
We assume that $\psi^*$, the solution of \eqref{eqn:dual_operator_5}, satisfies
the following elliptic regularity estimates:
\begin{align}
  & \psi^* \in H^4(\Omega) \cap H^2_0(\Omega), \label{eqn:dual_operator_6a} \\[0.2cm]
  & \| \psi^* \|_4 \le C \, \|g\|_{0}, \label{eqn:dual_operator_6b} \\[0.2cm]
  & \| \psi^* \|_3 \le C \, \|g\|_{-1}, \label{eqn:dual_operator_6c}
\end{align}
where $C$ is a generic constant that can depend on the data, but not on the
meshsize $h$.
\begin{remark}
  We note that this type of elliptic regularity was also assumed in
  \cite{Cayco86} for the streamfunction formulation of the 2D NSE.  In that
  report, it was also noted that for a polygonal domain satisfying a minimum
  angle condition, Rannacher et al. \cite{} had actually proved this elliptic
  regularity.
\end{remark}

The error estimates in the $L^2$ and $H^1$ norms that we prove in
Theorem~\ref{thm:Errors} are derived for the particular space $X^h\subset
H^2_0(\Omega)$ consisting of Argyris elements, although the same results can be
derived for other conforming $C^1$ finite element spaces.
\begin{thm} \label{thm:Errors}
  Let $\psi$ be the solution of \eqref{eqn:SQGEWF} and $\psi^h$ be the solution
  of \eqref{eqn:SQGEFEF}.  Assume that the same small data condition as in
  Theorem \ref{thm:EnergyNorm} is satisfied:
  \begin{equation}
    Re^{-2} \, Ro \geq \Gamma_1 \, \| F \|_{-2} .
    \label{eqn:small_data_condition_dual}
  \end{equation}
  Furthermore, assume that $\psi\in H^6(\Omega) \cap H^2_0(\Omega)$.  Then there
  exist positive constants $C_0, \, C_1 \text{ and } C_2$ that can depend on
  $Re$, $Ro$, $\Gamma_1$, $F$, but \emph{not} on the meshsize $h$, such that
  \begin{align}
    |\psi - \psi^h|_2 &\le C_2 \, h^4 , \label{eqn:H2Error} \\
    |\psi - \psi^h|_1 &\le C_1 \, h^5 , \label{eqn:H1Error} \\
    \|\psi - \psi^h\|_0 &\le C_0 \, h^6 . \label{eqn:L2Error}
  \end{align}
\end{thm}
\begin{proof}
  Estimate \eqref{eqn:H2Error} follows immediately from
  \eqref{eqn:argyris_approximation_0} and Theorem \ref{thm:EnergyNorm}.
  Estimates \eqref{eqn:L2Error} and \eqref{eqn:H1Error} follow from a duality
  argument.

  The error in the primal problem \eqref{eqn:SQGEWF} and the interpolation error
  in the dual problem  \eqref{eqn:dual_operator_5} are denoted as
  \begin{equation}
    e := \psi - \psi^h \qquad e^* : = \psi^* - {\psi^*}^h ,
    \label{eqn:theorem_dual_1}
  \end{equation}
  respectively.

  We start by proving the $L^2$ norm estimate \eqref{eqn:L2Error}.
  \begin{align}
    |e|^2 = (e , e) =& (\mathcal{L} \, e , \psi^*)
      = (e , \mathcal{L}^* \, \psi^*) \nonumber \\
    =& (e , \mathcal{L}^* \, e^*) + (e , \mathcal{L}^* \, {\psi^*}^h)
      = (\mathcal{L} \, e , e^*) + (\mathcal{L} \, e , {\psi^*}^h) .
    \label{eqn:theorem_dual_2}
  \end{align}
  The last term on the right-hand side of \eqref{eqn:theorem_dual_2} is given by
  \begin{equation}
    (\mathcal{L} \, e , {\psi^*}^h) = \left( Re^{-1} \, \Delta^2 e
      + J(e , \Delta \, \psi) + J(\psi , \Delta \, e)
      - Ro^{-1} \, \frac{\partial e}{\partial x} \, , \, {\psi^*}^h \right) .
    \label{eqn:theorem_dual_3}
  \end{equation}
  To estimate this term, we consider the error equation obtained by subtracting
  \eqref{eqn:SQGEFEF} (with $\psi^h= {\psi^*}^h$) from \eqref{eqn:SQGEWF} (with
  $\chi = {\psi^*}^h$):
  \begin{equation}
    \left( Re^{-1} \, \Delta^2 e - Ro^{-1} \, \frac{\partial e}{\partial x} \, ,
      \, {\psi^*}^h \right) + \left( J(\psi , \Delta \, \psi)
      - J(\psi^h , \Delta \, \psi^h) \, , \, {\psi^*}^h \right) = 0 .
    \label{eqn:theorem_dual_4}
  \end{equation}
  Using \eqref{eqn:theorem_dual_4}, equation \eqref{eqn:theorem_dual_3} can be
  written as follows:
  \begin{equation}
    (\mathcal{L} \, e , {\psi^*}^h) = \left( J(e , \Delta \, \psi)
      + J(\psi , \Delta \, e) - J(\psi , \Delta \, \psi)
      + J(\psi^h , \Delta \, \psi^h) \, , \, {\psi^*}^h \right) .
    \label{eqn:theorem_dual_5}
  \end{equation}
  Thus, by using \eqref{eqn:theorem_dual_5} equation \eqref{eqn:theorem_dual_2}
  becomes:
  \begin{align}
    |e|^2 =& (\mathcal{L} \, e , e^*) + (\mathcal{L} \, e , {\psi^*}^h) \nonumber \\
    =& Re^{-1} \, (\Delta e , \Delta e^*)
      - Ro^{-1} \, \left( \frac{\partial e}{\partial x} , e^* \right)
      + \left( J(e , \Delta \psi) + J(\psi , \Delta e) , e^* \right) \nonumber \\
    +& \left( J(e , \Delta \, \psi) + J(\psi , \Delta \, e)
      - J(\psi , \Delta \, \psi) + J(\psi^h , \Delta \, \psi^h)
      \, , \, {\psi^*}^h \right) \nonumber \\
    =& Re^{-1} (\Delta e ,\Delta e^*) - Ro^{-1}(e_x , e^*)
      + b(e , \psi , e^*) + b(\psi , e, e^*) + b(e , \psi , {\psi^*}^h) \nonumber \\
    & - b(\psi , \psi , e^*) + b(\psi^h , \psi^h , e^*) \nonumber \\
    =& Re^{-1} (\Delta , \Delta e^*) - Ro^{-1}(e_x , e^*)
      + b(e , \psi , e^*) + b(\psi , e, e^*) \nonumber \\
    & - b(e , \psi , e^*) + b(e , \psi^h , e^*) + b(e , e , \psi^*)
    \label{eqn:theorem_dual_6}
  \end{align}
  Using the bounds in \eqref{eqn:a1Cont}-\eqref{eqn:a3Cont},
  \eqref{eqn:theorem_dual_6} yields:
  \begin{align}
    |e|^2 =& Re^{-1} (\Delta e ,\Delta e^*) - Ro^{-1} (e_x , e^*)
      + b(e , \psi , e^*) + b(\psi , e, e^*) \nonumber \\
    & - b(e , \psi , e^*) + b(e , \psi^h , e^*) + b(e , e , \psi^*) \nonumber \\
    \leq& Re^{-1} \, | e |_2 \, |e^* |_2 + Ro^{-1} \, | e |_2 \, |e^* |_2
      + \Gamma_1 \, | e |_2 \, | \psi |_2 \, | e^* |_2
      + \Gamma_1 \, | \psi |_2 \, | e |_2 \, | e^* |_2 \nonumber \\
    & + \Gamma_1 \, | e |_2 \, | \psi |_2 \, | e^* |_2
      + \Gamma_1 \, | e |_2 \, | \psi^h |_2 \, | e^* |_2
      + \Gamma_1 \, | e |_2 \, | e |_2 \, | \psi^* |_2 \nonumber \\
    =& | e |_2 \, |e^* |_2 \, \left( Re^{-1} + Ro^{-1} + \Gamma_1 \, | \psi |_2
      + \Gamma_1 \, | \psi |_2
      + \Gamma_1 \, | \psi |_2
      + \Gamma_1 \, | \psi^h |_2 \right) \nonumber \\
    & + | e |_2^2 \, \left( \Gamma_1 \, | \psi^* |_2 \right) .
    \label{eqn:theorem_dual_7}
  \end{align}
  We start bounding the terms on the right-hand side of
  \eqref{eqn:theorem_dual_7}.  First, we note that using the stability estimates
  \eqref{eqn:stability_sqge} for $\psi$ and \eqref{eqn:stability_fem_sqge} for
  $\psi^h$, the right-hand side of \eqref{eqn:theorem_dual_7} can be bounded as
  follows:
  \begin{equation}
    |e|^2 \leq C \, | e |_2 \, |e^* |_2 + | e |_2^2 \,
      \left( \Gamma_1 \, | \psi^* |_2 \right) ,
    \label{eqn:theorem_dual_8}
  \end{equation}
  where $C$ is a generic constant that can depend on $Re$, $Ro$, $\Gamma_1$,
  $F$, but \emph{not} on the meshsize $h$.  By using the approximation results
  \eqref{eqn:argyris_approximation_1}, we get:
  \begin{equation}
    |e^* |_2 \leq C \, h^2 \, | \psi^* |_4 .
    \label{eqn:theorem_dual_9}
  \end{equation}
  By using \eqref{eqn:dual_operator_6a} and \eqref{eqn:dual_operator_6b}, the
  elliptic regularity results of the dual problem \eqref{eqn:dual_operator_5}
  with $g := e$, we also get:
  \begin{equation}
    | \psi^* |_4 \leq C \, | e | ,
    \label{eqn:theorem_dual_10}
  \end{equation}
  which obviously implies
  \begin{equation}
    | \psi^* |_2 \leq C \, | e | .
    \label{eqn:theorem_dual_11}
  \end{equation}
  Inequalities \eqref{eqn:theorem_dual_9}-\eqref{eqn:theorem_dual_10} imply:
  \begin{equation}
    |e^* |_2 \leq C \, h^2 \, | e | .
    \label{eqn:theorem_dual_12}
  \end{equation}
  Inserting \eqref{eqn:theorem_dual_11} and \eqref{eqn:theorem_dual_12}  in
  \eqref{eqn:theorem_dual_8}, we get:
  \begin{equation}
    |e|^2 \leq C \, h^2 \, | e |_2 \, | e | + C \, | e |_2^2 \, | e | .
    \label{eqn:theorem_dual_13}
  \end{equation}
  Using the obvious simplifications and the $H^2$ error estimate
  \eqref{eqn:H2Error} in \eqref{eqn:theorem_dual_13} yields:
  \begin{equation}
    |e| \leq C \, h^2 \, | e |_2 + C \, | e |_2^2 \leq C \, h^6 + C \, h^8
      = C_0 \, h^6 ,
    \label{eqn:theorem_dual_14}
  \end{equation}
  which proves the $L^2$ error estimate \eqref{eqn:L2Error}.

  Next, we prove the $H^1$ norm estimate \eqref{eqn:H1Error}.  Since the duality
  approach we use is similar to that we used in proving the  $L^2$ norm estimate
  \eqref{eqn:L2Error}, we only highlight the main differences.  We start again
  by writing the $H^1$ norm of the error in terms of the dual operator
  $\mathcal{L}^*$:
  \begin{align}
    |e|_1^2 =& (\nabla e , \nabla e) = ( e , - \Delta e) =
      (\mathcal{L} \, e , \psi^*) = (e , \mathcal{L}^* \, \psi^*) \nonumber \\
    =& (e , \mathcal{L}^* \, e^*) + (e , \mathcal{L}^* \, {\psi^*}^h)
      = (\mathcal{L} \, e , e^*) + (\mathcal{L} \, e , {\psi^*}^h) .
    \label{eqn:theorem_dual_15}
  \end{align}
  Thus, the second and fourth equalities in \eqref{eqn:theorem_dual_15} clearly
  indicate that in the dual problem \eqref{eqn:dual_operator_5}, one should
  choose $g = - \Delta e$, and not $g = e$, as we did in
  \eqref{eqn:theorem_dual_10}, when we proved the $L^2$ error estimate
  \eqref{eqn:L2Error}.  Using \eqref{eqn:dual_operator_6a} and
  \eqref{eqn:dual_operator_6c}, the elliptic regularity results of the dual
  problem \eqref{eqn:dual_operator_5} with $g :=  - \Delta e$, we also get:
  \begin{equation}
    | \psi^* |_3 \leq C \, | - \Delta e |_{-1} \leq C \, | e |_1 ,
    \label{eqn:theorem_dual_16}
  \end{equation}
  where in the last inequality we used the fact that $e \in H_0^2(\Omega)$.
  Inequality \eqref{eqn:theorem_dual_16} obviously implies
  \begin{equation}
    | \psi^* |_2 \leq C \, | e |_1 .
    \label{eqn:theorem_dual_17}
  \end{equation}
  All the results in \eqref{eqn:theorem_dual_3}-\eqref{eqn:theorem_dual_7} carry
  over to our setting.  Thus, we get:
  \begin{equation}
    |e|_1^2 \leq C \, | e |_2 \, |e^* |_2 + C \, | e |_2^2 \, | \psi^* |_2 ,
    \label{eqn:theorem_dual_18}
  \end{equation}
  where $C$ is a generic constant that can depend on $Re$, $Ro$, $\Gamma_1$,
  $F$, but \emph{not} on the meshsize $h$.  By using the approximation result
  \eqref{eqn:argyris_approximation_2}, we get:
  \begin{equation}
    |e^* |_2 \leq C \, h \, | \psi^* |_3 .
    \label{eqn:theorem_dual_19}
  \end{equation}
  Inequalities \eqref{eqn:theorem_dual_19}-\eqref{eqn:theorem_dual_16} imply:
  \begin{equation}
    |e^* |_2 \leq C \, h \, | e |_1 .
    \label{eqn:theorem_dual_20}
  \end{equation}
  Inserting \eqref{eqn:theorem_dual_17} and \eqref{eqn:theorem_dual_20}  in
  \eqref{eqn:theorem_dual_18}, we get:
  \begin{equation}
    |e|_1^2 \leq C \, h \, | e |_2 \, | e |_1 + C \, | e |_2^2 \, | e |_1 .
    \label{eqn:theorem_dual_21}
  \end{equation}
  Using the obvious simplifications and the $H^2$ error estimate
  \eqref{eqn:H2Error} in \eqref{eqn:theorem_dual_21} yields:
  \begin{equation}
    |e|_1 \leq C \, h \, | e |_2 + C \, | e |_2^2 \leq C \, h^5 + C \, h^8
      = C_1 \, h^5 ,
    \label{eqn:theorem_dual_22}
  \end{equation}
  which proves the $H^1$ error estimate \eqref{eqn:H1Error}.
\end{proof}
