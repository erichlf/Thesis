We are now ready to derive the strong formulation of the QGE in streamfunction
form, \eqref{eqn:QGE_psi}. To this end, it is necessary to introduce the
appropriate functional setting.  Let $\psi$ be a strong solution \cite{Layton08}
of the QGE, \eqref{eqn:QGE_psi}, then
\begin{equation}
  \psi \in L^{\infty}(0, T; L^2(\Omega)) \cap L^2(0, T;H^2_0(\Omega))
  \label{eqn:StrongSpace}
\end{equation}
where
\begin{align*}
  L^2(0, T;H^2_0(\Omega)) &:= \left\{ \psi(t,\mathbf{x}):[0, T] \to H^2_0(\Omega):
    \int_{0}^{T}\! \|\Delta \psi\| \, dt < \infty \right\} \\
  L^{\infty}(0, T;L^2(\Omega)) &:= \left\{ \psi(t,\mathbf{x}):[0, T] \to L^2(\Omega):
    \ess \sup_{0<t<T} \|\psi\| < \infty\right\}.
\end{align*}
With the functional setting in place we can derive the strong formulaiton for
the QGE, \eqref{eqn:QGE_psi}.

Multiplying \eqref{eqn:QGE_psi} by a test function $\chi \in X$ and then
transferring some of the derivatives from $\psi$ to $\chi$ by use of the
divergence theorem , we get in a standard way the \emph{strong formulation} of
the QGE in streamfunction formulation:
\begin{equation*}
  \int_{\Omega}\! \left(-\frac{\partial \left[ \Delta \psi \right]}{\partial t} + Re^{-1}
    \Delta^2 \psi +  J(\psi,\Delta \psi) - Ro^{-1}\frac{\partial \psi}{\partial
    x}\right)\, \chi \, d\mathbf{x} = Ro^{-1} \int_{\Omega}\! F\, \chi \, d\mathbf{x}
\end{equation*}
We first rewrite the Jacobian term, from \eqref{eqn:QGE_psi} in a more useful
form
\begin{align*}
  -\frac{\partial}{\partial t} \int_{\Omega}\! \Delta \psi \, \chi \, d\mathbf{x}
    &+ Re^{-1} \int_{\Omega}\! \Delta^2 \psi \, d\mathbf{x}
    - \int_{\Omega}\! \left(\frac{\partial\left[\Delta\psi \right]}{\partial x}\, \frac{\partial
      \psi}{\partial y} -
    \frac{\partial\left[ \Delta\psi \right]}{\partial y}\,
    \frac{\partial \psi}{\partial x}\right)\, \chi\, d\mathbf{x} \\
    &- Ro^{-1} \int_{\Omega}\! \frac{\partial \psi}{\partial x}\, \chi \, d\mathbf{x}
    = Ro^{-1} \int_{\Omega}\! F\, \chi \, d\mathbf{x},
\end{align*}
which yields the following equation:
\begin{equation*}
  -\frac{\partial}{\partial t} \int_{\Omega}\! \Delta \psi \, \chi \, d\mathbf{x}
    + Re^{-1} \int_{\Omega}\! \Delta^2 \psi \, d\mathbf{x}
    - \int_{\Omega}\! \left(\nabla \left[\Delta\psi\right]\cdot \nabla \times
    \psi \right)\, \chi\,
    d\mathbf{x} -Ro^{-1} \int_{\Omega}\! \frac{\partial \psi}{\partial x}\,
    \chi \, d\mathbf{x} = Ro^{-1} \int_{\Omega}\! F\, \chi \, d\mathbf{x}
\end{equation*}
For the remaining portion of the strong formulation we shall deal with each
summand individually (from left to right) and so
\begin{align*}
  -\frac{\partial}{\partial t} \int_{\Omega}\! \Delta \psi\, \chi \, d\mathbf{x} &=
    -\frac{\partial}{\partial t} \int_{\Omega}\! \nabla \left( \nabla \psi \, \chi \right) -
    \nabla \psi \cdot \nabla \chi \, d\mathbf{x} \\
  &= -\frac{\partial}{\partial t} \cancelto{0}{\int_{\partial \Omega}\! \nabla \psi \, \chi
    \cdot \mathbf{n} \, dS} + \frac{\partial}{\partial t} \int_{\Omega}\! \nabla
    \psi\, \nabla \chi \, d\mathbf{x}.
\end{align*}
The next summand gives
\begin{align*}
  Re^{-1}\int_{\Omega}\! \Delta^2 \psi\, \chi \, d\mathbf{x} &= Re^{-1}\int_{\Omega}\! \nabla\left[
    \nabla^3 \psi \, \chi \right] + \nabla^3 \psi\, \nabla \chi \, d\mathbf{x} \\
  &= Re^{-1}\cancelto{0}{\int_{\partial\Omega}\, \nabla^3 \psi\, \chi\cdot \mathbf{n}
    \, dS} - Re^{-1}\int_{\Omega}\! \nabla\left[ \Delta\psi \nabla \chi \right] - \Delta
    \psi \Delta \chi \, d\mathbf{x} \\
  &= -Re^{-1} \cancelto{0}{\int_{\partial\Omega}\, \Delta\psi \nabla \chi \cdot
    \mathbf{n}\, dS} + Re^{-1} \int_{\Omega} \Delta \psi \Delta \chi \,
    d\mathbf{x}.
\end{align*}
The third and final summand that needs to be evaluated gives
\begin{align*}
  -\int_{\Omega}\! \nabla\left[ \Delta\psi\right]\cdot
    \nabla\times\psi \, \chi \, d\mathbf{x} &= -\int_{\Omega}\!
    \nabla\left[\Delta\psi\right]\cdot \nabla\times\psi \, \chi\, d\mathbf{x}\\
  &= -\int_{\Omega}\! \nabla\left[ \Delta\psi \cdot \nabla\times\psi\, \chi
    \right] -\Delta\psi\, \cancelto{0}{\nabla\left[ \nabla\times\psi \chi
    \right]} \\
      &\qquad - \Delta\psi\, \nabla\times\psi \cdot \nabla\chi \, d\mathbf{x} \\
  &= -\cancelto{0}{\int_{\partial\Omega}\, \Delta\psi \, \nabla\times\psi\, \chi
    \cdot \mathbf{n} \, dS} \\
    &\qquad + \int_{\Omega}\! \Delta \psi \, \nabla\times \psi \cdot
      \nabla \chi \, d\mathbf{x} \\
  &= \int_{\Omega}\! \Delta\psi\left( \psi_y\chi_x - \psi_x\chi_y \right)\, d\mathbf{x}.
\end{align*}
Thus, the weak form of the QGE in streamfunction form is
\begin{eqnarray}
  && \frac{\partial}{\partial t} \int_{\Omega}\! \nabla \psi\, \nabla \chi  \, d\mathbf{x}
    + Re^{-1} \, \int_{\Omega}\! \Delta \psi \, \Delta \chi \, d{\bf x} + \int_{\Omega}\! \Delta
    \zeta \, \left( \psi_y \, \chi_x - \psi_x \, \chi_y \right)\, d{\bf x} - Ro^{-1} \, \int_{\Omega}
    \, \psi_x \, \chi \, d{\bf x} \nonumber \\
  && \hspace*{3.0cm} = Ro^{-1}\, \int_{\Omega}\! F \, \chi \, d{\bf x} \qquad \forall \, \chi \in X .
\label{eqn:qge_psi_weak}
\end{eqnarray}
Therefore, taking $(\cdot,\cdot)$ to be the $L^2$ inner product and letting
\begin{equation}
  b(\psi; \psi, \chi) = \int_{\Omega}\! \Delta \psi\, (\psi_y \chi_x - \psi_x
  \chi_y)\, d\mathbf{x}
  \label{eqn:b}
\end{equation}
gives the following strong formulation of the QGE in streamfunction formulation:
\begin{equation}
  \begin{split}
    \text{Find } \psi &\in X \text{ such that} \\
    \frac{\partial}{\partial t} (\nabla \psi, \nabla \chi) + Re^{-1} (\Delta
      \psi, \Delta \chi) +& b(\psi,\psi,\chi) - Ro^{-1}(\psi_x,\chi)
      = Ro^{-1} (F,\chi),\quad \forall \chi \in X.
  \end{split}
  \label{eqn:QGEWF}
\end{equation}
