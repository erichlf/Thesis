Ocean currents have influenced humanity since the beginning of time, both
through the effects ocean currents have on climate, but also through the effects
they have on human travel. Vikings and Polynesians both made use of ocean
currents in their exploration \cite{Dijkstra08, Ingstad}. The Vikings used the
ocean currents to travel between Europe and the Americas \cite{Ingstad}. In
fact, the Vikings were able to sail all along the Greenland coast, because of a
lack of ice bergs caused by a more northerly reaching Gulf Stream, which receded
a century later \cite{Morner95}.

The first documentation of ocean currents were made by Christopher Columbus
(1492-1494), Vasco da Gama (1497-1499), and Ferdinand Magellan (1519-1522). In
fact, the Gulf Stream was first mapped by Benjamin Franklin in 1769
(\autoref{fig:Franklin}) \cite{Dijkstra08, Vallis06}. The first instrumental
observations of ocean currents on a global scale were begun by the HMS
\emph{Challenger} in 1872 when she set sail for over three years around the
world \cite{Siedler01}.

\begin{figure}%[H]
  \begin{center}
    \includegraphics[scale=0.5]{Franklin-Folger.png}
    \caption{Map of the Gulf Stream by Franklin-Folger, 1769}
    \label{fig:Franklin}
  \end{center}
\end{figure}

In more recent times we read of plans for shipping through the Northwestern
Passage, which until recently had been closed by sea ice. Numerical climate
models predicted that the passage would eventually open, however the passage has
opened much earlier than anticipated \cite{NatGeo}. Canada claims it has full
rights to the passage while the United States and Europe claim it is part of
international waters. On the other hand Russia has laid claims to the sea floor
in the artic and thereby raising tensions in the region. Clearly, climate change
can influence geopolitics.

The climate system is driven by energy from the sun. This solar energy is
transfered from the low latitudes to the high latitudes via radiative processes,
and oceanic and atmosphic circulations. The oceans make up approximately $71\%$
of the Earths surface, therefore the absorption of solar energy is dominated by
the oceans. Thus, climate variability is, to a large extent, an ocean-related
phenomenon \cite{Siedler01}. Because of this we see that ocean circulation
changes are strongly linked to changes in climate \cite{Morner95, Siedler01}. In
fact, the cold periods in Western Europe during the time periods 1440-1460,
1687-1703, and 1808-1821 can all be linked to a southward deflection of the Gulf
Stream and a southward penetration of Arctic cold water \cite{Morner95}.

%Two of the most important features of oceanic flows is the effects of rotation and stratification
%\cite{Majda}.

The large scale ocean flows are characterized by three major features: the wind
forcing, stratification, and the effects of rotation \cite{Majda, Vallis06}.
Annual mean wind patterns, where winds are westerward near the equator and
eastward at the midlatitudes \cite{Dijkstra08}, drive the subtropical and
subpolar gyres, which correspond to the strong, persistent, sub-tropical and
sub-polar western boundary currents: in the North Atlantic (the Gulf Stream and
the Labrador Current), in the South Atlantic (the Brazil and the Falkland
Currents), in North Pacific (the Kuroshio and the Oyashio Currents), in the
South Pacific (East Australia Current) and in the Indian Ocean (the Agulhas
Current) \cite{Dijkstra08,Vallis06}. These major ocean currents are dipicted in
\autoref{fig:Currents}. One of the common features of these gyres is that they
display strong western boundary currents, weaker interior flows, and weak
eastern boundary currents. Its these wind-driven ocean circulations that play a
significant role in climate dynamics \cite{Dijkstra05,Ghil08}.

\begin{figure}%[H]
  \begin{center}
    \includegraphics[scale=0.5]{Currents.pdf}
    \caption{Ocean Surface Currents}
    \label{fig:Currents}
  \end{center}
\end{figure}

