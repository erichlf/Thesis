For completeness we present the transformation developed by Dominguez et. al. in
this section. All notations and conventions are the same as those developed by
Dominguez, but we present the derivation of the transformation here, because of
the novelty and importance of this transformation to this thesis. For the
original derivation one should see \cite{Dominguez06} and \cite{Dominguez08}.
Now that we have the Argyris basis functions on the reference triangle we need
to relate those to the Argyris basis functions on a general triangle. First,
consider the affine transformation from $\hat{K}$ to $K$, i.e. $F: \hat{K} \to
K$ such that
\begin{equation}
  F(\hat{\mathbf{x}}) = B\hat{\mathbf{x}} + \mathbf{b} :=
  \begin{bmatrix}
    x_2 - x_1 & x_3 - x_1 \\ y_2 - y_1 & y_3 - y_1
  \end{bmatrix} \begin{bmatrix}
    \hat{x} \\ \hat{y}
  \end{bmatrix} + \begin{bmatrix}
    x_1 \\ y_1
  \end{bmatrix}.
  \label{eqn:Affine}
\end{equation}
Also denote the vectors representing the edge of the triangle as
\begin{equation*}
  \mathbf{v}_1 = \mathbf{x}_2 - \mathbf{x}_1\quad \mathbf{v}_2 = \mathbf{x}_3 -
  \mathbf{x}_1\quad \mathbf{v}_3 = \mathbf{x}_3 - \mathbf{x}_2.
\end{equation*}
Let $\mathbf{n}_i$ be the unit normal vector corresponding to the $i^{th}$ side
obtained by rotating the vector $\mathbf{v}_i$ by $\nicefrac{\pi}{2}$ in the
positive direction. Additionally, let $\mathbf{m}_i$ be the midpoint to the
$i^{th}$ side.  Now consider the linear functionals corresponding to the
vertices of the triangle $K$ denoted as
\begin{align*}
  \mathcal{L}_i(\varphi) &:= \varphi(\mathbf{x}_i), \\
  \mathcal{L}_i^\circ(\varphi) &:= \partial_\circ \varphi(\mathbf{x}_i) \quad
  \text{if }\circ \in
    \{x,y\}, \\
  \mathcal{L}_i^\circ(\varphi) &:= \partial_\circ \varphi(\mathbf{x}_i) \quad
  \text{if }\circ \in
    \{xx,xy,yy\}
\end{align*}
for $i\in \{1,2,3\}$. For the functionals corresponding to the sides define
\begin{equation*}
  \mathcal{L}_i^n(\varphi) := \nabla_{\mathbf{x}} \phi(\mathbf{m}_i) \cdot
  \mathbf{n}_i \quad i\in\{1,2,3\}.
\end{equation*}
Now renumber the linear functionals as $\mathcal{L}_j$ for
$j\in\{1,\dots,21\}$ as they are listed below
\begin{align*}
  &\mathcal{L}_1, \mathcal{L}_2, \mathcal{L}_3, \\
  &\mathcal{L}_1^x, \mathcal{L}_1^y, \mathcal{L}_2^x, \mathcal{L}_2^y,
    \mathcal{L}_3^x, \mathcal{L}_3^y, \\
  &\mathcal{L}_1^{xx}, \mathcal{L}_1^{xy}, \mathcal{L}_1^{yy},
    \mathcal{L}_2^{xx}, \mathcal{L}_2^{xy}, \mathcal{L}_2^{yy},
    \mathcal{L}_3^{xx}, \mathcal{L}_3^{xy}, \mathcal{L}_3^{yy}, \\
  &\mathcal{L}_1^n, \mathcal{L}_2^n, \mathcal{L}_3^n,
\end{align*}
where the linear functionals $\hat{\mathcal{L}}_i$ and $\hat{ \mathcal{L}_i^n}$
are the corresponding functionals on the reference triangle $\hat{K}$. The basis
functions of the Argyris triangle are fifth degree polynomials in
$\mathbb{P}_5(K)$
that satisfy
\begin{equation*}
  \mathcal{L}_i(\varphi_j) = \delta_{ij} \quad i,j\in\{1,\dots,21\}
\end{equation*}
and similarly on the reference triangle
\begin{equation*}
  \hat{\mathcal{L}}_i(\hat{\varphi}_j) = \delta_{ij} \quad
  i,j\in\{1,\dots,21\}.
\end{equation*}
Now define a new set of functionals by
\begin{equation}
  \tilde{\mathcal{L}}_i(\varphi) := \hat{\mathcal{L}}_i(\varphi\circ F).
  \label{eqn:Functional}
\end{equation}
Since $\{\tilde{\mathcal{L}}_i\}$ and $\{\hat{\mathcal{L}}_i\}$ are both basis
for the dual space of $\mathbb{P}_5(K)$ there is a nonsingular matrix $C$ such
that
\begin{equation}
  \tilde{\mathcal{L}}_i = \sum_{j=1}^{21} c_{ij} \mathcal{L}_j \quad i\in
  \{1,\dots,21\}.
  \label{eqn:FunctionalsC}
\end{equation}
Therefore, it follows that (see \cite{Dominguez08})
\begin{equation}
  \varphi_i\circ F = \sum_{j=1}^{21} c_{ij}\hat{\varphi}_j \quad i\in
  \{1,\dots,21\}.
  \label{eqn:PolyC}
\end{equation}
Thus we have shown there exists a $C$ which transforms any triangle $K$ to a
reference triangle $\hat{K}$.

Now, we must determine a simple expression for matrix $C$. To do this we
introduce a new set of linear functionals
\begin{equation*}
  \mathcal{L}^*_i \quad i\in\{1,\dots,24\}.
\end{equation*}
We definie these functionals in the following way: first we let
$\mathcal{L}^*_i:=\mathcal{L}_i\; i=\{1,\dots,18\}$, and then introduce the new
functionals
\begin{equation*}
  \mathcal{L}^\circ_i \quad \circ \in \{\perp,||\}\quad i\in \{1,2,3\}.
\end{equation*}
These functionals will be defined using the following relationship:
\begin{equation*}
  \mathcal{L}_i^{\perp}(\varphi) := \nabla_{\mathbf{x}} \varphi(\mathbf{m}_i)
    \cdot R\mathbf{v}_i, \quad
  \mathcal{L}^{||}_i (\varphi) := \nabla_{\mathbf{x}} \varphi(\mathbf{m}_i)
    \cdot \mathbf{v}_i,
\end{equation*}
where
\begin{equation*}
  R = \begin{bmatrix}
    0 & -1 \\ 1 & 0
  \end{bmatrix}
\end{equation*}
is the matrix that rotates a vector $\nicefrac{\pi}{2}$ counter-clockwise. These
relationships will define a relationship of the edge and firsts derivatives at a
midpoint to the directional derivatives, i.e. the normal derivative and the
derivative pointing in the direction of the corresponding edge.

Now, order the linear functionals as
$\mathcal{L}^{\perp}_1,\mathcal{L}^{\perp}_2,\mathcal{L}^{\perp}_3,
\mathcal{L}^{||}_1,\mathcal{L}^{||}_2,\mathcal{L}^{||}_3$.
We can write each $\tilde{\mathcal{L}}_i$ as a linear combination of
$\mathcal{L}_j^*$ and therefore we have the relationship
\begin{equation}
  \tilde{\mathcal{L}}_i = \sum_{j=1}^{24} d_{ij} \mathcal{L}^*_j, \quad
    i\in\{1,\dots,21\},
  \label{eqn:FunctionalsD}
\end{equation}
where $d_{ij}$ are constants.

Now from \eqref{eqn:Affine} we see that
\begin{align*}
  &\frac{\partial x}{\partial \hat{x}} = B_{11}   &\frac{\partial x}{\partial \hat{y}} = B_{12} \\
  &\frac{\partial y}{\partial \hat{x}} = B_{21}   &\frac{\partial y}{\partial
    \hat{y}} = B_{22}.
\end{align*}
With this we can determine the relationship between the gradient of $\hat{\varphi}$ on the
triangle $\hat{K}$ to the gradient of $\varphi$ on the triangle $K$
\begin{align*}
  \nabla_{\mathbf{\hat{x}}}\left( \varphi \circ F \right) &= \begin{bmatrix}
    \dfrac{\partial \varphi\circ F}{\partial \hat{x}} \\[1em]
    \dfrac{\partial \varphi\circ F}{\partial \hat{y}}
  \end{bmatrix} \\
  &=\begin{bmatrix}
    \dfrac{\partial \varphi\circ F}{\partial x} \, \dfrac{\partial x}{\partial \hat{x}} +
      \dfrac{\partial \varphi\circ F}{\partial y} \, \dfrac{\partial y}{\partial \hat{x}} \\[1em]
    \dfrac{\partial \varphi\circ F}{\partial x} \, \dfrac{\partial x}{\partial \hat{y}} +
      \dfrac{\partial \varphi\circ F}{\partial y} \, \dfrac{\partial y}{\partial \hat{y}}
  \end{bmatrix} \\
  &= \begin{bmatrix}
    \dfrac{\partial x}{\partial \hat{x}} & \dfrac{\partial y}{\partial \hat{x}} \\[1em]
    \dfrac{\partial x}{\partial \hat{y}} & \dfrac{\partial y}{\partial \hat{y}}
  \end{bmatrix}\, \begin{bmatrix}
    \dfrac{\partial \varphi\circ F}{\partial x} \\[1em] \dfrac{\partial \varphi\circ F}{\partial y}
  \end{bmatrix}
\end{align*}
thus we have
\begin{equation}
  \nabla_{\mathbf{\hat{x}}}\left( \varphi \circ F \right) = B^T \nabla_{\mathbf{x}} \varphi \circ F
  \label{eqn:Gradient}
\end{equation}

If we define the condensed Hessian, in the same way as Dominguez et. al.
\cite{Dominguez08}, as $H_x(\varphi) = \left[ \varphi_{xx},\, \varphi_{xy},\,
\varphi_{yy}\right]^T$ then using \eqref{eqn:Affine} we can determine the
relationship between the condensed Hessian of $\hat{\varphi}$ on the triangle
$\hat{K}$ to the condensed Hessian of $\varphi$ on the triangle $K$
\begin{align*}
  H_{\mathbf{\hat{x}}}(\varphi\circ F) &= \begin{bmatrix}
    \dfrac{\partial^2 \varphi\circ F}{\partial \hat{x}^2} \\[1em]
    \dfrac{\partial^2 \varphi\circ F}{\partial \hat{x} \partial\hat{y}} \\[1em]
    \dfrac{\partial^2 \varphi\circ F}{\partial \hat{y}^2}
  \end{bmatrix} \\
  &= \begin{bmatrix}
    \dfrac{\partial}{\partial \hat{x}}\left( \dfrac{\partial \varphi\circ
      F}{\partial x}\, \dfrac{\partial x}{\partial \hat{x}} + \dfrac{\partial
      \varphi\circ F}{\partial y}\, \dfrac{\partial y}{\partial
      \hat{x}} \right) \\[1em]
    \dfrac{\partial}{\partial \hat{y}}\left( \dfrac{\partial \varphi\circ
      F}{\partial x}\, \dfrac{\partial x}{\partial \hat{x}} + \dfrac{\partial
      \varphi\circ F}{\partial y}\, \dfrac{\partial y}{\partial
      \hat{x}} \right) \\[1em]
    \dfrac{\partial}{\partial \hat{y}}\left( \dfrac{\partial \varphi\circ
      F}{\partial x}\, \dfrac{\partial x}{\partial \hat{y}} + \dfrac{\partial
      \varphi\circ F}{\partial y}\, \dfrac{\partial y}{\partial
      \hat{y}} \right)
  \end{bmatrix}
\end{align*}
\begin{align*}
  &= \begin{bmatrix}
    \dfrac{\partial^2 \varphi\circ F}{\partial x^2}\, \left(\dfrac{\partial
      x}{\partial \hat{x}}\right)^2 + 2 \dfrac{\partial^2 \varphi\circ F}{\partial x
      \partial y} \, \dfrac{\partial x}{\partial \hat{x}} \,
      \dfrac{\partial y}{\partial \hat{x}} + \dfrac{\partial \varphi\circ
      F}{\partial y^2}\, \left(\dfrac{\partial y}{\partial \hat{x}}\right)^2 \\[1em]
    \dfrac{\partial^2 \varphi\circ F}{\partial x^2}\, \dfrac{\partial
      x}{\partial \hat{x}}\, \dfrac{\partial x}{\partial \hat{y}} + \left(
      \dfrac{\partial x}{\partial \hat{x}}\, \dfrac{\partial y}{\partial
      \hat{y}} + \dfrac{\partial x}{\partial \hat{y}}\,
      \dfrac{\partial y}{\partial \hat{x}} \right) \dfrac{\partial^2 \varphi\circ F}{\partial x
      \partial y} + \dfrac{\partial \varphi\circ F}{\partial y^2}\,
      \dfrac{\partial y}{\partial \hat{x}} \, \dfrac{\partial y}{\partial
      \hat{y}} \\[1em]
    \dfrac{\partial^2 \varphi\circ F}{\partial x^2}\, \left(\dfrac{\partial
      x}{\partial \hat{y}}\right)^2 + 2 \dfrac{\partial^2 \varphi\circ F}{\partial x
      \partial \hat{y}} \, \dfrac{\partial x}{\partial \hat{y}} \,
      \dfrac{\partial y}{\partial \hat{y}} + \dfrac{\partial \varphi\circ
      F}{\partial y^2}\, \left(\dfrac{\partial y}{\partial \hat{y}}\right)^2
  \end{bmatrix}
\end{align*}
\begin{align*}
  &= \begin{bmatrix}
    \left(\dfrac{\partial x}{\partial \hat{x}}\right)^2
      & 2 \dfrac{\partial x}{\partial \hat{x}}\, \dfrac{\partial y}{\partial \hat{x}}
      & \left(\dfrac{\partial y}{\partial \hat{x}}\right)^2 \\[1em]
    \dfrac{\partial x}{\partial \hat{x}}\, \dfrac{\partial x}{\partial \hat{y}}
      & \left(\dfrac{\partial x}{\partial \hat{x}}\, \dfrac{\partial y}{\partial
      \hat{y}} + \dfrac{\partial x}{\partial \hat{y}}\,
      \dfrac{\partial y}{\partial \hat{x}} \right)
      & \dfrac{\partial y}{\partial \hat{x}} \, \dfrac{\partial y}{\partial \hat{y}} \\[1em]
    \left(\dfrac{\partial x}{\partial \hat{y}}\right)^2
      & 2 \dfrac{\partial x}{\partial \hat{y}}\, \dfrac{\partial y}{\partial \hat{y}}
      & \left(\dfrac{\partial y}{\partial \hat{y}}\right)^2 \\[1em]
  \end{bmatrix}\, \begin{bmatrix}
    \dfrac{\partial^2 \varphi\circ F}{\partial x^2} \\[1em]
    \dfrac{\partial^2 \varphi\circ F}{\partial x \partial y} \\[1em]
    \dfrac{\partial^2 \varphi\circ F}{\partial y^2}
  \end{bmatrix}
\end{align*}
and so
\begin{equation}
  H_{\mathbf{\hat{x}}}(\varphi\circ F) = \Theta H_{\mathbf{x}}(\varphi)\circ F,
  \label{eqn:Hessian}
\end{equation}
where
\begin{equation*}
  \Theta = \begin{bmatrix}
    B_{11}^2 & 2B_{11}B_{21} & B_{21}^2 \\[0.5em]
    B_{12}B_{11} & B_{12}B_{21} + B_{11}B_{22} & B_{21}B_{22} \\[0.5em]
    B_{12}^2 & 2B_{22}B_{12} & B_{22}^2
  \end{bmatrix}.
\end{equation*}

Therefore we see that
\begin{equation*}
  \tilde{\mathcal{L}}_i = \mathcal{L}_i, \quad
  \begin{bmatrix}
    \tilde{\mathcal{L}}^x_i \\ \tilde{\mathcal{L}}^y_i
  \end{bmatrix} = B^T \begin{bmatrix}
    \mathcal{L}^x_i \\ \mathcal{L}^y_i
  \end{bmatrix}, \quad
  \begin{bmatrix}
    \tilde{\mathcal{L}}^{xx}_i \\ \tilde{\mathcal{L}}^{xy}_i \\ \tilde{\mathcal{L}}^{yy}_i
  \end{bmatrix} = \Theta \begin{bmatrix}
    \mathcal{L}^{xx}_i \\ \mathcal{L}^{xy}_i \\ \mathcal{L}^{yy}_i
  \end{bmatrix}.
\end{equation*}
Additionally, notice $\mathbf{v}_i = B\hat{\mathbf{v}}_i$ for
$i=\{1,2,3\}$ where $\hat{\mathbf{v}}_i$ are defined on the reference triangle
$\hat{K}$. It should also be noted that
\begin{equation*}
  \begin{bmatrix}
    \mathcal{L}^{\perp}_i \\ \mathcal{L}^{||}_i
  \end{bmatrix} = \begin{bmatrix} -v^y_i & v^x_i \\ v^x_i & v^y_i \end{bmatrix}
  \begin{bmatrix}
    \mathcal{L}^x_i \\ \mathcal{L}^y_i
  \end{bmatrix}
\end{equation*}
and therefore
\begin{equation*}
  \begin{bmatrix}
    \mathcal{L}^{x}_i \\ \mathcal{L}^{y}_i
  \end{bmatrix} = \frac{1}{|v_i|^2}
  \begin{bmatrix} -v^y_i & v^x_i \\ v^x_i & v^y_i \end{bmatrix}
  \begin{bmatrix}
    \mathcal{L}^{\perp}_i \\ \mathcal{L}^{||}_i
  \end{bmatrix}.
\end{equation*}
Then
\begin{align*}
  \tilde{\mathcal{L}}^n_i(\varphi) &= \hat{\mathcal{L}}^n_i(\varphi\circ F) \\
  &= \frac{1}{|\hat{\mathbf{v}}_i|}R \hat{\mathbf{v}}_i \cdot
    \nabla_{\hat{\mathbf{x}}}(\varphi \circ F)(\hat{\mathbf{m}}_i) \\
  &= \frac{1}{|\hat{\mathbf{v}}_i|}R \hat{\mathbf{v}}_i \cdot B^T
    \nabla_{\mathbf{x}} \varphi(\mathbf{m}_i) \\
  &= \frac{1}{|\hat{\mathbf{v}}_i|}R \hat{\mathbf{v}}_i \cdot \ell_i^{-2} B^T
  \begin{bmatrix} -v^y_i & v^x_i \\ v^x_i & v^y_i \end{bmatrix}
  \begin{bmatrix}
    \mathcal{L}^{\perp}_i \\ \mathcal{L}^{||}_i
  \end{bmatrix},
\end{align*}
where $\ell_i$ is the length of the $i^{th}$ side of the triangle $K$. This can be written in the form
\begin{equation*}
  \tilde{\mathcal{L}}_i^n = f_i \mathcal{L}^{\perp}_i + g_i \mathcal{L}^{||}_i,
\end{equation*}
where
\begin{equation*}
  f_i = \frac{1}{\ell_i^2 |\hat{\mathbf{v}}_i|}R \hat{\mathbf{v}}_i \cdot B^T
    R\mathbf{v}_i \quad
  g_i = \frac{1}{\ell_i^2 |\hat{\mathbf{v}}_i|}R \hat{\mathbf{v}}_i \cdot B^T
    \mathbf{v}_i .
\end{equation*}

Therefore, we can construct a $21\times 24$ the matrix $D$ in block diagonal
form in the following way:
\begin{equation*}
  D = \text{diag}[I_3, B^T, B^T, B^T, \Theta, \Theta, \Theta, Q],
\end{equation*}
where
\begin{equation*}
  Q = \left[\begin{array}{ccc|ccc}
    f_1 & & & g_1 & & \\
    & f_2 & & & g_2 &\\
    & & f_3 & & & g_3
  \end{array}\right].
\end{equation*}

It also holds that
\begin{equation}
  \mathcal{L}^*_i = \sum_{j=1}^{21} e_{ij} \mathcal{L}_j \quad i\in
  \{1,\dots,24\}.
  \label{eqn:FunctionalsE}
\end{equation}
For the degrees of freedom corresponding to the vertices, i.e. function values,
first derivative values, and second/mixed derivative values, it should be clear
that $\mathcal{L}^*_i = \mathcal{L}_i$ for $i=1,\dots,18$. For the degrees of
freedom corresponding to the normal derivatives at the midpoints we see that
since $\ell_i \mathbf{n}_i = R \mathbf{v}_i$ we see that $\mathcal{L}^{\perp}_i
= \ell_i \mathcal{L}^n_i$. Now let $\phi$ be an arbitrary polynomial in
$\mathbb{P}_5(K)$ and define
\begin{equation*}
  \psi(t) := \varphi(t\mathbf{x}_{\beta} + (1-t)\mathbf{x}_{\alpha}) \in
  \mathbb{P}_5(t) \; \alpha<\beta \text{ and} \alpha,\,\beta \in \{1,2,3\}.
\end{equation*}
Then, we get
\begin{equation}
  \psi'(\nicefrac{1}{2}) = \frac{15}{8} (\psi(1)-\psi(0)) -
  \frac{7}{16} (\psi'(1)+\psi'(0)) + \frac{1}{32} (\psi''(1)-\psi''(0)).
  \label{eqn:Psi}
\end{equation}
Now take $\gamma$ to be the index corresponding to $\mathbf{v}_{\gamma} =
\mathbf{x}_{\beta} - \mathbf{x}_{\alpha}$ and since
\begin{align*}
  \psi'(\nicefrac{1}{2}) &= \mathcal{L}_{\gamma}(\varphi) \\
  \psi(0) &= \mathcal{L}_{\alpha}(\varphi) \\
  \psi(1) &= \mathcal{L}_{\beta}(\varphi) \\
  \psi'(0) &= v^x_{\gamma}\mathcal{L}^x_{\alpha}(\varphi) +
    v^y_{\gamma}\mathcal{L}^y_{\alpha}(\varphi) \\
  \psi'(1) &= v^x_{\gamma}\mathcal{L}^x_{\beta}(\varphi) +
    v^y_{\gamma}\mathcal{L}^y_{\beta}(\varphi) \\
  \psi''(0) &= (v^x_{\gamma})^2\mathcal{L}^{xx}_{\alpha}(\varphi) +
    2 v^x_{\gamma}v^y_{\gamma}\mathcal{L}^{xy}_{\alpha}(\varphi) +
    (v^y_{\gamma})^2\mathcal{L}^{yy}_{\alpha}(\varphi) \\
  \psi''(1) &= (v^x_{\gamma})^2\mathcal{L}^{xx}_{\beta}(\varphi) +
    2 v^x_{\gamma}v^y_{\gamma}\mathcal{L}^{xy}_{\beta}(\varphi) +
    (v^y_{\gamma})^2\mathcal{L}^{yy}_{\beta}(\varphi),
\end{align*}
we see that by applying this to \eqref{eqn:Psi} we get the following expression
\begin{equation*}
  \begin{split}
    \mathcal{L}^{||}_{\gamma} = \frac{15}{8}(-\mathcal{L}_{\alpha} +
      \mathcal{L}_{\beta})
    - \frac{7}{16} ( v^x_{\gamma}\mathcal{L}^x_{\alpha}(\varphi) +
      v^y_{\gamma}\mathcal{L}^y_{\alpha}(\varphi) +
      v^x_{\gamma}\mathcal{L}^x_{\beta}(\varphi) +
      v^y_{\gamma}\mathcal{L}^y_{\beta}(\varphi))
      \\
    + \frac{1}{32}(-(v^x_{\gamma})^2\mathcal{L}^{xx}_{\alpha}(\varphi) -
      2 v^x_{\gamma}v^y_{\gamma}\mathcal{L}^{xy}_{\alpha}(\varphi) -
      (v^y_{\gamma})^2\mathcal{L}^{yy}_{\alpha}(\varphi) \\
      + (v^x_{\gamma})^2\mathcal{L}^{xx}_{\beta}(\varphi) +
      2 v^x_{\gamma}v^y_{\gamma}\mathcal{L}^{xy}_{\beta}(\varphi) +
      (v^y_{\gamma})^2\mathcal{L}^{yy}_{\beta}(\varphi)).
  \end{split}
\end{equation*}
Therefore, we have the $24\times 21$ matrix
\begin{equation*}
  E = \begin{bmatrix} I_{18} & \mathbf{0} \\
    \mathbf{0} & L \\
    T & \mathbf{0}
  \end{bmatrix}.
\end{equation*}
We then can define
\begin{equation*}
  \quad L = \text{diag}[\ell_1,\ell_2,\ell_3]
\end{equation*}
and the $(3\times 18)$ block matrix T as being composed of three sub blocks
($3\times 3,\, 3\times 6,\text{ and } 3\times 9$ respectively)
\begin{equation*}
  \frac{15}{8} \begin{bmatrix} -1 & 1 & 0\\ -1 & 0 & 1\\ 0 & -1 & 1\end{bmatrix}, \quad
  -\frac{7}{16} \begin{bmatrix} \mathbf{v}_1^T & \mathbf{v}_1^T & \mathbf{0}\\
    \mathbf{v}_2^T & \mathbf{0} & \mathbf{v}_2^T\\ \mathbf{0} & \mathbf{v}_3^2 & \mathbf{v}_3^T\end{bmatrix}, \quad
  \frac{1}{32} \begin{bmatrix} -\mathbf{w}_1^T & \mathbf{w}_1^T & \mathbf{0}\\
    -\mathbf{w}_2^T & \mathbf{0} & \mathbf{w}_2^T\\ \mathbf{0} & -\mathbf{w}_3^2 & \mathbf{w}_3^T\end{bmatrix},
\end{equation*}
where $\mathbf{w}_i^T = \left[ (v^x_i)^2, 2v^x_i v^y_i, (v^y_i)^2\right]$.

Finally, notice that by combining \eqref{eqn:FunctionalsD} and \eqref{eqn:FunctionalsE}
we get the matrix in \eqref{eqn:FunctionalsC} and
\begin{equation}
  C=DE.
  \label{eqn:Transformation}
\end{equation}
Thus, we have the transformation for the Argyris triangle $K$ into a reference
triangle $\hat{K}$.

