The main goal of this section is to verify the theoretical rates of convergence developed in
\autoref{sse:SQGE2LE} by comparing with the observed rates of convergence in our numerical tests.
%Additonally, we intend to demonstrate the time savings from applying the Two-Level method to the
%SQGE. To demonstrate that the two-level method is an affective way to address computational time we
%must demonstrate that the error is at least the same order of convergence as the FE applied to SQGE
%without the two-level method, or we must demonstrate that the cost of using a finer mesh that will
%produce the same level of accuracy is still cheaper computationally.
Unlike the previous section, for the two-level method we must demonstrate rates of convergence for
two different meshes, the fine mesh, $h$, and the coarse mesh $H$. Since the error corresponding to
the coarse mesh dominates the error corresponding to the fine mesh the rate of convergence in the
coarse mesh is easy to demonstrate. To this end one need only pick a constant fine mesh and then
refine the coarse mesh. By doing this one is able to demonstrate the rate of convergence in the
coarse mesh. However, the rate of convergence in the fine mesh is a bit more trick to demonstrate,
because as stated previously the coarse mesh error dominates the fine mesh error and therefore one
must balance the error between the fine mesh and the coarse mesh so as to be able to demonstrate the
rate of convergence in the fine mesh. To this end we set the error terms in
\eqref{eqn:TwoLevelError} equal to eachother to come up with the relationship 
\begin{equation}
  h = \left(\ln h\right)^{\frac{1}{8}} H^{\frac{5}{4}}.
  \label{eqn:OrderRelation}
\end{equation}
This relation is super linear and essentially means that the coarse mesh $H$ must be about half the
size the fine mesh $h$. 

To this end we apply the Two-Level method to the SQGE \eqref{sqge_psi_1} with $Re=Ro=1$ and exact
solution
\begin{equation}
  \left(\sin 4\pi x \cdot \sin 2\pi y\right)^2.
  \label{eqn:2LExact}
\end{equation}
Additionally, the homogeneous boundary conditions are $\psi=\dfrac{\partial \psi}{\partial
\mathbf{n}}=0$ and the forcing function $F$ corresponds to applying the SQGE to the exact solution
\eqref{eqn:2LExact}. Theses BCs and exact solution will be used in all the Two-Level tests that
follow.

First, we benchmark our numerical tests by running SQGE without the two-level method. Comparing to
this benchmark will allows us to compare errors in the $H^2$ norm for SQGE to the errors in $H^2$
norm with the two-level method applied.
\begin{table}[H]
  \begin{center}
  \begin{tabular}{|c|c|c|c|c|}
    \hline
    $h$ & $DoFs$ & $e_2$ & $H^2$ order & time (s) \\
    \hline
    $\nicefrac{1}{2}$ & $70$ & $174.5$ & $-$ & $0.70114$ \\[0.2em] 
    $\nicefrac{1}{4}$ & $206$ & $45.11$ & $1.951$ & $0.63921$ \\[0.2em] 
    $\nicefrac{1}{8}$ & $694$ & $9.286$ & $2.28$ & $3.3169$ \\[0.2em] 
    $\nicefrac{1}{16}$ & $2534$ & $0.6503$ & $3.836$ & $15.105$ \\[0.2em] 
    $\nicefrac{1}{32}$ & $9670$ &  $0.03566$ & $4.189$ & $66.037$ \\[0.2em] 
    $\nicefrac{1}{64}$ & $37766$ & $0.002109$ & $4.08$ & $303.05$ \\[0.2em]
    \hline
  \end{tabular}
  \caption{Benchmark Errors and Rate of Convergence SQGE \eqref{sqge_psi_1}.}
  \label{tab:2LBenchmark}
  \end{center}
\end{table}
As can be seen in \autoref{tab:2LBenchmark} the initial error is quite large in the $H^2$
norm. This is due to lack of resolution in the approximate solution because of the mesh coarseness.
The given exact solution \eqref{eqn:2LExact} has a period of $\dfrac{1}{2}$ in the $x$-direction and
a period of one in the $y$-direction, therefore we must have a mesh size that much smaller than
$\dfrac{1}{2}$ to be able to approximate the function appropriately. However, we see that by
$h=\dfrac{1}{64}$ the error has on the order of $10^{-3}$ and the rate of convergence matches the
theoretical rate of convergence predicted in \autoref{sec:QGEError}.

\begin{table}[H]
  \begin{center}
  \begin{tabular}{|c|c|c|c|c|c|}
    \hline
    $H$ & $DoFs$ for $H$ & $h$ & $DoFs$ for $h$ & $e_2$ & $H^2$ order \\
    \hline
    $\nicefrac{1}{2}$ & $70$ & $\nicefrac{1}{4}$ & $206$ & $45.44$ & $-$ \\[0.2em] 
    $\nicefrac{1}{4}$ & $206$ & $\nicefrac{1}{8}$ & $694$ & $10.89$ & $2.061$ \\[0.2em] 
    $\nicefrac{1}{8}$ & $694$ &$\nicefrac{1}{16}$ & $2534$ & $0.8404$ & $3.696$ \\[0.2em] 
    $\nicefrac{1}{16}$ & $2534$ & $\nicefrac{1}{32}$ & $9670$ & $0.04075$ & $4.366$ \\[0.2em] 
    $\nicefrac{1}{32}$ & $9670$ & $\nicefrac{1}{64}$ & $37766$ &  $0.002141$ & $4.25$ \\[0.2em] 
    $\nicefrac{1}{64}$ & $37766$ & $\nicefrac{1}{128}$ & $149254$ & $0.0001298$ & $4.044$ \\[0.2em]
    \hline
  \end{tabular}
  \caption{Errors and Rate of Convergence in $h$ for the Two-Level method applied to SQGE \eqref{sqge_psi_1}.}
  \label{tab:TwoLevelh}
  \end{center}
\end{table}

In \autoref{tab:TwoLevelh} we use the relationship \eqref{eqn:OrderRelation} to determine a mesh
size relationship between $h$ and $H$. As stated previously, this relation implies that $H \approx
\dfrac{h}{2}$ and therefore we chose the $(H,h)$ pairs corresponding to 
\begin{equation*}
  \left\{ \left(\frac{1}{2}, \frac{1}{4}\right),
  \left(\frac{1}{4}, \frac{1}{8}\right),
  \left(\frac{1}{8}, \frac{1}{16}\right),
  \left(\frac{1}{16}, \frac{1}{32}\right),
  \left(\frac{1}{32}, \frac{1}{64}\right),
  \left(\frac{1}{64}, \frac{1}{128}\right)\right\}
\end{equation*}
so as to demonstrate the rate of convergence in $h$. As can be seen in \autoref{tab:TwoLevelh} the
rate of convergence does, in deed, approach the rate of convergence predicted in
\autoref{sse:SQGE2LE}. Additionally, we see that the $H^2$ errors are of the same order
as the corresponding errors in \autoref{tab:2LBenchmark} for the same value of $h$.

To demonstrate the rate of convergence in $H$ we take a constant fine mesh size of
$h=\dfrac{1}{128}$ and vary the coarse mesh $H$. To this end we chose the $(H,h)$ pairs
corresponding to 
\begin{equation*}
  \left\{ \left(\frac{1}{2}, \frac{1}{128}\right),
  \left(\frac{1}{4}, \frac{1}{128}\right),
  \left(\frac{1}{8}, \frac{1}{128}\right),
  \left(\frac{1}{16}, \frac{1}{128}\right),
  \left(\frac{1}{32}, \frac{1}{128}\right),
  \left(\frac{1}{64}, \frac{1}{128}\right)\right\}.
\end{equation*}

\begin{table}[H]
  \begin{center}
  \begin{tabular}{|c|c|c|c|c|c|}
    \hline
    $H$ & $DoFs$ for $H$ & $h$ & $DoFs$ for $h$ & $e_2$ & $H^2$ order \\
    \hline
    $\nicefrac{1}{2}$ & $70$ & $\nicefrac{1}{128}$ & $149254$ & $7.653$ & $-$ \\[0.2em] 
    $\nicefrac{1}{4}$ & $206$ & $\nicefrac{1}{128}$ & $149254$ & $5.916$ & $0.3715$ \\[0.2em] 
    $\nicefrac{1}{8}$ & $694$ & $\nicefrac{1}{128}$ & $149254$ & $0.5373$ & $3.461$ \\[0.2em] 
    $\nicefrac{1}{16}$ & $2534$ & $\nicefrac{1}{128}$ & $149254$ & $0.0199$ & $4.755$ \\[0.2em] 
    $\nicefrac{1}{32}$ & $9670$ & $\nicefrac{1}{128}$ & $149254$ &  $0.0003779$ & $5.719$ \\[0.2em] 
    $\nicefrac{1}{64}$ & $37766$ & $\nicefrac{1}{128}$ & $149254$ & $1.381\times 10^{-5}$ & $4.775$ \\[0.2em]
    \hline
  \end{tabular}
  \caption{Errors and Rate of Convergence in $H$ for the Two-Level method applied to SQGE \eqref{sqge_psi_1}.}
  \label{tab:TwoLevelH}
  \end{center}
\end{table}

As can be seen in \autoref{tab:TwoLevelH} the rate of convergence does, in deed, approach the rate
of convergence predicted in \autoref{sse:SQGE2LE}. Additionally, we see that the $H^2$ errors are of
the same order as the corresponding errors in \autoref{tab:2LBenchmark} for the same value of $h$.

%Finally, to demonstrate the computational efficiency of the two-level method we compare the time
%required to run simulations for the SQGE with out the two-level method on various mesh sizes and
%then we simulate the SQGE with the two-level method for the same fine mesh size as those used in the
%non two-level method, while maintaining the same coarse mesh size accross all simulations. For these
%simulations we will take the coarse mesh to be $H=\dfrac{1}{2}$ and vary the fine mesh $h$ over 
%\begin{equation*}
%  \left\{\frac{1}{2}, \frac{1}{4}, \frac{1}{8}, \frac{1}{16}, \frac{1}{32}, \frac{1}{64}\right\}.
%\end{equation*}
%
%\begin{table}[H]
%  \begin{center}
%  \begin{tabular}{|c|c|c|c|c|c|c|}
%    \hline
%    $H$ & $DoFs$ for $H$ & $h$ & $DoFs$ for $h$ & $e_2$ & $H^2$ order & time (s) \\
%    \hline
%    $\nicefrac{1}{2}$ & $70$ & $\nicefrac{1}{2}$ & $70$ & $174.55$ & $-$ & $0.2287$ \\[0.2em] 
%    $\nicefrac{1}{2}$ & $70$ & $\nicefrac{1}{4}$ & $206$ & $45.44$ & $1.941$ & $0.2898$ \\[0.2em] 
%    $\nicefrac{1}{2}$ & $70$ &$\nicefrac{1}{8}$ & $694$ & $11.99$ & $ 1.922$ & $0.6902$ \\[0.2em] 
%    $\nicefrac{1}{2}$ & $70$ & $\nicefrac{1}{16}$ & $2534$ & $7.681$ & $0.6431$ & $2.803$ \\[0.2em] 
%    $\nicefrac{1}{2}$ & $70$ & $\nicefrac{1}{32}$ & $9670$ &  $7.653$ & $0.00511$ & $12.75$ \\[0.2em] 
%    $\nicefrac{1}{2}$ & $70$ & $\nicefrac{1}{64}$ & $37766$ & $7.653$ & $1.526\times 10^{-5}$ & $58.67$ \\[0.2em]
%    \hline
%  \end{tabular}
%  \caption{Errors and Rate of Convergence in $H$ for the Two-Level method applied to SQGE \eqref{sqge_psi_1}.}
%  \label{tab:TwoLevelH}
%  \end{center}
%\end{table}
