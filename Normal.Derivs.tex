During the discussion above we have been concerned with transforming a basis on
a local triangle to that of a basis on a reference triangle. However, we have
not considered the implications that results from the assumption of always
rotating the normal derivative $\dfrac{\pi}{2}$ counterclockwise.  If we always
rotate $\dfrac{\pi}{2}$ counterclockwise there will be a jump discontinuity in
the basis function along that edge.  This results from assuming
$\dfrac{\partial\varphi_i}{\partial\mathbf{n}_i} = 1$ on triangle $I$ while at
the same time assuming $\dfrac{\partial\varphi_i}{\partial\mathbf{n}_i} = -1$ on
the adjacent triangle $J$. The potentional issue can be seen graphically in
\autoref{fig:Normals}. This is what causes the discontinuity along shared edges.
To address this one has two options: one can multiply the value of $\varphi_i$ by
negative one on the triangle $J$, without changing $\varphi_i$ on triangle
$I$, or one can avoid the issue all together by numbering the nodes in such a way
that this mismatching of normal derivatives doesn't occur.

\begin{figure}%[H]
	\begin{center}
    \begin{tikzpicture}%[scale=0.75]
\tikzstyle{every node}=[font=\tiny]
      \draw (0,0) node[below left,fill=none]
      {$\begin{array}{l}1,\\2,3\\4,5,6\end{array}$}
      -- (3,0) node[below,fill=none] {$37$}
      -- (6,0) node[below right,fill=none]
      {$\begin{array}{l}7,\\8,9\\10,11,12\end{array}$}
      -- (6,3) node[above right,fill=none] {$42$}
      -- (6,6) node[right,fill=none]
      {$\begin{array}{l}19,\\20,21\\22,23,24\end{array}$}
      -- (3,6) node[above left,fill=none] {$38$}
      -- (0,6) node[left,fill=none]
      {$\begin{array}{l}13,\\14,15\\16,17,18\end{array}$}
      -- (0,3) node[above left,fill=none] {$40$}
      -- cycle;
      \draw (0,0) -- (3,3) node[above left,fill=none] {$41$} -- (6,6);
      \draw (6,6)
      -- (6,9) node[above right,fill=none] {$45$}
      -- (6,12) node[above right,fill=none]
      {$\begin{array}{l}31,\\32,33\\34,35,36\end{array}$}
      -- (3,12) node[above,fill=none] {$39$}
      -- (0,12) node[above left,fill=none]
      {$\begin{array}{l}25,\\26,27\\28,29,30\end{array}$}
      -- (0,9) node[above left,fill=none] {$43$}
      -- (0,6);
      \draw (0,6) -- (3,9) node[above left,fill=none] {$44$} -- (6,12);
      \draw[<->] (3,7) node[above,fill=none]
      {$\dfrac{\partial\varphi_{38}}{\partial\mathbf{n}_{38}}$}
      -- (3,5) node[below,fill=none]
      {$\dfrac{\partial\varphi_{38}}{\partial\mathbf{n}_{38}}$} ;
    \end{tikzpicture}
	\end{center}
  \caption{Illustration of continuity issue in normal derivatives for $C^1$
  finite elements.}
	\label{fig:Normals}
\end{figure}



