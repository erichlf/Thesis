In this thesis we studied the finite element discretization of the pure
streamfunction form of the quasi-geostrophic equations. The pure streamfunction
form of the QGE requires $C^1$ FEs to be conforming, thus the implementation of
the of the $C^1$ FE was discussed in length. In particular, the implementation
of the Argyris element was discussed in \autoref{sec:Argyris}, including an in
depth discussion of the novel transformation developed by Dominguez et. el.
\cite{Dominguez08} in \autoref{sse:Trans}. Additionally, we proposed a two-level
FE discretization for the SQGE in \autoref{sec:TwoLevel} based on the conforming
FE discretization used for the SQGE in \autoref{sec:SQGEFEM}. The two-level
algorithm consists of two steps. In the first step, the nonlinear system is
solved on a coarse mesh. In the second step, the nonlinear system is linearized
around the approximation found in the first step, and the resulting linear
system is solved on the fine mesh.  This method significantly improved
simulation times for the SQGE over that of the one-level method.

For the conforming FE discretization of SQGE, in \autoref{sec:SQGEErrors} we
proved, for the first time, optimal error estimates in the $H^2,H^1$, and $L^2$
norms. The error analysis only relied on the FE discretization being conforming
and was not specific to any particular element. Specifically, for the Argyris
element we showed that the order of convergence in the $H^2,H^1,L^2$ norms were
$O(h^4),O(h^5),$ and $O(h^6)$, respectively.

Rigorous error estimates for the two-level FE discretization were derived. These
estimates are optimal in the following sense: for an appropriately chosen
scaling between the coarse mesh, $H$, and the fine mesh, $h$, the error in the
two-level method is of the same order as the error in the standard one-level
method (i.e. solving the nonlinear system directly on the fine mesh).

Numerical experiments for the two-level algorithm, with the Argyris element,
were also carried out. The numerical results verified, numerically, the
theoretical error estimates, both with respect to the coarse mesh size, $H$, and
the fine mesh size, $h$.  Furthermore, the numerical results showed that, for an
appropriate scaling between the coarse and fine meshes, the two-level method
significantly decreases the computational time of the standard one-level method.

We plan to extend this study in several directions, including the time-dependent
quasi-geostrophic equations and the two-layer quasi-geostrophic equations.
