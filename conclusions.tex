In this thesis we studied the finite element discretization of the pure
streamfunction form of the quasi-geostrophic equations. The pure streamfunction
form of the QGE requires $C^1$ FEs to be conforming, thus the implementation of
the of the $C^1$ FE was discussed in length. In particular, the implementation
of the Argyris element was discussed in \autoref{sec:Argyris}, including an in
depth discussion of the novel transformation developed by Dominguez et. el.
\cite{Dominguez08} in \autoref{sse:Trans}. Additionally, we proposed a two-level
FE discretization for the SQGE in \autoref{sec:TwoLevel} based on the conforming
FE discretization used for the SQGE in \autoref{sec:SQGEFEM}. The two-level
algorithm consists of two steps. In the first step, the nonlinear system is
solved on a coarse mesh. In the second step, the nonlinear system is linearized
around the approximation found in the first step, and the resulting linear
system is solved on the fine mesh.  This method significantly improved
simulation times for the SQGE over that of the one-level method. Nest, a
conforming FE semi-discretization was developed for the time-dependent QGE based
upon the FE discretization developed in \autoref{sec:SQGEFEM}.

For the conforming FE discretization of SQGE, in \autoref{sec:SQGEErrors} we
proved, for the first time, optimal error estimates in the $H^2,H^1$, and $L^2$
norms. The error analysis only relied on the FE discretization being conforming
and was not specific to any particular element. Specifically, for the Argyris
element we showed that the order of convergence in the $H^2,H^1,L^2$ norms were
$O(h^4),O(h^5),$ and $O(h^6)$, respectively. The error analysis in
\autoref{sec:SQGEErrors} was then used to develop new rigorous error estimates
for the two-level FE discretization. These estimates are optimal in the
following sense: for an appropriately chosen scaling between the coarse mesh,
$H$, and the fine mesh, $h$, the error in the two-level method is of the same
order as the error in the standard one-level method (i.e. solving the nonlinear
system directly on the fine mesh). For the semi-discretization of the QGE we
proved similar optimal error estimates, i.e. the order of convergence for the
energy norm is of $O(h^4)$.

Numerical experiments for the SQGE (\autoref{sec:SQGETests}), two-level
algorithm (\autoref{sec:SQGE2LTests}), and QGE (\autoref{sec:QGETests}) with the
Argyris element, were also carried out. For the SQGE a few benchmark problems
were used to verify our code. These benchmark problems included the Linear
Stommel model \cite{Vallis06,Myers}, and the Linear Stommel-Munk model
\cite{Cascon}. With the agreement of our solutions to that of the literature our
code was thus verfied. We then verified numerically the theoretical error
estimates developed in \autoref{sec:SQGETests}. We then performed numerical
experiments on non-rectangular domains, including a FE discretization of the
SQGE applied to the Mediterranean. The test results for the Mediterranean appeared
to be in agreement with observed surface currents and the numerical experiments
developed by Galan del Sastre \cite{Galan-del-Sastre2004}. The code which was
verified in \autoref{sec:SQGETests} was then modified to allow for the use of
the two-level method and numerical experiments were then carried out for the
two-level method.  These numerical experiments verified numerically the
theoretical error estimates developed in \autoref{sse:SQGE2LE}, both with
respect to the coarse mesh size, $H$, and the fine mesh size, $h$.  Furthermore,
the numerical results showed that, for an appropriate scaling between the coarse
and fine meshes, the two-level method significantly decreases the computational
time of the standard one-level method. Next, the same code which was developed
and verified for the SQGE was then modified to deal with time-dependence. We
applied a Implicit Euler scheme and verified numerical the theoretical spatial
rates of convergence developed for the semi-discretization. Additionally, we
observed expected rates of convergence in time for the Implicit Euler scheme
($O(h)$).

The QGE has many unique challenges for numerical modelling. These challenges
include, but are not limited to, unstable solutions, resulting from internal
layers and western boundary layers, and the QGE can be computationally expensive
for large domains, such as the North Atlantic. To address these issues we plan
to extend these studies in several directions, including stabilization methods
such as \emph{Petrov-Galerkin} stabilization, \emph{adaptive mesh refinement},
model reduction using \emph{Proper Orthogonal Decomposition}, and the
incorporation of observed windstress data, which will include \emph{parameter
estimation}. Additionally, the streamfunction formulation suffers from
non-unique streamfunctions for multiply connected domains, such as the ocean and
seas and large lakes with islands. To address this issue a method such as the
one developed by van Gijzen et. al. \cite{van-Gijzen1998} will be explored.
Thus, a more realistic Ocean mesh will be used and islands will no longer have
to be connected to continents.
