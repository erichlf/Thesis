The main goal of this section is to develop a rigorous numerical analysis for
\autoref{alg:TwoLevel}, the two-level algorithm introduced in
\autoref{sec:TwoLevel}. The proof for the error bounds follows a pattern that is
similar to that presented in \cite{Fairag98}.

To this end, we first introduce an improved bound on the trilinear form
$b(\zeta; \xi, \chi)$ using the discrete Sobolev inequality
\cite{Ciarlet,Fairag98}:
\begin{equation*}
  \|\nabla \varphi^h\|_{L^{\infty}} \le c \sqrt{|\ln(h)|}\cdot |\varphi^h|_2.
\end{equation*}
The following lemma follows from the above inequality and \eqref{eqn:BH2Bounds}:
\begin{lemma} \label{lma:bImproved}
  For any $\chi^h\in X^h$, the following inequalities hold:
  \begin{align*}
    |b(\psi;\chi^h,\xi)| &\le C\sqrt{|\ln(h)|} \cdot |\psi|_2 |\xi|_1 |\chi^h|_2, \\
    |b(\psi;\xi,\chi^h)| &\le C\sqrt{|\ln(h)|} \cdot |\psi|_2 |\xi|_1 |\chi^h|_2.
  \end{align*}
\end{lemma}
The following lemma will prove useful for proving the error bounds for
\autoref{alg:TwoLevel}:
\begin{lemma} \label{lma:trilinear}
  For $\psi,\,\xi,\,\chi\in H^2_0(\Omega)$, we have
  \begin{equation}
    b(\psi; \xi, \chi) = b^*(\xi; \chi, \psi) - b^*(\chi; \xi, \psi),
    \label{eqn:eqn:trilinear}
  \end{equation}
  where
  \begin{equation}
    b^*(\xi; \chi, \psi) = \int_{\Omega}\! (\chi_y\xi_{xy}-\xi_x\chi_{yy}) \psi_y -
    (\xi_y\chi_{yx}-\xi_y\chi_{xx}) \psi_x \,d\mathbf{x}.
    \label{eqn:trilinear}
  \end{equation}
\end{lemma}
For a proof see \cite{Fairag98}.

Before proving any error bounds we first prove that the continuous problem
linearized about $\psi^H$ has a unique solution.
\begin{lemma}\label{lma:Fine}
  Given a solution $\psi^H$ of \eqref{eqn:Coarse}, then the solution to the
  following problem exists uniquely:
    \begin{equation}
      \begin{split}
        &\text{Find } \hat{\psi} \in H^2_0(\Omega) \text{ such that, for all }
          \chi\in H^2_0(\Omega), \\
        Re^{-1}&(\Delta \hat{\psi}, \Delta \chi)
          + b(\psi^H; \hat{\psi}, \chi)
          - Ro^{-1} (\hat{\psi}_x,\chi)
          = Ro^{-1} (F,\chi),
      \end{split}
      \label{eqn:FineProb}
    \end{equation}
    and satisfies $\|\hat{\psi}\|_2 \le Re\, Ro^{-1} \|F\|_{-2}$.
\end{lemma}
\begin{proof}
  First we introduce a new continuous bilinear form $B:\, H^2_0(\Omega) \times
  H^2_0(\Omega) \to \R$ given by
  \begin{equation*}
    B(\psi,\chi) = Re^{-1} (\Delta \psi, \Delta \chi)
      + b(\psi^H;\psi,\chi)
      - Ro^{-1} (\psi_x,\chi).
  \end{equation*}
  B is continuous and coercive and therefore $\hat{\psi}$ exists and is unique.
  Now setting $\chi=\hat{\psi}$ in \eqref{eqn:FineProb} and noting that
  $(\psi_x,\chi) = -(\chi_x,\psi)$ which implies that
  $(\hat{\psi}_x,\hat{\psi}) = 0$ gives
  \begin{align*}
    Re^{-1} \|\hat{\psi}\|_2^2 &= Ro^{-1} (F,\hat{\psi}) \\
    \|\hat{\psi}\|_2 &= Re\, Ro^{-1} \frac{(F,\hat{\psi})}{\|\hat{\psi}\|_2}.
  \end{align*}
  Taking the supremum over all $\hat{\psi}\in X$ leads to
  \begin{equation*}
    \|\hat{\psi}\|_{2} \le Re\, Ro^{-1} \sup_{\hat{\psi}\in X} \frac{(F,
      \hat{\psi})}{|\hat{\psi}\|_2}.
  \end{equation*}
  Therefore, by definition it follows that $\|\hat{\psi}\|_2 \le Re\, Ro^{-1} \|F\|_{-2}$.
\end{proof}
The following lemma shows that the solution to the fine mesh problem,
\eqref{eqn:Fine} exists and has a stability bound dependent on $Re,\, Ro$, and
the forcing function $F$.
\begin{lemma} \label{lma:Fineh}
  The solution to \eqref{eqn:Fine} exists and satisfies
  \begin{equation*}
    \|\psi^h\|_2 \le Re\, Ro^{-1} \|F\|_{-2}.
  \end{equation*}
\end{lemma}
\begin{proof}
  The bilinear form $B$ is continuous and coercive on $X^h$ and so $\psi^h$
  exists and is unique. Setting $\chi^h=\psi^h$ in \eqref{eqn:Fine} and again
  noting that $(\psi_x^h,\psi^h)=0$ and using \eqref{eqn:lCont} gives
  \begin{align*}
    Re^{-1} \|\psi^h\|_2^2 &= Ro^{-1} (F,\psi^h) \\
    \|\psi^h\|_2 &= Re Ro^{-1} \frac{(F,\psi^h)}{\|\psi^h\|_2},
  \end{align*}
  which implies
  \begin{equation*}
    \|\psi^h\| \le Re\, Ro^{-1} \|F\|_{-2}.
  \end{equation*}
\end{proof}

The following theorem gives the error bound after Step 2 and is the main result
of this section. The proof of this theorem is similar to the proof for a similar
theorem in \cite{Fairag98}.
\begin{thm} \label{thm:2LTwoLevel}
  Let $X^h,\, X^H\subset H^2_0(\Omega)$ be two finite element spaces. Let $\psi$ be
  the solution to \eqref{eqn:SQGEWF} and $\psi^h$ the solution to
  \eqref{eqn:Fine}. Then $\psi^h$ satisfies
  \begin{equation}
    |\psi-\psi^h|_2 \le C_1 \inf_{\lambda^h\in X^h} |\psi-\lambda^h|_2 + C_2
      \sqrt{|\ln h|}\cdot |\psi - \psi^H|_1,
    \label{eqn:Error}
  \end{equation}
  where $C_1 = 2 + Re\,Ro^{-1} + Re^2 Ro^{-1} \Gamma_1 \|F\|_{-2}$ and $C_2= 2
  Re^2 Ro^{-1} \Gamma_1 C\,\|F\|_{-2}$.
\end{thm}
\begin{proof}
  Subtracting \eqref{eqn:Fine} from \eqref{eqn:SQGEWF} and letting $\chi=\chi^h$
  yields
  \begin{equation*}
    Re^{-1} (\Delta \left[\psi - \psi^h\right], \Delta \chi^h) + b(\psi;\psi,\chi^h)
      - b(\psi^H;\psi^h,\chi^h) - Ro^{-1} (\left[\psi-\psi^h\right]_x,\chi^h)
      = 0, \quad \forall \chi^h \in X^h.
  \end{equation*}
  Using \autoref{lma:trilinear} gives
  \begin{equation*}
    \begin{split}
      Re^{-1} (\Delta \left[\psi - \psi^h\right], \Delta \chi^h)
        &+ b^*(\psi;\chi^h, \psi) - b^*(\chi^h;\psi,\psi) \\
      &- b^*(\psi^h;\chi^h,\psi^H) + b^*(\chi^h; \psi^h,\psi^H)
        - Ro^{-1} (\left[\psi-\psi^h\right]_x,\chi^h) = 0,
      \quad \forall \chi^h \in X^h.
    \end{split}
  \end{equation*}
  Now, adding the terms
  \begin{equation*}
    -b^*(\psi^h;\chi^h,\psi) + b^*(\chi^h;\psi^h,\psi) + b^*(\psi^h;\chi^h,\psi) - b^*(\chi^h;\psi^h,\psi)
  \end{equation*}
  gives
  \begin{equation*}
    \begin{split}
      &Re^{-1} (\Delta \left[\psi - \psi^h\right], \Delta \chi^h)
        + b^*(\psi-\psi^h;\chi^h, \psi) + b^*(\chi^h;\psi^h-\psi,\psi) \\
      &\quad+ b^*(\psi^h;\chi^h,\psi-\psi^H) + b^*(\chi^h; \psi^h,\psi^H-\psi)
        - Ro^{-1} (\left[\psi-\psi^h\right]_x,\chi^h) = 0,
    \quad \forall \chi^h \in X^h.
      \end{split}
  \end{equation*}
  Take $\lambda^h\in H^2_0(\Omega)$ arbitrary and define $e:= \psi - \psi^h =
  \eta - \Phi^h$, where $\Phi^h = \psi^h-\lambda^h$ and $\eta=\psi-\lambda^h$.
  We have
  \begin{equation*}
    \begin{split}
      &Re^{-1}(\Delta \Phi^h, \Delta \chi^h)
        + b^*(\Phi^h;\chi^h, \psi) - b^*(\chi^h;\Phi^h,\psi)
        - Ro^{-1} (\Phi^h_x,\chi^h) \\
      &\quad = Re^{-1} (\eta,\chi^h)
        + b^*(\eta;\chi^h, \psi) - b^*(\chi^h;\eta,\psi) \\
      &\quad+ b^*(\psi^h;\chi^h,\psi-\psi^H) + b^*(\chi^h; \psi^h,\psi^H-\psi)
        - Ro^{-1} (\eta_x,\chi^h),
      \quad \forall \chi^h \in X^h.
    \end{split}
  \end{equation*}
  Since this holds for any $\chi^h\in H^2_0(\Omega)$, it holds in particular for
  $\chi^h=\Phi^h\in H^2_0(\Omega)$, which implies
  \begin{equation*}
    \begin{split}
      Re^{-1} (\Delta \Phi^h, \Delta \Phi^h) - Ro^{-1} (\Phi^h_x,\Phi^h)
        &= Re^{-1} (\Delta \eta, \Delta \Phi^h)
        + b^*(\eta;\Phi^h, \psi) - b^*(\Phi^h;\eta,\psi) \\
      &\quad+ b^*(\psi^h;\Phi^h,\psi-\psi^H) + b^*(\Phi^h; \psi^h,\psi^H-\psi)
        - Ro^{-1} (\eta_x,\Phi^h).
    \end{split}
  \end{equation*}
  Note that $(\Phi_x,\Phi) = -(\Phi,\Phi_x)$ and so it follows that
  $(\Phi^h_x,\Phi^h) = 0$. This combined with \autoref{lma:trilinear} implies
  \begin{equation*}
    \begin{split}
      Re^{-1} (\Delta \Phi^h, \Delta \Phi^h) &= Re^{-1} (\Delta \eta, \Delta \Phi^h)
        + b(\psi;\eta,\Phi^h) \\
      &\quad+ b^*(\psi^h;\Phi^h,\psi-\psi^H) + b^*(\Phi^h; \psi^h,\psi^H-\psi)
        - Ro^{-1} (\eta_x,\Phi^h).
    \end{split}
  \end{equation*}
  Using the error bounds given in \eqref{eqn:a1Cont}, \eqref{eqn:BH2Bounds},
  \eqref{eqn:a3Cont}, \eqref{eqn:lCont}, \autoref{lma:bImproved},
  \autoref{thm:stability_sqge}, \autoref{thm:stability_fem_sqge} and
  \autoref{lma:Fine} gives
  \begin{align*}
    Re^{-1} |\Phi^h|_2^2 &\le Re^{-1} |\eta|_2\, |\Phi^h|_2 + \Gamma_1\,
      |\psi|_2\, |\eta|_2\, |\Phi^h|_2 \\
    &\quad+ 2 \Gamma_1\, C\, |\psi^H|_2\, |\Phi^h|_2\, |\psi - \psi^H|_1
      \sqrt{|\ln(h)|} + Ro^{-1}\, \Gamma_2 |\eta|_2\, |\Phi^h|_2 \\
    &= \left(Ro^{-1}\, \Gamma_2 + Re^{-1} + \Gamma_1\, |\psi|_2\right) |\eta|_2\,
      |\Phi^h|_2 + 2 \Gamma_1\, C\, |\psi^H|_2\, |\Phi^h|_2 |\psi - \psi^H|_1
      \sqrt{|\ln(h)|}  \\
    |\Phi^h|_2 &\le \left(1 + Re\, Ro^{-1}\, \Gamma_2 + Re^2 Ro^{-1} \Gamma_1\,
      \|F\|_{-2}\right) |\eta|_2 \\
      &\quad + 2 Re^2 Ro^{-1} \Gamma_1\, C\, \|F\|_{-2}\, |\psi - \psi^H|_1 \sqrt{|\ln(h)|} \\
  \end{align*}
  Adding $|\eta|_2$ to both sides and using the triangle inequality ($|\psi -
  \psi^h|_2 \le |\Phi^h|_2 + |\eta|_2$) gives
  \begin{align*}
    |\Phi^h|_2 &\le \left(2 + Re\, Ro^{-1}\, \Gamma_2 + Re^2 Ro^{-1} \Gamma_1\,
      \|F\|_{-2}\right) |\eta|_2 \\
    &\quad + 2 Re^2 Ro^{-1} \Gamma_1\, C\, \|F\|_{-2}\, |\psi - \psi^H|_1
      \sqrt{|\ln(h)|} \\
  \end{align*}
  Thus, we have the following estimate for the error bounds:
  \begin{equation*}
    |\psi-\psi^h|_2 \le C_1 \inf_{\lambda^h\in X^h} |\psi-\lambda^h|_2 + C_2
      \sqrt{|\ln h|}\cdot |\psi - \psi^H|_1
  \end{equation*}
  where $C_1 = 2 + Re\,Ro^{-1}\, \Gamma_2 + Re^2 Ro^{-1} \Gamma_1 \|F\|_{-2}$
  and $C_2= 2 Re^2 Ro^{-1} \Gamma_1 C\,\|F\|_{-2}$.
\end{proof}

As an example, consider the case of the Argyris triangle. For this element we
have the following inequalities, which follow from approximation theory
\cite{Bernadou94} and Theorem 6.1.1 \cite{Ciarlet}:
\begin{align}
  |\psi - \psi^h|_j &\le Ch^{6-j} \label{eqn:fineOrder} \\
  |\psi - \psi^H|_j &\le CH^{6-j} \label{eqn:coarseOrder}
\end{align}
\begin{corollary} \label{crl:Argyris2L}
  Let $X^h,\, X^H \in H^2_0(\Omega)$ be the Argyris finite elements. Then,
  $\psi^h$ satisfies
  \begin{equation}
    |\psi - \psi^h|_2 \le C_1 h^4 + C_2 \sqrt{|\ln(h)|}\, H^5.
    \label{eqn:TwoLevelError}
  \end{equation}
\end{corollary}
\begin{proof}
  This follows directly by substituting \eqref{eqn:fineOrder} and
  \eqref{eqn:coarseOrder} into \eqref{eqn:Error}.
\end{proof}
