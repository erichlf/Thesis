We consider an approximate solution to \eqref{eqn:SQGEWF} by a two-level finite
element procedure \cite{Fairag98,Layton93}. Let $X^h,\, X^H \subset
H^2_0(\Omega)$ denote two conforming finite element spaces with $H \gg h$. We
compute an approximate solution $\psi^h$ in the finite element space $X^h$ by
solving a linear system consisting of the degrees of freedom in $X^h$.  This
linear system requires us to first compute the approximate solution $\psi^H$ to
the nonlinear system in the finite element space $X^H$ where the mesh is very
coarse, i.e. $H \gg h$ and then using this solution, $\psi^H$, in the linear
system. This procedure is as follows

\begin{algorithm}%[H]
  \caption{}%Two-Level algorithm for the Streamfunction formulation of QGE}
  \label{alg:TwoLevel}
  \begin{enumerate}[Step 1:]
    \item Solve the nonlinear system on a coarse mesh for $\psi^H\in X^H$:
    \begin{equation}
      Re^{-1} (\Delta \psi^H, \Delta \chi^H)
        + b(\psi^H; \psi^H,\chi^H)
        - Ro^{-1} (\psi_x^H,\chi^H)
        = Ro^{-1} (F,\chi^H), \quad \text{for all } \chi^H \in X^H.
      \label{eqn:Coarse}
    \end{equation}
    \item Solve the linear system on a fine mesh for $\psi^h\in X^h$:
    \begin{equation}
      Re^{-1} (\Delta \psi^h, \Delta \chi^h)
        + b(\psi^H; \psi^h,\chi^h)
        - Ro^{-1} (\psi_x^h,\chi^h)
        = Ro^{-1} (F,\chi^h), \quad \text{for all } \chi^h \in X^h.
      \label{eqn:Fine}
    \end{equation}
  \end{enumerate}
\end{algorithm}
\begin{lemma}\label{lma:Fine}
  Given a solution $\psi^H$ of \eqref{eqn:Coarse}, then the solution to the
  following problem exists uniquely
    \begin{equation}
      \begin{split}
        &\text{Find } \hat{\psi} \in H^2_0(\Omega) \text{ such that, for all }
          \chi\in H^2_0(\Omega), \\
        Re^{-1}&(\Delta \hat{\psi}, \Delta \chi)
          + b(\psi^H; \hat{\psi}, \chi)
          - Ro^{-1} (\hat{\psi}_x,\chi)
          = Ro^{-1} (F,\chi),
      \end{split}
      \label{eqn:FineProb}
    \end{equation}
    and satisfies $\|\hat{\psi}\|_2 \le Re\, Ro^{-1} \|F\|_{-2}$.
\end{lemma}
\begin{proof}
  First we introduce a new continuous bilinear form $B:\, H^2_0(\Omega) \times
  H^2_0(\Omega) \to \R$ given by
  \begin{equation*}
    B(\psi,\chi) = Re^{-1} (\Delta \psi, \Delta \chi)
      + b(\psi^H;\psi,\chi)
      - Ro^{-1} (\psi_x,\chi).
  \end{equation*}
  B is continuous and coercive and therefore $\hat{\psi}$ exists and is unique.
  Now setting $\chi=\hat{\psi}$ in \eqref{eqn:FineProb} and noting that
  $(\psi_x,\chi) = -(\chi_x,\psi)$ which implies that
  $(\hat{\psi}_x,\hat{\psi}) = 0$ gives
  \begin{align*}
    Re^{-1} \|\hat{\psi}\|_2^2 &= Ro^{-1} (F,\hat{\psi}) \\
    \|\hat{\psi}\|_2 &= Re\, Ro^{-1} \frac{(F,\hat{\psi})}{\|\hat{\psi}\|_2} \\
    &\le Re\, Ro^{-1} \|F\|_{-2}\|\hat{\psi}\|_2 \\
  \end{align*}
  Therefore, it follows that $\|\hat{\psi}\|_2 \le Re\, Ro^{-1} \|F\|_{-2}$.
\end{proof}
\begin{lemma} \label{lma:Fineh}
  The solution to \eqref{eqn:Coarse} exists and satisfies
  \begin{equation*}
    \|\psi^H\|_2 \le Re\, Ro^{-1} \|F\|_{-2}.
  \end{equation*}
\end{lemma}
\begin{proof}
  The bilinear form $B$ is continuous and coercive on $X^h$ and so $\psi^h$
  exists and is unique. Setting, $\chi^h=\psi^h$ in \eqref{eqn:Fine} and again
  noting that $(\psi_x^h,\psi^h)=0$ and using \eqref{eqn:lCont} gives
  \begin{align*}
    Re^{-1} \|\psi^h\|_2^2 &= Ro^{-1} (F,\psi^h) \\
    \|\psi^h\|_2 &= Re Ro^{-1} \frac{(F,\psi^h)}{\|\psi^h\|_2} \\
    &\le Re\, Ro^{-1} \|F\|_{-2}.
  \end{align*}
\end{proof}
