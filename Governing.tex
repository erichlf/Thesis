One of the most popular mathematical models used in the study of large scale
wind-driven ocean circulation is the QGE \cite{Cushman11,Vallis06}. The QGE
represent a simplified model of the full-fledged equations (e.g., the Boussinesq
equations), which allows efficient numerical simulations while preserving many
of the essential features of the underlying large scale ocean flows. The
assumptions used in the derivation of the QGE include the hydrostatic balance,
the $\beta$-plane approximation, the geostrophic balance, and the eddy viscosity
parametrization.  Details of the derivation of the QGE and the approximations
used along the way can be found in standard textbooks on geophysical fluid
dynamics, such as
\cite{Cushman11,Majda,Majda03,McWilliams06,Pedlosky92,Vallis06}.

In the \emph{one-layer QGE}, sometimes called the barotropic vorticity equation,
the flow is assumed to be homogenous in the vertical direction. Thus,
stratification effects are ignored in this model.  The practical advantages of
such a choice are obvious: the computations are two-dimensional, and, thus, the
corresponding numerical simulation have a low computational cost. To include
stratification effects, QGE models of increasing complexity have been devised by
increasing the number of layers in the model (e.g., the two-layer QGE and the
$N$-layer QGE \cite{Vallis06}). As a first step, in this report we use the
one-layer QGE (referred to as ``the QGE" in what follows) to study the
wind-driven circulation in an enclosed, midlatitude rectangular basin, which is
a standard problem, studied extensively by modelers \cite{Cushman11, Layton08,
Majda03, Majda, McWilliams06, Vallis06, Pedlosky92}.

