In this section we present the convergence, and the error analysis associated
with the semi-discretization. We will rely heavily on the previous section
containing the finite error analysis for the SQGE (\autoref{sec:SQGEErrors}).

\begin{prop} \label{prop:Stability}
  The solution to \eqref{eqn:SemiDiscretization}, $\psi^h$, is stable and for any $t>0$
  \begin{equation}
    \frac{1}{2}\|\nabla \psi^h(t)\|^2 + \frac{Re^{-1}}{2}\int_{0}^{t}\! \|\Delta
      \psi^h(t')\|^2 \, dt' \le \frac{1}{2} \|\nabla \psi^h_0\|^2
      + \frac{Re\, Ro^{-2}}{2} \int_{0}^{t}\! \|F(t')\|^2_{-1}\, dt'.
    \label{eqn:Stability}
  \end{equation}
\end{prop}
\begin{proof}
  Take $\chi^h = \psi^h$ in \eqref{eqn:SemiDiscretization} and note $b(\psi^h;\psi^h,\psi^h)
  = 0$ while since $(\psi^h_x,\psi^h) = -(\psi^h,\psi^h_x)$ then $(\psi^h_x,
  \psi^h) = 0$. Thus, we have
  \begin{equation*}
    \frac{1}{2} \frac{d}{dt} \|\nabla \psi^h\|^2 + Re^{-1} \|\Delta \psi^h\|^2 =
      Ro^{-1} (F,\psi^h).
  \end{equation*}
  Using the H\"older, Poincar\'e, and Young inequalities gives
  \begin{equation*}
    \frac{1}{2} \frac{d}{dt} \|\nabla \psi^h\|^2 + Re^{-1} \|\Delta \psi^h\|^2 =
      \frac{Ro^{-2} C_p}{2\epsilon} \|F\|_{-1}^2 + \frac{\epsilon}{2}\|\nabla
      \psi^h\|^2.
  \end{equation*}
  Now taking $\epsilon = Re^{-1}$ results in
  \begin{equation*}
    \frac{1}{2} \frac{d}{dt} \|\nabla \psi^h\|^2 + \frac{Re^{-1}}{2} \|\Delta
      \psi^h\|^2 = \frac{Re\,Ro^{-2} C_p}{2} \|F\|_{-1}^2.
  \end{equation*}
  Assuming $\|\nabla \psi^h\| \in L^1(0,T)$ and integrating over $(0,t)$ gives
  the final result.
\end{proof}

\begin{lemma} \label{lma:BH1Bound}
  There are finite constants $\Gamma_3,\Gamma_4>0$ such hat for all
  $\psi,\, \varphi,\, \chi \in X$
  \begin{align}
    b(\psi;\varphi,\chi) &\le \Gamma_3 \|\Delta \psi\|\, \|\Delta \varphi\|\,
      \left(\|\nabla \chi\|^{\nicefrac{1}{2}}
      \|\Delta \chi\|^{\nicefrac{1}{2}}\right) \label{eqn:BH1BoundChi} \\
    b(\psi;\varphi,\chi) &\le \Gamma_4 \left(\|\nabla \psi\|^{\nicefrac{1}{2}}
      \|\Delta \psi\|^{\nicefrac{1}{2}}\right)\,
      \|\Delta \varphi\|\, \|\Delta \chi\| \label{eqn:BH1BoundPsi}
  \end{align}
\end{lemma}
\begin{proof}
  We will begin with the proof of \eqref{eqn:BH1BoundChi} and then prove \eqref{eqn:BH1BoundPsi} last. For \eqref{eqn:BH1BoundChi} we begin in the
  same way as we did for \autoref{eqn:BH2Bounds} in
  \autoref{lma:ContinuousForms}. That is, we apply the H\"older inequality to
  $b(\psi; \varphi, \chi)$ and set $p=2$, while setting $q=r=4$, giving
  \begin{equation*}
    b(\psi;\varphi,\chi) \le \|\Delta \psi\| \|\nabla \varphi\|_{L^4} \|\nabla
      \chi\|_{L^4}.
  \end{equation*}
  Now, applying the Ladyzhenskaya inequality we get
  \begin{equation*}
    b(\psi;\varphi,\chi) \le \Gamma \|\Delta \psi\|
      \|\nabla \varphi\|^{\nicefrac{1}{2}} \|\Delta \varphi\|^{\nicefrac{1}{2}}
      \|\nabla \chi\|^{\nicefrac{1}{2}} \|\Delta \chi\|^{\nicefrac{1}{2}}.
  \end{equation*}
  Next we apply the Poincar\'e inequality to $\|\nabla \chi\|$ resulting in
  \begin{equation*}
    b(\psi;\varphi,\chi) \le \Gamma_3 \|\Delta \psi\|\, \|\Delta \varphi\|\,
      \left(\|\nabla \chi\|^{\nicefrac{1}{2}}
      \|\Delta \chi\|^{\nicefrac{1}{2}}\right),
  \end{equation*}
  For \autoref{eqn:BH1BoundPsi} we apply Lemma 5.6 from \cite{Fairag98}, i.e.
  \begin{equation*}
    b(\psi; \varphi, \chi) = b_0(\varphi; \chi, \psi) - b_0(\chi; \varphi, \psi),
  \end{equation*}
  where
  \begin{equation}
    b_0(\varphi;\chi,\psi) = \int_{\Omega}\!
      (\varphi_{xy} \chi_y - \varphi_{yy} \chi_x) \psi_y
      - (\varphi_{xy} \chi_x - \varphi_{xx} \chi_y) \psi_x \, d\vec{x}.
    \label{eqn:b0def}
  \end{equation}
  Therefore using \eqref{eqn:b0def}, H\"older inequality reads
  \begin{equation}
    b(\psi; \varphi, \chi) \le \|\nabla \psi\|_{L^p} \|\Delta \varphi\|_{L^q}
    \|\nabla \chi\|_{L^r},\quad \frac{1}{p} + \frac{1}{q} + \frac{1}{r} = 1.
    \label{eqn:BStarHolder}
  \end{equation}
  Letting $p = r = 4$, and $q = 2$ in \eqref{eqn:BStarHolder} gives
  \begin{equation*}
    b(\psi; \varphi, \chi) \le \|\nabla \psi\|_{L^4} \|\Delta \varphi\|
    \|\nabla \chi\|_{L^4}.
  \end{equation*}
  Then applying the Ladyzhenskaya inequality we get
  \begin{equation*}
    b(\psi;\varphi,\chi) \le \Gamma \|\Delta \varphi\|
      \|\nabla \psi\|^{\nicefrac{1}{2}} \|\Delta \psi\|^{\nicefrac{1}{2}}
      \|\nabla \chi\|^{\nicefrac{1}{2}} \|\Delta \chi\|^{\nicefrac{1}{2}}.
  \end{equation*}
  Finally, applying the Poincar\'e inequality give the desired result
  \begin{equation*}
    b(\psi;\varphi,\chi) \le \Gamma_4 \left(\|\nabla \psi\|^{\nicefrac{1}{2}}
      \|\Delta \psi\|^{\nicefrac{1}{2}}\right)\,
      \|\Delta \varphi\|\, \|\Delta \chi\|
  \end{equation*}
\end{proof}

\begin{thm} \label{thm:StrongConvergence}
  Let $\psi$ be a strong solution to the QGE, then there exists a \\ $C(T, F,
  \psi_0, \Gamma_1, \Gamma_2, Re, Ro,\int_0^T\! |\Delta \psi|^4\, dt)$ such that
  for all $t\in [0,T]$
  \begin{equation}
    \begin{split}
      \|\nabla (\psi &- \psi^h) \|^2 + Re^{-1}
        \int_{0}^{T}\! \|\Delta (\psi - \psi^h)\|^2 \, dt \le C\biggl\{
        \|\nabla(\psi_0 - \psi^h(0))\|^2 \\
      & + \inf_{\chi^h(t) \in X^h} \biggl[
        \int_0^T\! \|\nabla(\psi - \chi^h)_t\|^2_{-1}
        + Re^{-1} \|\Delta(\psi - \chi^h)\|^2\, dt \\
      & \qquad + \|\Delta(\psi - \chi^h)\|_{L^4(0,T;L^2)}
        + \|\nabla (\psi - \chi^h)\|^2\biggr] \biggr\}
    \end{split}
    \label{eqn:StrongConvergence}
  \end{equation}
\end{thm}
\begin{proof}
  Let $\chi = \chi^h \in X^h$ and subtract \eqref{eqn:SemiDiscretization} from
  \eqref{eqn:QGEWF} then let $e:=\psi - \psi^h$ and we obtain
  \begin{equation*}
    (\nabla e_t, \nabla \chi^h) + \left[b(\psi;\psi,\chi^h) - b(\psi^h;\psi^h,\chi^h)\right]
      + Re^{-1}(\Delta e, \Delta \chi^h) - Ro^{-1} (e_x, \chi^h) = 0\quad
      \forall \chi^h \in X^h \subset X.
  \end{equation*}
  Now adding and subtracting $b(\psi^h;\psi,\chi^h)$ gives
  \begin{equation*}
    (\nabla e_t, \nabla \chi^h) + \left[b(e;\psi,\chi^h) + b(\psi^h;e,\chi^h)\right]
      + Re^{-1}(\Delta e, \Delta \chi^h) - Ro^{-1} (e_x, \chi^h) = 0\quad
      \forall \chi^h \in X^h \subset X.
  \end{equation*}
  Take $\omega^h:[0,T] \to X^h$ arbitrary and decomposing the error $e = \eta -
  \Phi^h$, where $\eta := \psi - \omega^h$ and $\Phi^h := \psi^h - \omega^h$,
  results in
  \begin{align*}
    (\nabla \Phi^h_t, \nabla \chi^h) + Re^{-1}(\Delta \Phi^h, \Delta \chi^h)
      & = (\nabla \eta_t, \nabla \chi^h) + Re^{-1}(\Delta \eta, \Delta \chi^h)
      - Ro^{-1} \left[(\eta_x, \chi^h) - (\Phi^h_x, \chi^h)\right] \\
    & + \left[ b(\eta;\psi,\chi^h) - b(\Phi^h;\psi,\chi^h)
      + b(\psi^h;\eta,\chi^h) - b(\psi^h;\Phi^h,\chi^h)\right].
  \end{align*}
  Let $\chi^h = \Phi^h$ then noting $b(\psi^h;\Phi^h,\Phi^h) = 0$ and $(\Phi^h_x,
  \Phi^h) = -(\Phi^h,\Phi^h_x)$ implies $(\Phi^h_x,\Phi^h) = 0$
  \begin{align*}
    \frac{1}{2} \frac{d}{dt} \|\nabla \Phi^h\|^2 + Re^{-1}\|\Delta \Phi^h\|^2
       = (\nabla \eta_t, \nabla \Phi^h) &+ Re^{-1}(\Delta \eta, \Delta \Phi^h)
      - Ro^{-1} (\eta_x, \Phi^h) \\
    & + \left[ b(\eta;\psi,\Phi^h) - b(\Phi^h;\psi,\Phi^h)
      + b(\psi^h;\eta,\Phi^h)\right].
  \end{align*}
  Using the H\"older inequality, bounds for $(\psi_x,\chi)$ given in
  \autoref{lma:ContinuousForms}, and the bounds for $b$ from
  \autoref{eqn:BH2Bounds} in \autoref{lma:ContinuousForms}, and \autoref{lma:BH1Bound} we have
  \begin{equation}
    \begin{split}
      \frac{1}{2} \frac{d}{dt} \|\nabla \Phi^h\|^2 + Re^{-1}\|\Delta \Phi^h\|^2
        &= \|\nabla \eta_t\|_{-1} \|\Delta \Phi^h\|
        + Re^{-1}\|\Delta \eta\|\, \|\Delta \Phi^h\|
        + Ro^{-1} \Gamma_2 \|\Delta \eta\| \|\Delta \Phi^h\| \\
      & + \left[ b(\eta;\psi,\Phi^h) - b(\Phi^h;\psi,\Phi^h)
        + b(\psi^h;\eta,\Phi^h)\right].
    \end{split}
    \label{eqn:HolderError}
  \end{equation}
  Then using the Young inequality we have, for all $\epsilon>0$
  \begin{align}
    \|\nabla \eta_t\|_{-1} \|\Delta \Phi^h\|
      &\le \frac{\epsilon}{2} \|\Delta \Phi^h\|^2
      + \frac{1}{2 \epsilon} \|\eta_t\|_{-1}^2 \label{eqn:YoungT} \\
    Re^{-1} \|\Delta \eta\| \|\Delta \Phi^h\|
      &\le \frac{\epsilon}{2} \|\Delta \Phi^h\|^2
      + \frac{Re^{-2}}{2 \epsilon} \|\Delta \eta\|^2 \label{eqn:YoungLaplace} \\
    Ro^{-1} \Gamma_2 \|\Delta \eta\| \|\Delta \Phi^h\|
      &\le \frac{\epsilon}{2} \|\Delta \Phi^h\|^2
      + \frac{Ro^{-2} \Gamma_2^2}{2 \epsilon} \|\Delta \eta\|^2. \label{eqn:YoungBeta}
  \end{align}
  Using the Young inequality with $\varepsilon = 2 > 0$ and
  \autoref{eqn:BH2Bounds} in \autoref{lma:ContinuousForms}
  \begin{align*}
    b(\eta;\psi,\Phi^h) &\le \Gamma_1 \|\Delta \eta\|\,\|\Delta \psi\|\, \|\Delta \Phi^h\| \\
    &\le \frac{\varepsilon}{2} \|\Delta \Phi^h\|^2
      + \frac{\Gamma_1^2}{2 \varepsilon} \|\Delta \eta\|^2 \|\Delta \psi\|^2, \\
  \end{align*}
  and letting $\varepsilon = 2 \epsilon$ we have
  \begin{equation}
    b(\eta; \psi, \Phi^h) \le \epsilon \|\Delta \Phi^h\|^2
      + \frac{\Gamma_1^2}{4 \epsilon} \|\Delta \eta\|^2 \|\Delta \psi\|^2.
      \label{eqn:YoungBH2}
  \end{equation}
  Combing \eqref{eqn:YoungT} - \eqref{eqn:YoungBH2} and \eqref{eqn:HolderError}
  we obtain
  \begin{equation}
    \begin{split}
    \frac{1}{2} \frac{d}{dt} \|\nabla \Phi^h\|^2 + \frac{1}{2}\left(2Re^{-1} -
      5 \epsilon \right)\|\Delta \Phi^h\|^2
      &\le \frac{1}{2 \epsilon}\left[\|\nabla \eta_t\|_{-1}^2
      + \left( Re^{-2} + Ro^{-2} \Gamma_2^2 \right) \|\Delta \eta\|^2\right] \\
     & + \frac{\Gamma_1^2}{4 \epsilon}\|\Delta \eta\|^2 \|\Delta \psi\|^2  -
     \left[b(\Phi^h;\psi,\Phi^h) - b(\psi^h;\eta,\Phi^h)\right].
    \end{split}
    \label{eqn:B1Inequality}
  \end{equation}
  For the term $b(\Phi^h; \psi, \Phi^h)$ we use \autoref{lma:BH1Bound} and the
  following version of the Young inequality (from \cite{Layton08} Equation (1.1.4)):
  given $a,\,b>0,$ for any $\epsilon > 0,\, 1\le p \le \infty,\, \frac{1}{p} +
  \frac{1}{q} = 1$, and
  $C(\epsilon,p,q)=\dfrac{\left(p\,\epsilon\right)^{-\nicefrac{q}{p}}}{q}$
  \begin{equation*}
    ab \le \epsilon\, a^p + C(\epsilon,p,q)\, b^q
  \end{equation*}
  with $p=\nicefrac{4}{3},\, q = 4$ to obtain
  \begin{align}
    b(\Phi^h; \psi, \Phi^h) &\le \Gamma_3\, \|\Delta \Phi^h\|^{\nicefrac{3}{2}}
      \left(\|\Delta \psi\| \|\nabla \Phi^h\|^{\nicefrac{1}{2}}\right)
      \nonumber \\
    &\le \epsilon \|\Delta \Phi^h\|^2 + C^*_1(\Gamma_3,\epsilon) \|\Delta \psi\|^4
      \|\nabla \Phi^h\|^2,
    \label{eqn:EpsYoungH1}
  \end{align}
  where $C^*_1(\Gamma_3,\epsilon) = \nicefrac{27}{256}\,\Gamma_3^4\,\epsilon^{-3}$.
  Combining \eqref{eqn:B1Inequality} and \eqref{eqn:EpsYoungH1} gives
  \begin{equation}
    \begin{split}
      \frac{1}{2} \frac{d}{dt} \|\nabla \Phi^h\|^2 + \frac{1}{2}\left(2Re^{-1} -
        7 \epsilon \right)
        &\|\Delta \Phi^h\|^2 \le \frac{1}{2 \epsilon}\left[\|\nabla \eta_t\|_{-1}^2
        + \left( Re^{-2} + Ro^{-2} \Gamma_2^2 \right) \|\Delta \eta\|^2\right] \\
      & + \frac{\Gamma_1^2}{4\epsilon}\|\Delta \eta\|^2 \|\Delta \psi\|^2
        + C^*_1(\Gamma_3,\epsilon) \|\Delta \psi\|^4 \|\nabla \Phi^h\|^2
        + b(\psi^h;\eta,\Phi^h).
    \end{split}
    \label{eqn:B2Inequality}
  \end{equation}
  For the final term involving $b(\psi^h; \eta, \Phi^h)$ we use
  Inequality \eqref{eqn:BH1BoundPsi} and the Young inequality with $\varepsilon =
  2 \epsilon$, i.e.
  \begin{align}
    b(\psi^h; \eta, \Phi^h) &\le \Gamma_4\left(\|\nabla \psi^h\|^{\nicefrac{1}{2}}
      \|\Delta \psi^h\|^{\nicefrac{1}{2}}\right) \|\Delta \eta\|\,
      \|\Delta \Phi^h\| \nonumber \\
    &\le \epsilon \|\Delta \Phi^h\|^2 + \frac{\Gamma_4^2}{4\epsilon}
      \|\nabla \psi^h\|\, \|\Delta \eta\|^2 \label{eqn:YoungPhih}
  \end{align}
  and by the stability estimate in \autoref{prop:Stability}, we have
  \begin{equation}
    \|\nabla \psi^h\| \le C^*_2(F,\psi_0, Re, Ro).
    \label{eqn:StabilityBoundPsih}
  \end{equation}
  Using \eqref{eqn:StabilityBoundPsih}, estimate \eqref{eqn:YoungPhih} becomes
  \begin{equation}
    b(\psi^h; \eta, \Phi^h) \le \epsilon \|\Delta \Phi^h\|^2 +
      \frac{\Gamma_4^2}{4\epsilon} C^*_2(F,\psi_0,Re,Ro) \|\Delta \psi^h\|\,
      \|\Delta \eta\|^2.
    \label{eqn:bPsihbound}
  \end{equation}
  Combining \eqref{eqn:B2Inequality} and \eqref{eqn:bPsihbound} gives
  \begin{equation}
    \begin{split}
      \frac{1}{2} \frac{d}{dt} \|\nabla \Phi^h\|^2 + &\frac{1}{2}\left(2Re^{-1} -
        9 \epsilon \right)
        \|\Delta \Phi^h\|^2 \le \frac{1}{2 \epsilon}\left[\|\nabla \eta_t\|_{-1}^2
        + \left( Re^{-2} + Ro^{-2} \Gamma_2^2 \right) \|\Delta \eta\|^2\right] \\
      & + \frac{\Gamma_1^2}{4 \epsilon} \|\Delta \psi\|^2 \|\Delta \eta\|^2
        + \frac{\Gamma_4}{4\epsilon}C^*_2(F,\psi_0,Re,Ro) \|\Delta \psi^h\|\,
        \|\Delta \eta\|^2 + C^*_1(\Gamma_3,\epsilon) \|\Delta \psi\|^4 \|\nabla \Phi^h\|^2.
    \end{split}
    \label{eqn:B3Inequality}
  \end{equation}
  Take $\epsilon = \nicefrac{Re^{-1}}{9}$ in \eqref{eqn:B3Inequality}, while
  letting $C^*_3(F,\psi_0,Re,Ro,\Gamma_4) = \dfrac{\Gamma_4}{2}\, C^*_2(F,\psi_0,Re,Ro)$,
  $C^*_4(Re) = \frac{9}{2} Re$, $C^*_5(Re,\Gamma_3)=\frac{1}{864}\,Re^{-3}$,
  $C^*_6(Re,Ro,\Gamma_2) = Re^{-2} + Ro^{-2}\Gamma_2^2$, and $C^*_7(\Gamma_1) = \dfrac{\Gamma_1^2}{2}$,
  we have
  \begin{equation}
    \begin{split}
      \frac{1}{2} \frac{d}{dt} &\|\nabla \Phi^h\|^2
        + \frac{Re^{-1}}{2} \|\Delta \Phi^h\|^2
        \le C^*_4(Re) \biggl[\|\nabla \eta_t\|_{-1}^2
        + C^*_6(Re, Ro,\Gamma_2) \|\Delta \eta\|^2 \\
      & + C^*_7(\Gamma_1)\, \|\Delta \psi\|^2 \|\Delta \eta\|^2
        + C^*_3(F,\psi_0,Re,Ro,\Gamma_4) \|\Delta \psi^h\|\,
        \|\Delta \eta\|^2\biggr] + C^*_5(Re,\Gamma_3) \|\Delta \psi\|^4 \|\nabla \Phi^h\|^2.
    \end{split}
    \label{eqn:NoEps}
  \end{equation}
  Let $a(t):= 2\,C^*_5(Re,\Gamma_3) \|\Delta \psi\|^4$. If $a(t) \in L^1(0,T)$, then
  \begin{equation}
    A(t) = \int_{0}^{t}\! a(t')\, dt' < \infty.
    \label{eqn:L4Bound}
  \end{equation}
  Multiplying \eqref{eqn:NoEps} by the integrating factor $e^{-A(t)}$
  \begin{align*}
    \biggl\{ \frac{d}{dt}\left[\|\nabla \Phi^h\|^2\right]
      &- 2\, C^*_5(Re,\Gamma_3) \|\Delta \psi\|^4 \|\nabla \Phi^h\|^2\biggr\} e^{-A(t)}
        + Re^{-1} \|\Delta \Phi^h\|^2 e^{-A(t)} \\
      & \le 2\, C^*_4(Re) \biggl[\|\nabla \eta_t\|_{-1}^2
        + C^*_6(Re,Ro,\Gamma_2) \|\Delta \eta\|^2 + C^*_7(\Gamma_1)\,
        \|\Delta \psi\|^2 \|\Delta \eta\|^2 \\
      & \qquad+ C^*_3(F,\psi_0,Re,Ro,\Gamma_4) \|\Delta \psi^h\|\, \|\Delta
        \eta\|^2\biggr] e^{-A(t)},
  \end{align*}
  which can also be written as
  \begin{align*}
    \biggl\{ e^{-A(t)}\frac{d}{dt}
      & \left[\|\nabla \Phi^h\|^2\right]
      - \frac{d}{dt}\bigl[ A(t)\bigr] e^{-A(t)} \|\nabla \Phi^h\|^2\biggr\}
      + Re^{-1} \|\Delta \Phi^h\|^2 e^{-A(t)} \\
    & \le 2\,C^*_4(Re) \biggl[\|\nabla \eta_t\|_{-1}^2
      + C^*_6(Re,Ro,\Gamma_2) \|\Delta \eta\|^2 + C^*_7(\Gamma_1)\,
      \|\Delta \psi\|^2 \|\Delta \eta\|^2 \\
    &\qquad + C^*_3(F,\psi_0,Re,Ro,\Gamma_4) \|\Delta \psi^h\|\, \|\Delta
      \eta\|^2\biggr] e^{-A(t)},
  \end{align*}
  and simplifies to
  \begin{align*}
    \frac{d}{dt}\bigl[e^{-A(t)} &\|\nabla \Phi^h\|^2\bigr]
      + Re^{-1} \|\Delta \Phi^h\|^2 e^{-A(t)} \\
    & \le 2\, C^*_4(Re) \biggl[\|\nabla \eta_t\|_{-1}^2
      + C^*_6(Re,Ro,\Gamma_2) \|\Delta \eta\|^2 + C^*_7(\Gamma_1)\,
      \|\Delta \psi\|^2 \|\Delta \eta\|^2 \\
    &\qquad + C^*_3(F,\psi_0,Re,Ro,\Gamma_4)\, \|\Delta \psi^h\|\,
      \|\Delta \eta\|^2\biggr] e^{-A(t)},
  \end{align*}
  Now, integrating over $[0,T]$ and multiplying by $e^{A(T)}$ gives
  \begin{equation}
    \begin{split}
      \|\nabla \Phi^h(T)\|^2 + Re^{-1} \int_0^T\! &\|\Delta \Phi^h\|^2
        e^{A(T) - A(t)}\, dt \le e^{A(T) - A(0)} \|\nabla \Phi^h(0)\|^2 \\
      & + 2\, C^*_4(Re)\biggl[ \int_0^T\! \|\nabla \eta_t\|_{-1}^2
        + C^*_6(Re,Ro,\Gamma_2) \|\Delta \eta\|^2 e^{A(T) - A(t)}\, dt \\
      & + \int_0^T\!  \left( C^*_7(\Gamma_1)\, \|\Delta \psi\|^2
        +  C^*_3(F,\psi_0,Re,Ro,\Gamma_4)\,\|\Delta \psi^h\|\right)
        \|\Delta \eta\|^2 e^{A(T) - A(t)}\, dt\biggr].
    \end{split}
    \label{eqn:IntegratedInequality}
  \end{equation}
  Noting that $e^{A(T) - A(t)} \ge 1$, $e^{A(T) - A(t)} \le e^{A(T)}$, and
  $A(0) = 0$, \eqref{eqn:IntegratedInequality} becomes
  \begin{equation}
    \begin{split}
      \|\nabla \Phi^h(T)\|^2 + Re^{-1} \int_0^T\! \|\Delta \Phi^h\|^2 &\, dt
        \le C^*_8(T,Re) \|\nabla \Phi^h(0)\|^2 \\
      & + C^*_9(T,Re)\biggl[ \int_0^T\! \|\nabla \eta_t\|_{-1}^2
        + C^*_6(Re,Ro,\Gamma_2) \|\Delta \eta\|^2\, dt \\
      & + \int_0^T\!  \left( C_7(\Gamma_1)\, \|\Delta \psi\|^2
        + C^*_3(F,\psi_0,Re,Ro,\Gamma_4)\,\|\Delta \psi^h\|\right)
        \|\Delta \eta\|^2\, dt\biggr],
    \end{split} \label{eqn:CTREInequality}
  \end{equation}
  where
  \begin{align}
    C^*_8(T,Re,\Gamma_3) &= \exp\!\left(\dfrac{Re^{-3}\Gamma_3^4}{432}
      \int_{0}^{T}\!\|\Delta \psi^4\|\, dt\right), \label{eqn:C1TRe} \\
    C^*_9(T,Re,\Gamma_3) &= \dfrac{9}{2} Re\, \exp\!\left(\dfrac{Re^{-3}\Gamma_3^4}{432}
      \int_{0}^{T}\!\|\Delta \psi^4\|\, dt\right). \label{eqn:C2TRe}
  \end{align}
  By the H\"older inequality we have
  \begin{align}
    \int_0^T\! \|\Delta \psi^h\| \|\Delta \eta\|^2\, dt &\le
      \|\Delta \psi^h\|^2_{L^2(0,T;L^2)} \|\Delta \eta\|^2_{L^4(0,T;L^2)},
    \label{eqn:HolderPsih} \\
    \int_0^T\! \|\Delta \psi\|^2 \|\Delta \eta\|^2\, dt &\le
      \|\Delta \psi^h\|^2_{L^4(0,T;L^2)} \|\Delta \eta\|^2_{L^4(0,T;L^2)},
    \label{eqn:HolderPsi}
  \end{align}
  and note that $\|\Delta \psi^h\|_{L^2(0,T;L^2)}\le C^*_{10}(Re,Ro,F)$ from the
  stability bound (\autoref{prop:Stability}) while $\|\Delta
  \psi\|_{L^4(0,T;L^2)}\le C^*_{11}$ by hypothesis. Thus, \eqref{eqn:CTREInequality}
  can be written as
  \begin{align*}
    \|\nabla \Phi^h\|^2 + Re^{-1} \int_0^T\! \|\Delta \Phi^h\|^2\, dt
      & \le C^*_8(T,Re,\Gamma_3) \|\nabla \Phi^h(0)\|^2
      + C^*_9(T,Re,\Gamma_3)\biggl[ \int_0^T\! \|\nabla \eta_t\|_{-1}^2
      + C^*_6(Re,Ro,\Gamma_2) \|\Delta \eta\|^2\, dt \\
    & +\left(C^*_7(\Gamma_1)\,C^*_{11}
      + C^*_3(Re,Ro,F,\Gamma_4)\,C^*_{10}(Re,Ro,F)\right)
      \|\Delta \eta\|^2_{L^4(0,T;L^2)}\biggr].
  \end{align*}
  \begin{remark}
    We note that the stability bound in \autoref{prop:Stability} does not
    provide an estimate for $\|\Delta \psi^h\|_{L^4(0,T;L^2)}$. This, was the
    reasoning for treating the nonlinear terms $b(\eta;\psi,\Phi^h)$ and
    $b(\psi^h;\eta,\Phi^h)$ in \eqref{eqn:HolderError} differently.
  \end{remark}
  Now adding $\|\nabla \eta(T)\|^2$ and $Re^{-1} \int_0^T\! \|\Delta \eta\|^2\,
  dt$ to both sides and using the triangle inequality gives
  \begin{align*}
    \frac{1}{2} \|\nabla \left( \psi - \psi^h\right)(T) \|^2
    &+ \frac{Re^{-1}}{2} \int_0^T\! \|\Delta \left(\psi - \psi^h\right)\|^2\, dt
      \le C^*_8(T,Re,\Gamma_3) \|\nabla \Phi^h(0)\|^2 \\
    & + C^*_9(T,Re,\Gamma_3) \int_0^T\! \|\nabla \left( \psi - \omega^h\right)_t\|_{-1}^2
      + \left(1 + C^*_6(Re,Ro,\Gamma_2)\right)
      \|\Delta \left(\psi - \omega^h\right)\|^2\, dt \\
    & + \left(C^*_7(\Gamma_1)\, C^*_{11}
      + C^*_3(Re,Ro,,F,\Gamma_4)\,C^*_{10}(Re,Ro,F)\right)
      \|\Delta \left(\psi - \omega^h\right)\|^2_{L^4(0,T;L^2)} \\
    & + \|\nabla \left(\psi - \omega^h\right)(T)\|^2.
  \end{align*}
  Finally, taking $\inf_{\omega^h \in X^h}$ of both sides and letting
  \begin{align*}
    C_1(T,Re,\Gamma_3) &= 2 C^*_8(T,Re,\Gamma_3) \\
    C_2(T,Re,\Gamma_3) &= 2 C^*_9(T,Re,\Gamma_3) \\
    C_3(Re,Ro,\Gamma_2) &= 1 + C^*_6(Re,Ro,\Gamma_2) \\
    C_4(T,Re,Ro,F,\Gamma_1,\Gamma_4,\int_{0}^{T}\!\|\Delta \psi\|^4\, dt) &=
      2\, C^*_9(T,Re)\,\left(C^*_7(\Gamma_1)\, \int_{0}^{T}\!\|\Delta \psi\|^4\,dt
      + C^*_3(Re,Ro,F,\Gamma_4)\, C^*_{10}(Re,Ro,F)\right)
  \end{align*}
  gives
  \begin{align*}
    \|\nabla \left( \psi - \psi^h\right)(T) \|^2
    &+ Re^{-1} \int_0^T\! \|\Delta \left(\psi - \psi^h\right)\|^2\, dt
      \le C_1(T,Re,\Gamma_3){\color{red} \|\nabla \Phi^h(0)\|^2} \\
      & + \inf_{\omega^h \in X^h} \biggl\{C_2(T,Re,\Gamma_3)
      \int_0^T\! \|\nabla \left( \psi - \omega^h\right)_t\|_{-1}^2
      + C_3(Re,Ro,\Gamma_2)\,\|\Delta \left(\psi - \omega^h\right)\|^2\, dt \\
      & + C_4(T,Re,Ro,F,\Gamma_1,\Gamma_2,\Gamma_4,\int_{0}^{T}\!\|\Delta
      \psi\|^4\, dt)\, \|\Delta \left(\psi - \omega^h\right)\|^2_{L^4(0,T;L^2)} \\
    & + 2\, \|\nabla \left(\psi - \omega^h\right)(T)\|^2.
  \end{align*}
  {\color{red} How does this become $\|\nabla\left(\psi_0 - \psi^h(0)\right)\|^2$?}
\end{proof}

\begin{lemma} \label{lma:Interpolation}

\end{lemma}

\begin{thm} \label{thm:SemiInterp}
  Let $X^h$ be the FE space associated with the Argyris element and let $I^h$
  denote the interpolant in the space of $C^1$ piecewise polynomials of order
  five. Suppose the interpolation estimates \autoref{lma:Interpolation}
  in $H^{-1}(\Omega)$ hold and that $\psi, \psi_t \in C^1(\Omega\times [0,T])$.
  Suppose also that the assumption of \autoref{thm:StrongConvergence} hold.
  Then,
\end{thm}
\begin{proof}

\end{proof}
