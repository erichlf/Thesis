In this section we present the convergence of strong solutions, and the error
analysis associated with the semi-discretization. We will rely heavily on the
previous section containing the finite error analysis for the SQGE
(\autoref{sec:SQGEErrors}).

\begin{prop} \label{prop:Stability}
  The solution to \eqref{eqn:SemiDiscretization}, $\psi^h$, is stable and for any $t>0$
  \begin{equation}
    \frac{1}{2}\|\nabla \psi^h(t)\|^2 + \frac{Re^{-1}}{2}\int_{0}^{t}\! \|\Delta
      \psi^h(t')\|^2 \, dt' \le \frac{1}{2} \|\nabla \psi^h_0\|^2
      + \frac{Re\, Ro^{-2}}{2} \int_{0}^{t}\! \|F(t')\|^2_{-1}\, dt'.
    \label{eqn:Stability}
  \end{equation}
\end{prop}
\begin{proof}
  Take $\chi^h = \psi^h$ in \eqref{eqn:SemiDiscretization} and note $b(\psi^h;\psi^h,\psi^h)
  = 0$ while since $(\psi^h_x,\psi^h) = -(\psi^h,\psi^h_x)$ then $(\psi^h_x,
  \psi^h) = 0$. Thus, we have
  \begin{equation*}
    \frac{1}{2} \frac{d}{dt} \|\nabla \psi^h\|^2 + Re^{-1} \|\Delta \psi^h\|^2 =
      Ro^{-1} (F,\psi^h).
  \end{equation*}
  Using H\"older's, Poincar\'e, and Young's inequalities gives
  \begin{equation*}
    \frac{1}{2} \frac{d}{dt} \|\nabla \psi^h\|^2 + Re^{-1} \|\Delta \psi^h\|^2 =
      \frac{Ro^{-2} C_p}{2\epsilon} \|F\|_{-1}^2 + \frac{\epsilon}{2}\|\nabla
      \psi^h\|^2.
  \end{equation*}
  Now taking $\epsilon = Re^{-1}$ results in
  \begin{equation*}
    \frac{1}{2} \frac{d}{dt} \|\nabla \psi^h\|^2 + \frac{Re^{-1}}{2} \|\Delta
      \psi^h\|^2 = \frac{Re\,Ro^{-2} C_p}{2} \|F\|_{-1}^2.
  \end{equation*}
  Assuming $\|\nabla \psi^h\| \in L^1(0,T)$ and integrating over $(0,t)$ gives
  the final result.
\end{proof}

\begin{lemma} \label{lma:BH2Bounds}
  There is a finite constant $\Gamma_1$ such that for all $\psi,\, \varphi,\,
  \chi \in X$,
  \begin{equation}
    b(\psi;\varphi,\chi) \le \Gamma_1 \|\Delta \psi\|\, \|\Delta \varphi\|\,
      \|\Delta \chi\|
    \label{eqn:BH2Bounds}
  \end{equation}
\end{lemma}
\begin{proof}
  By H\"older's inequality
  \begin{equation}
    b(\psi;\varphi,\chi) \le \|\Delta \psi\|_{L^p} \|\nabla \varphi\|_{L^q}
      \|\nabla \chi\|_{L^r},\quad \frac{1}{p}+\frac{1}{q}+\frac{1}{r}=1,
    \label{eqn:HolderB}
  \end{equation}
  where $1\le p,\,q,\,r\le \infty$. Thus, setting $p = 2$, and $q = r = 4$ gives
  \begin{equation*}
    b(\psi;\varphi,\chi) \le \|\Delta \psi\| \|\nabla \varphi\|_{L^4} \|\nabla
      \chi\|_{L^4}.
  \end{equation*}
  Then applying the Ladyzhenskaya inequality results in
  \begin{equation*}
    b(\psi;\varphi,\chi) \le \Gamma \|\Delta \psi\|
      \|\nabla \varphi\|^{\nicefrac{1}{2}} \|\Delta \varphi\|^{\nicefrac{1}{2}}
      \|\nabla \chi\|^{\nicefrac{1}{2}} \|\Delta \chi\|^{\nicefrac{1}{2}}.
  \end{equation*}
  Finally, using the Poincar\'e inequality give
  \begin{equation*}
    b(\psi;\varphi,\chi) \le \Gamma_1 \|\Delta \psi\|\, \|\Delta \varphi\|\,
      \|\Delta \chi\|.
  \end{equation*}
\end{proof}
\begin{lemma} \label{lma:BH1Bound}
  There is a finite constant $C$ such hat for all $\psi,\, \varphi,\, \chi \in
  X$
  \begin{equation}
    b(\psi;\varphi,\chi) \le C \|\Delta \psi\|\, \|\Delta \varphi\|\,
      \|\nabla \chi\|^{\nicefrac{1}{2}} \|\Delta \chi\|^{\nicefrac{1}{2}}
    \label{eqn:BH1Bound}
  \end{equation}
\end{lemma}
\begin{proof}
  The proof is almost identical to \autoref{lma:BH2Bounds} except we do not
  use the Poincar\'e inequality on $\|\nabla \chi\|$ after the Ladyzhenskaya
  inequality.
\end{proof}

\begin{thm} \label{thm:StrongConvergence}
  Let $\psi$ be a strong solution to the QGE, then there exists a \\ $C(T, F,
  \psi_0, \Gamma_1, \Gamma_2, Re, Ro,\int_0^T\! |\Delta u|^4\, dt)$ such that
  \begin{equation}
    \begin{split}
      \sup_{0\le t \le T} \|\nabla (\psi &- \psi^h) \|^2 + Re^{-1}
        \int_{0}^{T}\! \|\Delta (\psi - \psi^h)\|^2 \, dt \le C\biggl\{
          \|\nabla(\psi_0 - \psi^h(0))\|^2 \\
        & + \inf_{\chi^h(t) \in X^h} \biggl[\|\nabla(\psi - \chi^h)_t\|^2_{-1} +
          Re^{-1} \|\Delta(\psi - \chi^h)\|^2\, dt \\
        & \qquad + \|\Delta(\psi - \chi^h)\|_{L^4(0,T;L^2)} + \max_{0 \le t \le T}
          \|\nabla (\psi - \chi^h)\|^2\biggr] \biggr\}
    \end{split}
    \label{eqn:StrongConvergence}
  \end{equation}

\end{thm}
\begin{proof}
  Let $\chi = \chi^h \in X^h$ and subtract \eqref{eqn:SemiDiscretization} from
  \eqref{eqn:QGEWF} then let $e:=\psi - \psi^h$ and we obtain
  \begin{equation*}
    (\nabla e_t, \nabla \chi^h) + \left[b(\psi;\psi,\chi^h) - b(\psi^h;\psi^h,\chi^h)\right]
      + Re^{-1}(\Delta e, \Delta \chi^h) - Ro^{-1} (e_x, \chi^h) = 0\quad
      \forall \chi^h \in X^h \subset X.
  \end{equation*}
  Now adding and subtracting $b(\psi^h;\psi,\chi^h$ gives
  \begin{equation*}
    (\nabla e_t, \nabla \chi^h) + \left[b(e;\psi,\chi^h) + b(\psi^h;e,\chi^h)\right]
      + Re^{-1}(\Delta e, \Delta \chi^h) - Ro^{-1} (e_x, \chi^h) = 0\quad
      \forall \chi^h \in X^h \subset X.
  \end{equation*}
  Take $\omega:[0,T] \to X^h$ arbitrary and decomposing the error $e = \eta -
  \Phi^h$, where $\eta := \psi - \omega$ and $\Phi^h := \psi^h - \omega$,
  results in
  \begin{align*}
    (\nabla \Phi^h_t, \nabla \chi^h) + Re^{-1}(\Delta \Phi^h, \Delta \chi^h)
      & = (\nabla \eta_t, \nabla \chi^h) + Re^{-1}(\Delta \eta, \Delta \chi^h)
      - Ro^{-1} \Gamma_2 \left[(\eta_x, \chi^h) - (\Phi^h_x, \chi^h)\right] \\
    & + \left[ b(\eta;\psi,\chi^h) - b(\Phi^h;\psi,\chi^h)
      + b(\psi^h;\eta,\chi^h) - b(\psi^h;\Phi^h,\chi^h)\right].
  \end{align*}
  Let $\chi^h = \Phi^h$ and noting $b(\psi^h;\Phi^h,\Phi^h) = 0$ and $(\Phi^h_x,
  \Phi^h) = -(\Phi^h,\Phi^h_x) \Rightarrow (\Phi^h_x,\Phi^h) = 0$
  \begin{align*}
    \frac{1}{2} \frac{d}{dt} \|\nabla \Phi^h\|^2 + Re^{-1}\|\Delta \Phi^h\|^2
       = (\nabla \eta_t, \nabla \Phi^h) &+ Re^{-1}(\Delta \eta, \Delta \Phi^h)
      - Ro^{-1} \Gamma_2 (\eta_x, \Phi^h) \\
    & + \left[ b(\eta;\psi,\Phi^h) - b(\Phi^h;\psi,\Phi^h)
      + b(\psi^h;\eta,\Phi^h)\right].
  \end{align*}
  Using H\"older, bounds for $(\psi_x,\chi)$ given in \cite{Foster},
  \autoref{lma:BH2Bounds}, and \autoref{lma:BH1Bound} we have
  \begin{equation}
    \begin{split}
      \frac{1}{2} \frac{d}{dt} \|\nabla \Phi^h\|^2 + Re^{-1}\|\Delta \Phi^h\|^2
        &= {\color{red} C_p} \|\nabla \eta_t\|_{-1} \|\Delta \Phi^h\|
        + Re^{-1}\|\Delta \eta\|\, \|\Delta \Phi^h\|
        - Ro^{-1} \Gamma_2 \|\Delta \eta\|, \|\Delta \Phi^h\| \\
      & + \left[ b(\eta;\psi,\Phi^h) - b(\Phi^h;\psi,\Phi^h)
        + b(\psi^h;\eta,\Phi^h)\right].
    \end{split}
    \label{eqn:HolderError}
  \end{equation}
  Then using Young's inequality we have
  \begin{align}
    \|\nabla \eta_t\|_{-1} \|\Delta \Phi^h\|
      &\le \frac{\epsilon}{2} \|\Delta \Phi^h\|^2
      + \frac{1}{2 \epsilon} \|\eta_t\|_{-1}^2 \label{eqn:YoungT} \\
    Re^{-1} \|\Delta \eta\| \|\Delta \Phi^h\|
      &\le \frac{\epsilon}{2} \|\Delta \Phi^h\|^2
      + \frac{Re^{-2}}{2 \epsilon} \|\Delta \eta\|^2 \label{eqn:YoungLaplace} \\
    Ro^{-1} \Gamma_2 \|\Delta \eta\| \|\Delta \Phi^h\|
      &\le \frac{\epsilon}{2} \|\Delta \Phi^h\|^2
      + \frac{Ro^{-2} \Gamma_2^2}{2 \epsilon} \|\Delta \eta\|^2 \label{eqn:YoungBeta}
  \end{align}
  while using Young's inequality with $\varepsilon = 2 \epsilon$ and
  \autoref{lma:BH2Bounds}
  \begin{equation}
    b(\eta;\psi,\Phi^h)
      \le \epsilon \|\Delta \Phi^h\|^2
      + \frac{\Gamma_1^2}{4 \epsilon} \|\Delta \eta\|^2 \|\Delta \psi\|^2.
      \label{eqn:YoungBH2}
  \end{equation}
  Combing \eqref{eqn:YoungT} - \eqref{eqn:YoungBH2} and \eqref{eqn:HolderError}
  we obtain
  \begin{equation}
    \begin{split}
    \frac{1}{2} \frac{d}{dt} \|\nabla \Phi^h\|^2 + \frac{1}{2}\left(2Re^{-1} -
      5 \epsilon \right)\|\Delta \Phi^h\|^2
      &\le \frac{1}{2 \epsilon}\left[\|\nabla \eta_t\|_{-1}^2
      + \left( Re^{-2} - Ro^{-2} \Gamma_2^2 \right) \|\Delta \eta\|^2\right] \\
     & + \frac{\Gamma_1^2}{4 \epsilon}\|\Delta \eta\|^2 \|\Delta \psi\|^2  -
     \left[b(\Phi^h;\psi,\Phi^h) - b(\psi^h;\eta,\Phi^h)\right].
    \end{split}
    \label{eqn:B1Inequality}
  \end{equation}
  For the term $b(\Phi^h; \psi, \Phi^h)$ we use \autoref{lma:BH1Bound} and the
  following inequality: given $a,\,b>0,$ for any $\epsilon > 0,\, 1\le p \le
  \infty,\, \frac{1}{p} + \frac{1}{q} = 1$,
  \begin{equation*}
    ab \le \epsilon\, a^p + C(\epsilon,p)\, b^q
  \end{equation*}
  with $p=\nicefrac{4}{3},\, q = 4$ to obtain
  \begin{align}
    b(\Phi^h; \psi, \Phi^h) &\le \|\Delta \Phi^h\|^{\nicefrac{3}{2}}\left(C
      \|\Delta \psi\| \|\nabla \Phi^h\|^{\nicefrac{1}{2}}\right) \nonumber \\
    &\le \epsilon \|\Delta \Phi^h\|^2 + C(\epsilon,p) \|\Delta \psi\|^4
      \|\nabla \Phi^h\|^2.
    \label{eqn:EpsYoungH1}
  \end{align}
  Combining \eqref{eqn:B1Inequality} and \eqref{eqn:EpsYoungH1} gives
  \begin{align*}
    \frac{1}{2} \frac{d}{dt} \|\nabla \Phi^h\|^2 + \frac{1}{2}\left(2Re^{-1} -
      7 \epsilon \right)\|\Delta \Phi^h\|^2
      &\le \frac{1}{2 \epsilon}\left[\|\nabla \eta_t\|_{-1}^2
      + \left( Re^{-2} - Ro^{-2} \Gamma_2^2 \right) \|\Delta \eta\|^2\right] \\
    & + \frac{\Gamma_1^2}{2}\|\Delta \eta\|^2 \|\Delta \psi\|^2
      + C(\epsilon, p) \|\Delta \psi\|^4 \|\nabla \Phi^h\|^2
      + b(\psi^h;\eta,\Phi^h).
  \end{align*}
  For the final term involving $b(\psi^h; \eta, \Phi^h)$ we use
  \eqref{eqn:YoungBH2}, which gives
  \begin{align*}
    \frac{1}{2} \frac{d}{dt} \|\nabla \Phi^h\|^2 + \frac{1}{2}\left(2Re^{-1} -
      9 \epsilon \right)\|\Delta \Phi^h\|^2
      &\le \frac{1}{2 \epsilon}\left[\|\nabla \eta_t\|_{-1}^2
      + \left( Re^{-2} - Ro^{-2} \Gamma_2^2 \right) \|\Delta \eta\|^2\right] \\
    & + \frac{\Gamma_1^2}{4 \epsilon}
      \left( \|\Delta \psi\|^2 +  \|\Delta \psi^h\|^2\right) \|\Delta \eta\|^2
      + C(\epsilon, p) \|\Delta \psi\|^4 \|\nabla \Phi^h\|^2.
  \end{align*}
  Take $\epsilon = \nicefrac{Re^{-1}}{9}$ and we have
  \begin{equation}
    \begin{split}
      \frac{1}{2} \frac{d}{dt} \|\nabla \Phi^h\|^2
        + \frac{Re^{-1}}{2} \|\Delta \Phi^h\|^2
        & \le C_1(Re) \biggl[\|\nabla \eta_t\|_{-1}^2
         + C(Ro) \|\Delta \eta\|^2 \\
      & + \frac{\Gamma_1^2}{2}
        \left( \|\Delta \psi\|^2 +  \|\Delta \psi^h\|^2\right)
        \|\Delta \eta\|^2\biggr] + C_2(Re) \|\Delta \psi\|^4 \|\nabla \Phi^h\|^2.
    \end{split}
    \label{eqn:NoEps}
  \end{equation}
  Let $a(t):= C_2(Re) \|\Delta \psi\|^4$. Then if $a(t) \in L^1(0,T)$ then
  \begin{equation}
    A(t) = \int_{0}^{t}\! a(t')\, dt' < \infty \text{ for } \|\Delta \psi\| \in
      L^4(0,T).
    \label{eqn:L4Bound}
  \end{equation}
  Multiplying \eqref{eqn:NoEps} by the integrating factor $e^{-A(t)}$
  \begin{align*}
    \biggl\{ \frac{d}{dt}\left[\|\nabla \Phi^h\|^2\right]
      &- C_2(Re) \|\Delta \psi\|^4 \|\nabla \Phi^h\|^2\biggr\} e^{-A(t)}
       + Re^{-1} \|\Delta \Phi^h\|^2 e^{-A(t)} \\
    & \le C_1(Re) \biggl[\|\nabla \eta_t\|_{-1}^2
       + C(Ro) \|\Delta \eta\|^2 + \frac{\Gamma_1^2}{2}
      \left( \|\Delta \psi\|^2 +  \|\Delta \psi^h\|^2\right)
      \|\Delta \eta\|^2\biggr] e^{-A(t)},
  \end{align*}
  which can also be written as
  \begin{align*}
    \biggl\{ e^{-A(t)}\frac{d}{dt}
      & \left[\|\nabla \Phi^h\|^2\right]
      - \frac{d}{dt}\bigl[ A(t)\bigr] e^{-A(t)} \|\nabla \Phi^h\|^2\biggr\}
      + Re^{-1} \|\Delta \Phi^h\|^2 e^{-A(t)} \\
    & \le C_1(Re) \biggl[\|\nabla \eta_t\|_{-1}^2
      + C(Ro) \|\Delta \eta\|^2 + \frac{\Gamma_1^2}{2}
      \left( \|\Delta \psi\|^2 +  \|\Delta \psi^h\|^2\right)
      \|\Delta \eta\|^2\biggr] e^{-A(t)},
  \end{align*}
  and simplifies to
  \begin{align*}
    \frac{d}{dt}\left[e^{-A(t)} \|\nabla \Phi^h\|^2\right]
      &+ Re^{-1} \|\Delta \Phi^h\|^2 e^{-A(t)} \\
    & \le C_1(Re) \biggl[\|\nabla \eta_t\|_{-1}^2
      + C(Ro) \|\Delta \eta\|^2 + \frac{\Gamma_1^2}{2}
      \left( \|\Delta \psi\|^2 +  \|\Delta \psi^h\|^2\right)
      \|\Delta \eta\|^2\biggr] e^{-A(t)}.
  \end{align*}
  Now, integrating over $(0,t)$ and multiplying by $e^{A(t)}$ gives
  \begin{align*}
    \|\nabla \Phi^h\|^2 + Re^{-1} \int_0^t\! \|\Delta \Phi^h\|^2\, dt'
      & \le C(T,Re) \|\nabla \Phi^h(0)\|^2 + C(T,Re) \int_0^t\! \|\nabla
      \eta_t\|_{-1}^2 + C(Ro) \|\Delta \eta\|^2\, dt' \\
    & + C \int_0^t\!  \left( \|\Delta \psi\|^2 +  \|\Delta \psi^h\|^2\right)
      \|\Delta \eta\|^2\, dt'.
  \end{align*}
  By H\"older's inequality we have
  \begin{equation}
    \int_0^t\! \|\Delta \psi^h\|^2 \|\Delta \eta\|^2\, dt' \le \|\Delta
      \psi^h\|^2_{L^2(0,T;L^2)} \|\Delta \eta\|^2_{L^4(0,T;L^2)},
    \label{eqn:HolderPsih}
  \end{equation}
  and note that $\|\Delta \psi^h\|_{L^2(0,T;L^2)}$ can be bounded by problem
  data from the stability bound.
\end{proof}

\begin{lemma} \label{lma:Interpolation}

\end{lemma}

\begin{thm} \label{thm:SemiInterp}
  Let $X^h$ be the FE space associated with the Argyris element and let $I^h$
  denote the interpolant in the space of $C^1$ piecewise polynomials of order
  five. Suppose the interpolation estimates \autoref{lma:Interpolation}
  in $H^{-1}(\Omega)$ hold and that $\psi, \psi_t \in C^1(\Omega\times [0,T])$.
  Suppose also that the assumption of \autoref{thm:StrongConvergence} hold.
  Then,
  \begin{equation}
    \begin{split}
      \sup_{0\le t \le T} \|\nabla (\psi &- \psi^h) \|^2 + Re^{-1}
        \int_{0}^{T}\! \|\Delta (\psi - \psi^h)\|^2 \, dt \le C\biggl\{
          \|\nabla(\psi_0 - \psi^h(0))\|^2 \\
        & + \inf_{\chi^h(t) \in X^h} \biggl[\|\nabla(\psi - \chi^h)_t\|^2_{-1} +
          Re^{-1} \|\Delta(\psi - \chi^h)\|^2\, dt \\
        & \qquad + \|\Delta(\psi - \chi^h)\|_{L^4(0,T;L^2)} + \max_{0 \le t \le T}
          \|\nabla (\psi - \chi^h)\|^2\biggr] \biggr\}
    \end{split}
    \label{eqn:SemiInterp}
  \end{equation}
\end{thm}
\begin{proof}

\end{proof}
