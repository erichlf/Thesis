In this section, we present the functional setting and some auxiliary results for the FEM
discretization of the streamfunction formulation of the SQGE \eqref{eqn:SQGEWF}. Since the SQGE is
QGE with $\dfrac{\partial [\Delta \psi]}{\partial t} = 0$ the weak formulation is only different in
that the derivative term does not exist. Thus, the functional space is simpler. Indeed,
let $\mathcal{T}^h$ denote a finite element triangulation of $\Omega$ with meshsize (maximum
triangle diameter) $h$. We consider a \emph{conforming} FEM discretization of \eqref{eqn:SQGEWF},
i.e., $X^h \subset X = H_0^2(\Omega)$.

The FEM discretization of the streamfunction formulation of the SQGE \eqref{eqn:SQGEWF} reads:
\begin{equation}
  \begin{split}
    &\text{Find } \psi^h \in X^h \text{ such that} \\
    a_1(\psi^h,\chi^h) + a_2&(\psi^h,\psi^h,\chi^h)
      + a_3(\psi^h,\chi^h) = \ell(\chi^h),\quad \forall \, \chi^h \in X^h.
    \label{eqn:SQGEFEF}
  \end{split}
\end{equation}
Using standard arguments \cite{Girault79,Girault86}, one can prove that, if the small data condition
used in proving the well-posedness result for the continuous case holds, then \eqref{eqn:SQGEFEF}
has a unique solution $\psi^h$ (see Theorem 2.1 in \cite{Cayco86}). Furthermore, one can prove the
following stability result for $\psi^h$ using the same arguments as those used in the proof of
\eqref{thm:stability_sqge} for the continuous setting.
\begin{thm} \label{thm:stability_fem_sqge} The
  solution $\psi^h$ of \eqref{eqn:SQGEFEF} satisfies the following stability estimate:
 \begin{equation}
   |\psi^h|_2 \le Re \, Ro^{-1} \, \| F \|_{-2} .
   \label{eqn:stability_fem_sqge}
 \end{equation}
\end{thm}
\begin{proof}
  The proof is almost identical to the proof of \autoref{thm:stability_sqge}, but is given here for
  completeness.

  Let $\chi^h = \psi^h$ in \eqref{eqn:SQGEFEF} which gives
  \begin{equation*}
    a_1(\psi^h,\psi^h) + a_2(\psi^h,\psi^h,\psi^h) + a_3(\psi^h,\psi^h) = \ell(\psi^h)\qquad  \forall
    \pi^h \in X^h.
  \end{equation*}
  Since, $a_2(\psi^h, \psi^h, \psi^h) =0$ and $a_3(\psi^h,\psi^h)=0$ we have
  \begin{align*}
    a_1(\psi^h,\psi^h) &= \ell(\psi^h) \\
    Re^{-1}\, \|\psi^h\|_2^2 &= Ro^{-1}\, (F,\psi^h) \\
    \|\psi^h\|_2 &\le Re\, Ro^{-1}\,\sup_{\psi^h \in X^h} \frac{(F,\psi^h)}{|\psi^h|_2} \\
    \|\psi^h\|_2 &\le Re\, Ro^{-1}\, \|F\|_{-2}.
  \end{align*}
  And the proof is complete.
\end{proof}

Again, we point out that as noted in Section 6.1 in \cite{Ciarlet} (see also Section 13.2 in
\cite{Gunzburger89}, Section 3.1 in \cite{Johnson}, and Theorem 5.2 in \cite{Braess}, in order to
develop a conforming FEM for the SQGE \eqref{eqn:SQGEWF}, we are faced with the problem of
constructing subspaces of the space $H^2_0(\Omega)$. Since the standard, piecewise polynomial FEM
spaces are locally regular, this construction amounts in practice to finding FEM spaces $X^h$ that
satisfy the inclusion $X^h \subset C^1({\bar \Omega})$, i.e., finding $C^1$ finite elements. As was
discussed previously the Argyris finite element is an element of class $C^1$ and therefore will be
the element of interest for the FE discretization of the SQGE.

