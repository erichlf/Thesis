With the continuous increase in computational power, complex mathematical models are becoming more
and more popular in the numerical simulation of oceanic and atmospheric flows. For some geophysical
flows in which computational efficiency is of paramount importance, however, simplified mathematical
models are central. For example, the \emph{quasi-geostrophic equations (QGE)}, a standard
mathematical model for large scale oceanic and atmospheric flows
\cite{Cushman11,Majda,Pedlosky92,Vallis06}, are often used in climate modeling \cite{Dijkstra05}.

The QGE are usually discretized in space by using the \emph{finite difference method} (FDM)
\cite{San11}. The \emph{finite element method} (FEM), however, offers several advantages over the
popular FDM, as outlined in \cite{Myers}: (i) an easy treatment of complex boundaries, such as those
of continents for the ocean, or mountains for the atmosphere; (ii) an easy grid refinement to
achieve a high resolution in regions of interest \cite{Cascon}; (iii) a natural treatment of
boundary conditions; and (iv) a straightforward approach for the treatment of multiply connected
domains \cite{Myers}. Despite these advantages, there are relatively few papers that consider the
FEM applied to the QGE \cite{Cascon, Fix, LeProvost94, Myers, Stevens82}.

To our knowledge, \emph{all} the FEM discretizations of the QGE have been developed for the
streamfunction-vorticity formulation, none using the streamfunction formulation. The reason is
simple: The streamfunction-vorticity formulation yields a second order \emph{partial differential
equation (PDE)}, whereas the streamfunction formulation yields a fourth order PDE. Thus, although
the streamfunction-vorticity formulation has two variables ($q$ and $\psi$) and the streamfunction
formulation has just one ($\psi$), the former is the preferred formulation used in practical
computations, since its conforming FEM discretization requires low-order ($C^0$) elements, whereas
the latter requires high-order ($C^1$) elements.

The streamfunction formulation is, from both mathematical and computational points of view,
completely different from the vorticity-streamfunction formulation. Indeed, the FEM discretization
of the streamfunction formulation generally requires the use of $C^1$ elements (for a conforming
discretization), which makes their implementation challenging. From a mathematical point of view,
however, the streamfunction formulation has the following significant advantage over the
vorticity-streamfunction formulation: there are optimal error estimates for the FEM discretization
of the streamfunction formulation (see the error estimate (13.5) and Table 13.1 in
\cite{Gunzburger89}), whereas the available error estimates for the vorticity-streamfunction
formulation are suboptimal.

Despite the simplification made in formulating the QGE from the full-fledged equations of
ocean, the numerical simulation of the QGE is still computationally challenging when integrating
over long time periods, as is the case in climate modeling. Therefore, it is necessary to reduce the
computational cost of QGE simulations. We will consider two (not necessarily independent) approaches
to reducing this computational cost; The first, by using a \emph{Two-Level} method to calculate the
streamfunction at each time step, and the second approach will be to apply \emph{Large Eddy
Simulation (LES)}, namely \emph{Approximate Deconvolution} (AD), to the QGE. To our knowledge it
will be the first time either of these approaches have been applied to the streamfunction formulation
of the QGE.

Two-level finite element discretizations are very promising approaches for finite element
discretizations of nonlinear partial differential equations \cite{Fairag98,Layton93}. A Two-level
finite element discretizations aims to solve a particular nonlinear elliptic equation by first
solving the system on a coarse mesh and then using the coarse mesh solution as a linearized variable
to solve the linearized elliptic equation on  a fine mesh. The attraction of such a method is that
one need only solve the nonlinear equations on a coarse mesh and then use this solution to solve on
a fine mesh, thereby reducing computational time without sacrificing much in the way of solution
accuracy. The development of the two-level finite element discretization was originally by Xu in
\cite{Xu94} and later algorithms for the Navier-Stokes Equations (NSE) were developed by Layton \cite{Layton93}, Fairag
\cite{Fairag98, Fairag03}, and Shao \cite{Shao11}.

LES was developed mainly by the engineering and geophysics communities and has emerged as one of the
most promising approaches for modeling turbulent flows \cite{Iliescu00}. In LES only the large
spatial structures are computed directly while the interaction with small scales are modeled,
allowing for a coarser spatial mesh. This coarser mesh leads to lower computational cost as compared
to the \emph{direct numerical simulation} (DNS). However, when using LES one must address the
\emph{closure problem}.  This closure problem is essentially the relationship between the energy
cascade from small scales to large scales, in two dimensional flows, or from large scales to small
scales, in three dimensional flows. Most of the development for LES models have been for 3D
engineering flows \cite{Berselli06}.  Due to the difference in the energy cascade for 3D (forward
energy cascade), as compared to 2D flows (inverse energy cascade), these LES closure models aren't
appropriate for geophysical flows \cite{San11}. 

The goals of this paper are five-fold. First, we use a $C^1$ finite element (the Argyris element)
to discretize the streamfunction formulation of the QGE. To the best of our knowledge, this is the
\emph{first} time that a $C^1$ finite element has been used in the numerical discretization of the
QGE. Second, we derive optimal error estimates for the FEM discretization of the QGE and present
supporting numerical experiments. To the best of our knowledge, this is the \emph{first} time that
optimal error estimates for the QGE have been derived. Third, we present a Two-Level algorithm for
solving the streamfunction formulation of the QGE and present the error analysis associated with
this algorithm. To the best of our knowledge, this is the \emph{first} time that such an algorithm
has been applied the the streamfunction formulation of the QGE. Fourth, we will present an LES
closure model for the streamfunction formulation of the QGE and its theoretical framework. To the
best of our knowledge, this is the \emph{first} time that LES has been applied to the streamfunction
formulation of the QGE. Finally, we will present a finite element error analysis associated with the
LES closure model applied to the streamfunction formulation of the QGE. To the best of our
knowledge, this is the \emph{first} time that such an error analysis has been derived for LES
applied to the streamfunction formulation of the QGE.

