In this section we discuss Large Eddy Simulation (LES). To our knowledge LES has not been applied to
the finite element formulation of the streamfunction formulation of the QGE. It is the application
of LES to this problem that will be the major contribution of this Thesis and therefore will be the
main focus of the thesis. 

To derive the quations for the filtered flow variable, $\bar{\psi}$, the QGE \eqref{qge_psi_3} has a
spacial filter applied to it. Letting a bar denote the filtered variable, the filtered QGE reads:
\begin{equation}
  -\frac{\partial [\Delta \bar{\psi}]}{\partial t} + Re^{-1} \Delta^2 \bar{\psi} +
    J(\bar{\psi},\Delta \bar{\psi}) - Ro^{-1} \frac{\partial \bar{\psi}}{\partial x} = Ro^{-1}
    \bar{F} + S
  \label{eqn:FQGE}
\end{equation}
where $S$ is the \emph{subfilter-scale} term, given by
\begin{equation}
  S = - \bar{J(\psi, \Delta \psi)} + J(\bar{\psi}, \Delta \bar{\psi}). 
  \label{eqn:Subfilter}
\end{equation}
It is precisely this point where the closure model comes in, since the first term in the subfilter
scale still requires the unfiltered variable. For the filtered QGE \eqref{eqn:FQGE} to be closed the
subfilter scale must be modeled in terms of the filtered variable $\bar{\psi}$, therefore closure
models are necessary so as to close the filtered equation. 

We propose to use \emph{Approximate Deconvolution (AD)} for the closure model. The AD method uses
repeated filtering in order to obtain an approximation to the unfiltered variable when the filtered
variable is available. These approximations are used in the subfilter scale to close the model. 

If we denote our spatial filtering operator by $G$ then $G\psi = \bar{\psi}$. Using the fact that $G
= I - (I - G)$ we can write the inverse filter $G^{-1}$ using the Neumann series:
\begin{equation}
  G^{-1} \approx \sum_{i=0}^{\infty} (I - G)^i.
  \label{eqn:NeumannInverse}
\end{equation}
For practical use we must truncate this series, thus truncating the series after $N$ terms gives the
approximate deconvolution operator, $Q_N$. Therefore the approximate deconvolution of $\psi$ is
obtained by 
\begin{equation}
  \psi \approx \psi^* = Q_N \bar{\psi}.
  \label{eqn:ADOperator}
\end{equation}
By increasing the value of $N$ we get a more accurate approximation to the variable $\psi$. Using
this idea of approximate convolution we can now approximate the Jacobian:
\begin{equation}
  \bar{J(\psi,\Delta \psi)} \approx \bar{J(\psi^*, \Delta \psi^*)}.
  \label{eqn:ADJac}
\end{equation}
Thus, we can close the filtered QGE \eqref{eqn:FQGE} and we get the AD model
\begin{equation}
  -\frac{\partial [\Delta \bar{\psi}]}{\partial t} + Re^{-1} \Delta^2 \bar{\psi} 
    + J(\bar{\psi},\Delta \bar{\psi}) - Ro^{-1} \frac{\partial \bar{\psi}}{\partial x} = Ro^{-1}
    \bar{F} + S^*
  \label{eqn:ADQGE}
\end{equation}
where $S^*$ is the subfilter scale modeled by
\begin{equation}
  S^* = -\bar{J(\psi^*, \Delta \psi^*)} + J(\bar{\psi}, \Delta \bar{\psi}). 
  \label{eqn:ApproxSubfilter}
\end{equation}

In the rest of this section we will discuss the AD model theory at length, since it is of great
importance to model results. 

