We are now ready to derive the weak formulation of the SQGE in streamfunction
formulation \eqref{eqn:SQGE_Psi}. To this end, we first introduce the
appropriate functional setting. Let
\begin{equation*}
  X := H^2_0(\Omega) = \left\{ \psi\in H^2(\Omega): \psi=\frac{\partial\psi}{\partial {\bf n}}=0
    \text{ on } \partial\Omega \right\} .
\end{equation*}
The difference in the functional setting for the QGE and the SQGE stems from the
lack of time derivative in the SQGE. However, in the spatial domain everything
remains the same.

Thus, the derivation of the weak formulation for the SQGE follows immediately
from the strong formulation for the QGE with the exception that we set the
bilinear form $(\nabla \psi, \nabla \chi)$ to zero. Therefore, using the
notation presented in \autoref{sec:QGEStrong}, the weak form of the SQGE in streamfunction
formulation is:
\begin{equation}
  \begin{split}
    \text{Find }\psi &\in X \text{ such that} \\
    Re^{-1} (\Delta \psi, \Delta \chi) + b(\psi,\psi,&\chi) - Ro^{-1} (\psi_x,\chi) =
      Ro^{-1} (F,\chi),\quad \forall \chi \in X.  \end{split} \label{eqn:SQGEWF}
\end{equation}
The linear form $(F, \chi)$, the bilinear forms $(\Delta \psi, \Delta \chi)$,
$(\psi_x, \chi)$ and the trilinear form $b(\psi;\psi,\chi)$ are still continuous
\cite{Cayco86} and have the same bounds as those for the QGE, i.e.
\eqref{eqn:a1Cont}, \eqref{eqn:BH2Bounds}, \eqref{eqn:a3Cont}, and
\eqref{eqn:lCont}.

