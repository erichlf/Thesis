We are now ready to derive the weak formulation of the SQGE in streamfunction
formulation \eqref{eqn:SQGE_Psi}. To this end, we first introduce the
appropriate functional setting. Let
\begin{equation*}
  X := H^2_0(\Omega) = \left\{ \psi\in H^2(\Omega): \psi=\frac{\partial\psi}{\partial {\bf n}}=0
    \text{ on } \partial\Omega \right\} .
\end{equation*}
The difference in the functional setting for the QGE and the SQGE stems from the
lack of time derivative in the SQGE. However, in the spacial domain everything
remains the same.

Thus, the derivation of the weak formulation for the SQGE follows imediately
from the weak formulation for the QGE with the exception that we set the
bilinear form \eqref{eqn:a0} to zero. Therefore, using the forms presented for
the QGE weak form, i.e.  \eqref{eqn:a1}, \eqref{eqn:a2}, \eqref{eqn:a3}, and
\eqref{eqn:l}, the weak form of the SQGE in streamfunction formulation is:
\begin{equation}
  \begin{split}
    \text{Find }\psi &\in X \text{ such that} \\
    a_1(\psi,\chi) + a_2(\psi,\psi,&\chi) + a_3(\psi,\chi)
    = \ell(\chi),\quad \forall \chi \in X.
  \end{split}
  \label{eqn:SQGEWF}
\end{equation}
The linear form $\ell$, the bilinear forms $a_1$, $a_3$ and the trilinear form
$a_2$ are still continuous \cite{Cayco86} and have the some bounds as those for
the QGE, i.e.  \eqref{eqn:a1Cont}, \eqref{eqn:a2Cont}, \eqref{eqn:a3Cont}, and
\eqref{eqn:lCont}.


%In addition to the bounds on the
%forms we also have the following theorem on stability bounds
%\begin{theorem} \label{thm:stability_sqge}
%  The solution $\psi$ of \eqref{eqn:SQGE_Psi} satisifies the following stability estimate:
%  \begin{equation}
%    \|\psi\|_2 \le Re\, Ro^{-1}\, \|F\|_{-2}.
%    \label{eqn:stability_sqge}
%  \end{equation}
%\end{theorem}
%\begin{proof}
%  Let $\chi = \psi$ in \eqref{eqn:SQGEWF} which gives
%  \begin{equation*}
%    a_1(\psi,\psi) + a_2(\psi,\psi,\psi) + a_3(\psi,\psi) = \ell(\psi)\qquad \forall \psi \in X.
%  \end{equation*}
%  Since, $a_2(\psi, \psi, \psi) =0$ and $a_3(\psi,\psi)=0$ we have
%  \begin{align*}
%    a_1(\psi,\psi) &= \ell(\psi) \\
%    Re^{-1}\, \|\psi\|_2^2 &= Ro^{-1}\, (F,\psi) \\
%    \|\psi\|_2 &\le Re\, Ro^{-1}\,\sup_{\psi \in X} \frac{(F,\psi)}{|\psi|_2} \\
%    \|\psi\|_2 &\le Re\, Ro^{-1}\, \|F\|_{-2}.
%  \end{align*}
%  Thus, we have completed the proof.
%\end{proof}
