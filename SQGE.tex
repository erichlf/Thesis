When testing Finite Element code it is useful to simplify the problem and so
even though in the ``real'' world stationary ``flows'' don't exist the
stationary QGE (SQGE) are useful in testing code.  Additionally, the time
dependence of the QGE (\eqref{qge_psi_1}) adds aditional complexity to the
finite element error analysis. Finite error analysis for time dependent problems
usualy can be split into two parts; analysis of the spatial discretization that
arizes through the application of finite elements, and the discretization error
in time that arizes from the application of the method of lines. Thus, a FE
error analysis of the SQGE is a good push off point for analysis of the
time-dependent QGE and is therefore the main motivation for presenting the SQGE.

The SQGE are obtained by setting $\dfrac{\partial q}{\partial t}$ to $0$ in
\eqref{qge_q_psi_1} and therefore we get the \emph{streamfunction formulation}
of the \emph{one-layer stationary quasi-geostrophic equations}
\begin{eqnarray}
  Re^{-1} \, \Delta^2 \psi + J(\psi , \Delta \psi) - Ro^{-1} \, \frac{\partial
    \psi}{\partial x} = Ro^{-1} \, F .
  \label{eqn:SQGE_Psi}
\end{eqnarray}
Equations \eqref{qge_q_psi_1}-\eqref{qge_q_psi_2}, \eqref{qge_psi_1}, and
\eqref{eqn:SQGE_Psi} are the usual formulations of the one-layer QGE in
streamfunction-vorticity formulation, the one-layer QGE in streamfunction
formulation, and the steady one-layer QGE in streamfunction formulation,
respectively.

To completely specify the equations in \eqref{eqn:SQGE_Psi}, we need to impose
boundary conditions. For consistency and with the QGE (\eqref{qge_psi_1})  we
consider
\begin{equation*}
  \psi = \frac{\partial \psi}{\partial \mathbf{n}} = 0 \qquad \text{on } \partial \Omega,
\end{equation*}
which, as stated previously, are also the boundary conditions used in
\cite{Gunzburger89} for the streamfunction formulation of the 2D NSE.

