By using Theorem 6.1.1 and inequality (6.1.5) in \cite{Ciarlet}, i.e.
\begin{equation}
  \|\psi - I^h \psi\|_{m,q} \le C h^{k+1-m} |\psi|_{k+1,p},
  \label{eqn:ArgyrisBound}
\end{equation}
we obtain the following two approximation properties for the Argyris FE space
$X^h$:
\begin{align}
  \forall \, \chi \in H^6(\Omega) \cap H^2_0(\Omega), \ \exists \, \chi^h \in X^h
  \quad \text{such that} \quad
  \| \chi - \chi^h \|_2
  &\leq C \, h^4 \, | \chi |_6 ,
  \label{eqn:argyris_approximation_0} \\[0.2cm]
  \forall \, \chi \in H^4(\Omega) \cap H^2_0(\Omega), \ \exists \, \chi^h \in X^h
  \quad \text{such that} \quad
  \| \chi - \chi^h \|_2
  &\leq C \, h^2 \, | \chi |_4 ,
  \label{eqn:argyris_approximation_1} \\[0.2cm]
  \forall \, \chi \in H^3(\Omega) \cap H^2_0(\Omega), \ \exists \, \chi^h \in X^h
  \quad \text{such that} \quad
  \| \chi - \chi^h \|_2
  &\leq C \, h \, | \chi |_3 ,
  \label{eqn:argyris_approximation_2}
\end{align}
where $C$ is a generic constant that can depend on the data, but not on the
mesh size $h$.  Approximation property \eqref{eqn:argyris_approximation_0}
follows from inequality (6.1.5) in \cite{Ciarlet} with $q = 2, \, p = 2, \, m =
2$ and $k+1 = 6$. Approximation property \eqref{eqn:argyris_approximation_1}
follows from inequality (6.1.5) in \cite{Ciarlet} with $q = 2, \, p = 2, \, m =
2$ and $k+1 = 4$. Approximation property \eqref{eqn:argyris_approximation_2}
follows from inequality (6.1.5) in \cite{Ciarlet} with $q = 2, \, p = 2, \, m =
2$ and $k+1 = 3$.

