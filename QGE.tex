The QGE are usually written as follows (see, e.g., equation (14.57) in
\cite{Vallis06}, equation (1.1) in \cite{Majda}, equation (1.1) in
\cite{Wang94}, and equation (1) in \cite{Greatbatch00}):
\begin{align}
  \frac{\partial q}{\partial t} + J(\psi , q) &= A \, \Delta q + F
    \label{qge_q_psi_dim_1} \\
  q &= \Delta \psi + \beta \, y , \label{qge_q_psi_dim_2}
\end{align}
where $q$ is the potential vorticity, $\psi$ is the velocity streamfunction,
$\beta$ is the coefficient multiplying the $y$ coordinate (which is oriented
northward) in the $\beta$-plane approximation \eqref{eqn:beta_plane}, $F$ is the
forcing, $A$ is the eddy viscosity parametrization, and $J(\cdot , \cdot)$ is
the Jacobian operator given by
\begin{align}
  J(\psi , q) := \frac{\partial \psi}{\partial x} \, \frac{\partial q}{\partial y} -
    \frac{\partial \psi}{\partial y} \, \frac{\partial q}{\partial x} . \label{eqn:jacobian}
\end{align}
The $\beta$-plane approximation reads
\begin{equation}
  f = f_0 + \beta \, y , \label{eqn:beta_plane}
\end{equation}
where $f$ is the Coriolis parameter and $f_0$ is the reference Coriolis
parameter (see, e.g., the discussion on page 84 in \cite{Cushman94} or Section
2.3.2 in \cite{Vallis06}).

As noted in Chapter 10.7.2 in \cite{Vallis06} (see also \cite{San12}), the eddy
viscosity parameter $A$ in \eqref{qge_q_psi_dim_1} is usually several orders of
magnitude higher than the molecular viscosity. This choice allows the use of a
coarse mesh in numerical simulations. The horizontal velocity $\mathbf{u}$ can
be recovered from $\psi$ and $q$ by using the following formula:
\begin{align}
  \mathbf{u} := \nabla^{\perp} \psi =
    \begin{pmatrix} - \frac{\partial \psi}{\partial y} \\
    \frac{\partial \psi}{\partial x}
  \end{pmatrix} .
\label{eqn:u_psi}
\end{align}

The computational domain considered in this section (and in most of this thesis)
is standard \cite{Greatbatch00}, a rectangular, closed basin on a $\beta$-plane
with the $y$ coordinate increasing northward and the $x$ coordinate eastward.
The center of the basin is at $y=0$, the northern and southern boundaries are at
$y = \pm \, L$, respectively, and the western and eastern boundaries are at $x =
0$ and $x = L$ (see Figure 1 in \cite{Greatbatch00}).

We are now ready to nondimensionalize the QGE
\eqref{qge_q_psi_dim_1}-\eqref{qge_q_psi_dim_2}.  There are several ways of
nondimensionalizing the QGE, based on different scalings and involving different
parameters (see standard textbooks on geophysical fluid dynamics, such as
\cite{Cushman11,Majda,Pedlosky92,Vallis06}).  Since the FE error analysis in
this report is based on a precise relationship among the nondimensional
parameters of the QGE (see, e.g., \eqref{eqn:small_data_condition}), we present
below a careful nondimensionalization of the QGE.

We first need to choose a length scale and a velocity scale. The length scale we
choose is $L$, the width of the computational domain. The velocity scale is the
Sverdrup velocity
\begin{align}
  U := \frac{\pi \, \tau_0}{\rho \, H \, \beta \, L} ,
  \label{eqn:velocity_scale}
\end{align}
where $\tau_0$ is the amplitude of the wind stress, $\rho$ is the density of the
fluid, and $H$ is the height of the fluid. The Sverdrup balance, which was used
in the derivation of \eqref{eqn:velocity_scale}, expresses the balance between
the two dominant effects in the system: the $\beta$-effect and the curl of the
divergence of the wind stress (see, e.g., Section 14.1.3 in \cite{Vallis06}).
Once the length and velocity scales are chosen, the variables in the QGE
\eqref{qge_q_psi_dim_1}-\eqref{qge_q_psi_dim_2} can be nondimensionalized as
follows:
\begin{align}
  x^* = \frac{x}{L}, \quad
  y^* = \frac{y}{L}, \quad
  t^* = \frac{t}{L / U}, \quad
  q^* = \frac{q}{\beta \, L}, \quad
  \psi^* = \frac{\psi}{U \, L} ,
\label{eqn:nondimensional_variables}
\end{align}
where a $^*$ superscript denotes the nondimensional variable. We first
nondimensionalize \eqref{qge_q_psi_dim_2}. Using
\eqref{eqn:nondimensional_variables}, \eqref{qge_q_psi_dim_2} becomes
\begin{align}
  \beta \, L \, q^* = \frac{1}{L^2} \, \Delta^* (U \, L \, \psi^*) + \beta \, (L \, y^*) .
  \label{qge_q_psi_nondim_1}
\end{align}
Dividing \eqref{qge_q_psi_nondim_1} by $\beta \, L$, we get:
\begin{align}
  q^* = \left( \frac{U}{\beta \, L^2} \right) \, \Delta^* \psi^* + y^* .
  \label{qge_q_psi_nondim_2}
\end{align}
Defining the \emph{Rossby number $Ro$} as follows
\begin{align}
  Ro := \frac{U}{\beta \, L^2} , \label{eqn:rossby_number}
\end{align}
equation \eqref{qge_q_psi_nondim_2} becomes
\begin{align}
  q^* = Ro \, \Delta^* \psi^* + y^* .
  \label{qge_q_psi_nondim_3}
\end{align}

We note that all the nondimensionalizations in
\eqref{eqn:nondimensional_variables} are naturally based on the velocity scale
$U$ and the length scale $L$, except $q^*$. Indeed, a nondimensionalization of
the form
\begin{align}
  {\tilde q} = \frac{U}{L},
  \label{eqn:qge_q_psi_nondim_4}
\end{align}
would probably be more natural. Note that the alternative nondimensionalization
in \eqref{eqn:qge_q_psi_nondim_4} is indeed correct, i.e., the variable ${\tilde
q}$ is nondimensional.  The main reason for which the nondimensionalization in
\eqref{eqn:nondimensional_variables} is used instead the one in
\eqref{eqn:qge_q_psi_nondim_4} is that the former yields just one constant (the
Rossby number $Ro$) in \eqref{qge_q_psi_nondim_3}, whereas the latter would
yield two constants.

Next, we nondimensionalize \eqref{qge_q_psi_dim_1}. We start with the left-hand
side:
\begin{align}
  \frac{\partial q}{\partial t} &= ( \beta \, U) \, \frac{\partial q^*}{\partial t^*} ,
    \label{eqn:qge_q_psi_nondim_5} \\[0.2cm]
  J(\psi,q) &=  \frac{\partial \psi}{\partial x} \, \frac{\partial q}{\partial y} - \frac{\partial
    \psi}{\partial y} \, \frac{\partial q}{\partial x}
    = U \, \frac{\partial \psi^*}{\partial x^*} \, \beta \, \frac{\partial q^*}{\partial y^*} - U \,
      \frac{\partial \psi^*}{\partial y^*} \,  \beta \, \frac{\partial q^*}{\partial x^*}
      \nonumber \\[0.2cm]
  &= (  \beta \, U ) \, J^*(\psi^*,q^*) .
  \label{eqn:qge_q_psi_nondim_6}
\end{align}

Next, we nondimensionalize the right-hand side of \eqref{qge_q_psi_dim_1}. The
first term can be nondimensionalized as follows:
\begin{align}
  \frac{\partial q}{\partial t} A \, \Delta q &= A \, \left( \frac{\partial^2 q}{\partial x^2} +
    \frac{\partial^2 q}{\partial y^2} \right)
%\nonumber \\[0.2cm]
%&=
    = A \,  \left( \frac{1}{L^2} \, \frac{\partial^2}{\partial {x^*}^2} \, (\beta \, L \, q^*) +
      \frac{1}{L^2} \, \frac{\partial^2}{\partial {y^*}^2} \, (\beta \, L \, q^*) \right) \nonumber
      \\[0.2cm]
  &= A \, \frac{\beta}{L} \, \Delta^* q^* .
  \label{eqn:qge_q_psi_nondim_7}
\end{align}
Thus, inserting \eqref{eqn:qge_q_psi_nondim_5}-\eqref{eqn:qge_q_psi_nondim_7} in
\eqref{qge_q_psi_dim_1}, we get:
\begin{align}
  ( \beta \, U) \, \frac{\partial q^*}{\partial t^*} + (  \beta \, U ) \, J^*(\psi^*,q^*)
  &= A \, \frac{\beta}{L} \, \Delta^* q^* + F .
  \label{eqn:qge_q_psi_nondim_8}
\end{align}
Dividing by $\beta \, U$, we get:
\begin{align}
  \frac{\partial q^*}{\partial t^*} + J^*(\psi^*,q^*) &= \left( \frac{A}{U \, L} \right) \,
    \Delta^* q^* + \frac{F}{\beta \, U} .
\label{eqn:qge_q_psi_nondim_9}
\end{align}
Defining the \emph{Reynolds number $Re$} as follows
\begin{align}
  Re := \frac{U \, L}{A} ,
  \label{eqn:reynolds_number}
\end{align}
equation \eqref{eqn:qge_q_psi_nondim_9} becomes
\begin{align}
  \frac{\partial q^*}{\partial t^*} + J^*(\psi^*,q^*) &= Re^{-1} \, \Delta^* q^*
    + \frac{F}{\beta\, U} .
  \label{eqn:qge_q_psi_nondim_10}
\end{align}
As noted above, the Sverdrup balance expresses the balance between the two
dominant effects in the system: the $\beta$-effect and the curl of the
divergence of the wind stress (see, e.g., Section 14.1.3 in \cite{Vallis06}).
The velocity scale $U$ in \eqref{eqn:velocity_scale} was chosen according to the
Sverdrup balance.  Thus, the last term on the right-hand side of
\eqref{eqn:qge_q_psi_nondim_10} has the following units:
\begin{align}
\left[ \frac{F}{\beta \, U} \right]
\sim \left[ \frac{\frac{\nabla \times (\nabla \cdot \tau_0)}{\rho}}{\beta \, U}  \right]
\stackrel{\eqref{eqn:velocity_scale}}{\sim}
\left[ \frac{\frac{\nabla \times (\nabla \cdot \tau_0)}{\rho}}{\frac{\pi \, \tau_0}{\rho \, H \, L}}  \right] ,
\label{eqn:qge_q_psi_nondim_11}
\end{align}
which, after the obvious simplification, is clearly nondimensional.  Thus,
\eqref{eqn:qge_q_psi_nondim_11} clearly shows that the last term on the
right-hand side of \eqref{eqn:qge_q_psi_nondim_10} is nondimensional, so
\eqref{eqn:qge_q_psi_nondim_10} becomes:
\begin{align}
  \frac{\partial q^*}{\partial t^*} + J^*(\psi^*,q^*)
  &= Re^{-1} \, \Delta^* q^* + F^* .
\label{eqn:qge_q_psi_nondim_12}
\end{align}

Dropping the $^*$ superscript in \eqref{eqn:qge_q_psi_nondim_12}, we obtain the
nondimensional {\it vorticity-streamfunction formulation} of the \emph{one-layer
quasi-geostrophic equations}
\begin{align}
  \frac{\partial q}{\partial t} + J(\psi , q) &= Re^{-1} \, \Delta q + F \label{qge_q_psi_1} \\
  q &= Ro \, \Delta \psi + y, \label{qge_q_psi_2}
\end{align}
where $Re$ and $Ro$ are the Reynolds and Rossby numbers, respectively.

Substituting \eqref{qge_q_psi_2} in \eqref{qge_q_psi_1} and dividing by $Ro$, we
get the {\it streamfunction formulation} of the \emph{one-layer quasi-geostrophic
equations}
\begin{align}
  \frac{\partial \left[ \Delta \psi \right]}{\partial t} - Re^{-1} \, \Delta^2 \psi + J(\psi
    , \Delta \psi) + Ro^{-1} \, \frac{\partial \psi}{\partial x} = Ro^{-1} \, F. \label{qge_psi_1}
\end{align}
Equations \eqref{qge_q_psi_1}-\eqref{qge_q_psi_2} and \eqref{qge_psi_1} are the
usual formulations of the one-layer QGE in streamfunction-vorticity and
streamfunction formulations, respectively. We note that the
streamfunction-vorticity formulation has two unknowns ($q$ and $\psi$), whereas
the streamfunction formulation has only one unknown ($\psi$). The
streamfunction-vorticity formulation, however, is more popular than the
streamfunction formulation, since the former is a second-order PDE, whereas the
latter is a fourth-order PDE.

We also note that \eqref{qge_q_psi_2}-\eqref{qge_q_psi_1} and \eqref{qge_psi_1}
are similar in form to the 2D NSE written in the streamfunction-vorticity and
streamfunction formulations, respectively.  Indeed,
\eqref{qge_q_psi_1}-\eqref{qge_q_psi_2} and \eqref{qge_psi_1} are almost the
same as (11.3)-(11.4) and (13.1) in \cite{Gunzburger89}, which are obtained by
first writing
\begin{align}
  \mathbf{u} = \begin{pmatrix} \frac{\partial \psi}{\partial y} \\[0.2cm]
      - \frac{\partial \psi}{\partial x}
    \end{pmatrix} \label{qge_psi_2}
\end{align}
and then taking the curl of the 2D NSE.

There are, however, several significant differences between the QGE and the 2D
NSE. First, we note that the term $y$ in \eqref{qge_q_psi_2} and the
corresponding term $\dfrac{\partial \psi}{\partial x}$ in \eqref{qge_psi_1},
which model the \emph{rotation effects} in the QGE, do not have counterparts in
the 2D NSE.  Furthermore, the Rossby number, $Ro$, in the QGE, which is a
measure of the rotation effects, does not appear in the 2D NSE.  However, apart
from these two significant differences, the streamfunction-vorticity and
streamfunction formulations of the QGE and the 2D NSE in the streamfunction form
are quite similar.

Thus, for notation consistency (i.e., to ensure that the velocity and the
streamfunction are related by \eqref{qge_psi_2}), we will consider the QGE
\eqref{qge_psi_1} with $\psi$ replaced with $-\psi$:
\begin{align}
  -\frac{\partial \left[ \Delta \psi \right]}{\partial t}
    + Re^{-1} \, \Delta^2 \psi + J(\psi , \Delta \psi)
    - Ro^{-1} \, \frac{\partial \psi}{\partial x} = Ro^{-1} \, F .
    \label{eqn:QGE_psi}
\end{align}

At this point, let us comment on the significance of the two parameters in
\eqref{eqn:QGE_psi}, the Reynolds number $Re$ and the Rossby number $Ro$.  As in
the 2D NSE case, $Re$ is the coefficient of the diffusion term $- \Delta q =
\Delta^2 \psi$.  The higher $Re$, the smaller the magnitude of the diffusion
term as compared with the nonlinear convective term $- J(\psi, \Delta \psi) = (
\nabla \psi^{\perp} \cdot \nabla ) q = (\mathbf{u} \cdot \nabla ) q$.  Since
$Ro$, quantifying the rotation effects in the QGE, does not appear in the 2D
NSE, its significance deserves a special attention.  We first note that, for
small $Ro$, which corresponds to large rotation effects, the forcing term
$Ro^{-1} \, F$ becomes large as compared with the other terms.  But probably the
most interesting term in \eqref{eqn:QGE_psi} is $\displaystyle Ro^{-1} \,
\frac{\partial \psi}{\partial x}$, which could be interpreted as a convection
type term with respect to $\psi$, not to $q = -\Delta \psi$.  When $Ro$ is
small, $\displaystyle Ro^{-1} \, \frac{\partial \psi}{\partial x}$ becomes
large.  In conclusion, the physically relevant cases for large scale oceanic
flows, in which $Re$ is large and $Ro$ is small (i.e., small diffusion and high
rotation, respectively) translate mathematically into a
\emph{convection-dominated} PDE with \emph{large forcing}.  Thus, from a
mathematical point of view, we expect the restrictive conditions used to prove
the well-posedness of the 2D NSE \cite{Girault79,Girault86,Gunzburger89} will be
even more restrictive in the QGE setting, due to the rotation effects.  We will
later see that this is indeed the case.

To completely specify the equations in \eqref{eqn:QGE_psi}, we need to impose
boundary conditions.  The question of appropriate boundary conditions for the
QGE is a thorny one, especially for the vorticity-streamfunction formulation
(see \cite{Vallis06,Cummins} for a careful discussion of this issue).  In this
report, we consider
\begin{equation}
  \psi = \frac{\partial \psi}{\partial \mathbf{n}} = 0 \qquad \text{on } \partial \Omega
  \label{eqn:QGEBCs}
\end{equation}
\begin{equation}
  \psi(0) = \psi_0 \qquad \text{when } t = 0
  \label{eqn:QGEICs}
\end{equation}
which are also the boundary conditions used in \cite{Gunzburger89} for the
streamfunction formulation of the 2D NSE.

