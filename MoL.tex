The \emph{method of lines} refers to the application of the \emph{finite
difference} method to a semi-discretization of an equation, i.e. the spatial
domain has been discretized by the finite element method and then the time
domain is discretized using a finite difference method. This is the preferable
method for discretization of the time domain, due to its simplicity and the fact
that the time domain does not have geometry.

In particular, we may apply a a \emph{backward Euler} finite difference scheme
to \eqref{eqn:SemiDiscretization} which would produce the following
discretization of the streamfunction form of the QGE
\begin{equation}
  a_0(\psi_h^{n+1} - \psi_h^n, \chi_h) + k\, \left[a_1(\psi_h^n,\chi_h) + b(\psi_h^n,\psi_h^n,\chi_h)
      + (\psi_x_h^n,\chi_h)\right] = k\, \ell^n(\chi_h),\quad \forall \, \chi_h \in X^h.
  \label{eqn:BEQGE}
\end{equation}
where $k$ is the time step, subscripts represent the discretization in spatial
domains, and superscripts represent the discretization in time domain.
Additionally, fince the linear form $(F,\chi_h)$ is, in fact, time dependent
we place a superscript to indicate the discretization of the $\ell$ in the time
domain.  The attractiveness of using this backward Euler scheme is that it is
explicit and therefore one need not solve the nonlinear system that would arize
from the \emph{forward Euler} scheme. However, it is well known that the
backward Euler scheme is not a-stable and therefore one needs to be careful to
satisfy a CFL type condition.

Here in lies the problem with using a high order finite element scheme, such as
the Argyris element, and the method of lines; the backward Euler scheme is first
order in time whereas the Argyris element is sixth order in the $L^2$ norm.
Thus, we might expect a CFL condition of the form
\begin{equation*}
  \frac{k}{h^6} \le C
\end{equation*}
where $C$ is a dimensionless constant that does not depend on either $k$ or $h$.
Thus, we might expect to need quite a small time step, which may be unreasonably
small for numerical simulations.  Therefore, either a higher order finite
difference scheme will need to be used, or an explicit finite difference scheme
will need to be employed.

While an implicit scheme might seem an attactive option, one must take into
account the ammount of time required to solve the nonlinear system at each time
step. For larger spatial discretizations the time to solve such nonlinear
systems might be prohibitive. Therefore, higher order finite difference schemes
appear to be the more attractive approach.

However, there may be hope for the explicit finite element schemes after all. It
might be possible to apply the Two-level method (\autoref{alg:TwoLevel}) to the
QGE at each time step. That is we solve the nonlinear system on a coarse mesh
and then use that solution to linearize the nonlinear system on the fine mesh.
This method may allow for much quicker calculations than might be expected using
just an explicit finite difference scheme alone. We expect that this method will
produce good results.
