Although the FEM discretization of the QGE is relatively scarce, the
corresponding error analysis seems to be even more scarce. To our knowledge,
\emph{all} the error analysis for the FEM discretization of the QGE has been
done for the vorticity-streamfunction formulation, none being done for the
streamfunction formulation. Furthermore, to the best of our knowledge, all the
available error estimates for the FEM discretization of the QGE are
\emph{suboptimal}. The first FEM error analysis for the FEM discretization of
the QGE was carried out by Fix \cite{Fix}, in which suboptimal error estimates
for the vorticity-streamfunction formulation were proved. Indeed, relationships
(4.7) and (4.8) (and the discussion above these) in \cite{Fix} show that the FEM
approximations for \emph{both} the potential vorticity (denoted by $\zeta$) and
streamfunction (denoted by $\psi$) consist of piecewise polynomials of degree
$k-1$. At the top of page 381, the author concludes that the error analysis
yields the following estimates:
\begin{eqnarray}
  \| \psi - \psi^h \|_1 &=& O(h^{k-1}), \label{eqn:fix_1} \\
  \| \zeta - \zeta ^h \|_0 &=& O(h^{k-1}) . \label{eqn:fix_2}
\end{eqnarray}
Although the streamfunction error estimate \eqref{eqn:fix_1} appears to be
optimal, the potential vorticity error estimate \eqref{eqn:fix_2} is clearly
suboptimal. Indeed, using piecewise polynomials of degree $k-1$ for the FEM
approximation of the vorticity, one would expect an $O(h^k)$ error estimate in
the $L^2$ norm. Medjo \cite{Medjo99, Medjo00} used a FEM discretization of the
vorticity-streamfunction formulation and proved error estimates for the time
discretization, but no error estimates for the spatial discretization. Finally,
Cascon et al. \cite{Cascon} proved both \emph{a priori} and \emph{a posteriori}
error estimates for the FEM discretization of the \emph{linear Stommel-Munk}
model (see Section~\ref{ch:Tests} for more details). This model, while similar
to the QGE, has one significant difference: the linear Stommel-Munk model is
linear, whereas the QGE is nonlinear.
%Thus, it appears that \emph{no optimal} error estimates for the FEM
%discretization of the QGE exist.

We note that the state-of-the-art in the FEM error analysis for the QGE seems to
reflect that for the \emph{two-dimensional Navier-Stokes equations (2D NSE)}, to
which the QGE are similar in form. {\color{red} Include QGE is not NSE plots.} Indeed, as carefully discussed in
\cite{Gunzburger89}, the 2D NSE in streamfunction-vorticity formulation are easy
to implement (only $C^0$ elements are needed for a conforming discretization),
but the available error estimates are suboptimal (see Section 11.6 in
\cite{Gunzburger89}). Next, we summarize the discussion in \cite{Gunzburger89},
since we believe it sheds light on the QGE setting. For $C^0$ piecewise
polynomial of degree $k$ FEM approximation for \emph{both} the vorticity
(denoted by $\omega$) and streamfunction (denoted by $\psi$), the error
estimates given in \cite{Girault86} are (see (1.26) in \cite{Gunzburger89}):
\begin{eqnarray}
  | \psi - \psi^h |_1 + \| \omega - \omega^h \|_0 \leq C \, h^{k - 1/2} \, | \ln h |^{\sigma} ,
  \label{eqn:gunzburger_1}
\end{eqnarray}
where $\sigma = 1$ for $k = 1$ and $\sigma = 0$ for $k > 1$. It is noted in
\cite{Gunzburger89} that the error estimate in \eqref{eqn:gunzburger_1} is not
optimal: one may loose a half power in $h$ for the derivatives of the
streamfunction (i.e., for the velocity), and three-halves power for the
vorticity. It is also noted that there is computational and theoretical evidence
that \eqref{eqn:gunzburger_1} is not sharp with respect to the streamfunction
error. Furthermore, in \cite{Fix84} it was shown that, for the \emph{linear}
Stokes equations, the derivatives of the streamfunction are essentially
optimally approximated (see (11.27) in \cite{Gunzburger89}):
\begin{eqnarray}
  | \psi - \psi^h |_1 \leq C \, h^{k - \varepsilon} , \label{eqn:gunzburger_2}
\end{eqnarray}
where $\varepsilon = 0$ for $k > 1$ and $\varepsilon > 0$ is arbitrary for $k =
1$. That being said, it is then noted in \cite{Gunzburger89} that
\eqref{eqn:gunzburger_1} seems to be sharp for the vorticity error and thus
vorticity approximations are, in general, very poor.
