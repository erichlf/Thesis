The rest of the paper is organized as follows: \autoref{sec:QGE} presents the QGE, 
\autoref{sec:SQGE} presents the stationary QGE, and \autoref{sec:LES} presents the filter QGE and
a discussion on the appropriate closure model. Then \autoref{ch:WellPosed} will discusses the well
posedness of both the QGE and SQGE with special attention paid to the weak formulations and the
mathematical support for the weak formulation presented in \autoref{sec:QGEWeak} and
\autoref{sec:SQGEWeak} for the QGE and SQGE, respectively. Additionally, the weak formulation of the
filtered QGE are presented in \autoref{sec:LESWeak}. In \autoref{ch:FEM} we outline the FEM
discretization for the QGE (\autoref{sec:QGEFEM}), SQGE (\autoref{sec:SQGEFEM}, and Filtered QGE
(\autoref{sec:LESFEF}), with special emphasis placed on the Argyris element in
\autoref{sec:Argyris}. Additional, a discussion of the Two-Level method is presented in
\autoref{sec:TwoLevel}. Rigorous error estimates for the FE discretization and the Two-Level method
applied to the stationary QGE are derived in \autoref{sec:SQGEErrors} and \autoref{sse:SQGE2LE},
respectively. Several numerical experiments supporting the theoretical results for the SQGE and the
Two-Level method are presented in \autoref{sec:SQGETests} and \autoref{sec:SQGE2LTests}. Finally,
conclusions and our future research directions are included in \autoref{ch:Future}.



