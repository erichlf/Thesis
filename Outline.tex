The rest of the paper is organized as follows: \autoref{sec:QGE} presents the
QGE, and \autoref{sec:SQGE} presents the stationary QGE. Then
\autoref{ch:WellPosed} discusses the mathematical framework of both the QGE
and SQGE.  In \autoref{ch:FEM} we outline the FE discretization for the QGE
(\autoref{sec:QGEFEM}), and SQGE (\autoref{sec:SQGEFEM}), with special emphasis
placed on the Argyris element in \autoref{sec:Argyris}. Additionally, a discussion
of the two-level method is presented in \autoref{sec:TwoLevel}. Rigorous error
estimates for the FE discretization of the SQGE, the two-level method applied
to the SQGE, and the QGE are derived in \autoref{sec:SQGEErrors},
\autoref{sse:SQGE2LE}, and \autoref{sec:QGEError}, respectively.  Several
numerical experiments supporting the theoretical results for the QGE, SQGE and
the two-level method are presented in \autoref{sec:QGETests},
\autoref{sec:SQGETests}, and \autoref{sec:SQGE2LTests}. These numerical
experiments also tackle geophysical flows in realistic complex geometries with
realistic parameter values. Finally, conclusions are found in
\autoref{ch:Conclusions}.

